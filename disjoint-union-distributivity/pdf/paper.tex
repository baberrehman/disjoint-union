\documentclass[a4paper]{article}

\usepackage{geometry}
\geometry{left=2.5cm,right=2.5cm,top=2.5cm,bottom=2.5cm}

% Basics
\usepackage{fixltx2e}
\usepackage{url}
\usepackage{fancyvrb}
\usepackage{mdwlist}  % Miscellaneous list-related commands
\usepackage{xspace}   % Smart spacing
\usepackage{supertabular}

% https://www.nesono.com/?q=book/export/html/347
% Package for inserting TODO statements in nice colorful boxes - so that you
% won’t forget to fix/remove them. To add a todo statement, use something like
% \todo{Find better wording here}.
\usepackage{todonotes}

%% Math
\usepackage{amsmath}
\usepackage{amsthm}
\usepackage{amssymb}
\usepackage{bm}       % Bold symbols in maths mode

% http://tex.stackexchange.com/questions/114151/how-do-i-reference-in-appendix-a-theorem-given-in-the-body
\usepackage{thmtools, thm-restate}

%% Theoretical computer science
\usepackage{stmaryrd}
\usepackage{mathtools}  % For "::=" ( \Coloneqq )

%% Code listings
\usepackage{listings}

\usepackage{listings}
\usepackage{xcolor}

\definecolor{codegreen}{rgb}{0,0.6,0}
\definecolor{codegray}{rgb}{0.5,0.5,0.5}
\definecolor{codepurple}{rgb}{0.58,0,0.82}
\definecolor{backcolour}{rgb}{0.95,0.95,0.92}

\lstdefinestyle{mystyle}{
    %backgroundcolor=\color{backcolour},   
    commentstyle=\color{codegreen},
    keywordstyle=\color{magenta},
    numberstyle=\tiny\color{codegray},
    stringstyle=\color{codepurple},
    basicstyle=\ttfamily\footnotesize,
    breakatwhitespace=false,         
    breaklines=true,                 
    captionpos=b,                    
    keepspaces=true,                 
    numbers=left,                    
    numbersep=5pt,                  
    showspaces=false,                
    showstringspaces=false,
    showtabs=false,                  
    tabsize=2
}

\lstset{style=mystyle}

%% Font
\usepackage[euler-digits,euler-hat-accent]{eulervm}


\usepackage{ottalt}

\renewcommand\ottaltinferrule[4]{
  \inferrule*[right={#1}]
    {#3}
    {#4}
}


% Ott includes
\inputott{ott-rules}


\title{Rules Version 2 (Disjoint Union Type)}
\author{Baber}
\begin{document}

\maketitle

\begin{align*}
&Type &A, B&::= [[Top]] ~ | ~[[Bot]]~|~ Int ~|~| ~ [[A->B]]~ |~ [[A \/ B]] ~ |~ [[A /\ B]]\\
&Expr &e &::= x ~|~ n ~| ~| ~ e:A ~|~[[\x.e]] ~ | ~ e1e2 ~|~ [[typeof e as A e1 B e2]]\\
&PExpr & p &::= n ~| ~ b ~ | ~ s ~|~ [[\x.e : A->B]] \\
%&RExpr & r &::= e1e2 ~|~ [[typeof e as A e1 B e2]] \\
%&UExpr & u &::= r ~|~ p ~ |~ u : A \\
&Value &v &::= p : A \\
&Context & [[G]] &::= empty~ |~ [[G , x : A]]
\end{align*}

%\begin{align*}
%&Isomorphic & A \sim B & ::= [[A <: B]] \wedge [[B <: A]]
%\end{align*}


%\begin{align*}
%&BottomLikeSpec & C & ::= (\forall A ~ B, ~ [[A /\ B]] \sim C \rightarrow \neg ( [[ A <: B ]] ) \wedge \neg ( [[ B <: A ]] )) \vee ([[C <: Bot]])
%\end{align*}

%\begin{align*}
%&DisjSpec & A * B & ::= \forall C, [[C <: A]] \wedge [[C <: B]] \rightarrow  \rfloor [[C]] \lfloor
%\end{align*}


%\begin{table}
     {\renewcommand{\arraystretch}{1.5}
     \begin{center}
     \begin{tabular}{|lcll|}
%       \hline
%      Isomorphic & A $\backsimeq$ B & ::= & $[[A <: B]] \wedge [[B <: A]]$ \\
    %   \hline
    %  BottomLikeSpec & $[[BottomLikeSpec C]]$ & ::= & $\forall A, ~ Ord ~ A ~ \rightarrow \neg [[A <: C]]$ \\
%      BottomLikeSpec & C & ::= & $(\exists A ~ B, ~ ([[A /\ B]]) \backsimeq C \rightarrow \neg ( [[ A <: B ]] ) \vee \neg ( [[ B <: A ]] )) \vee ([[C <: Bot]])$ \\
       \hline
      DisjSpec & A $*_s$ B & ::= & ~$\forall C, ~ Ord ~ C ~ \rightarrow \neg [[C <: A /\ B]]$\\
       \hline
      DisjAlgo & A $*_a$ B & ::= & ~ $(FindSubtypes ~ A) ~ `inter` ~ (FindSubtypes ~ B) = []$ \\
       \hline
     \end{tabular}
     \end{center} }
%\end{table}

\bigskip

\begin{lstlisting}[language=Haskell]
Fixpoint FindSubtypes (A: typ) :=
    match A with
    | t_top         => [t_int; t_arrow t_top t_bot; t_top]
    | t_bot         => []
    | t_int         => [t_int]
    | t_arrow A1 B1 => [t_arrow t_top t_bot]
    | t_union A1 B1 => (FindSubtypes A1) `union` (FindSubtypes B1)
    | t_and A1 B1   => (FindSubtypes A1) `inter` (FindSubtypes B1)
    end.
\end{lstlisting}

\bigskip

\ottdefnsOrdinary

%\ottdefnsBottomLike

%\ottdefnsDisjointness

\ottdefnsSubtyping

\ottdefnsTyping

\ottdefnsReduction


\end{document}
