\documentclass[a4paper]{article}

\usepackage{geometry}
\geometry{left=2.5cm,right=2.5cm,top=2.5cm,bottom=2.5cm}

% Basics
\usepackage{fixltx2e}
\usepackage{url}
\usepackage{fancyvrb}
\usepackage{mdwlist}  % Miscellaneous list-related commands
\usepackage{xspace}   % Smart spacing
\usepackage{supertabular}

% https://www.nesono.com/?q=book/export/html/347
% Package for inserting TODO statements in nice colorful boxes - so that you
% won’t forget to fix/remove them. To add a todo statement, use something like
% \todo{Find better wording here}.
\usepackage{todonotes}

%% Math
\usepackage{amsmath}
\usepackage{amsthm}
\usepackage{amssymb}
\usepackage{bm}       % Bold symbols in maths mode

% http://tex.stackexchange.com/questions/114151/how-do-i-reference-in-appendix-a-theorem-given-in-the-body
\usepackage{thmtools, thm-restate}

%% Theoretical computer science
\usepackage{stmaryrd}
\usepackage{mathtools}  % For "::=" ( \Coloneqq )

%% Code listings
\usepackage{listings}

%% Font
\usepackage[euler-digits,euler-hat-accent]{eulervm}


\usepackage{ottalt}

\renewcommand\ottaltinferrule[4]{
  \inferrule*[right={#1}]
    {#3}
    {#4}
}


% Ott includes
\inputott{ott-rules}


\title{Rules Version 2 (Disjoint Union Type)}
\author{Baber}
\begin{document}

\maketitle

\begin{align*}
&Type &A, B&::= [[Top]] ~ | ~[[Bot]]~| Int ~|~ [[A->B]]~ |~ [[A \/ B]]\\
&Expr &e &::= x ~|~ n ~|~ e:A ~|~[[\x.e]] ~ | ~ e1e2 ~|~ [[typeof e as A e1 B e2]]\\
&PExpr & p &::= n ~|~ [[\x.e : A->B]] \\
%&RExpr & r &::= e1e2 ~|~ [[typeof e as A e1 B e2]] \\
%&UExpr & u &::= r ~|~ p ~ |~ u : A \\
&Value &v &::= p : A \\
&Context & [[G]] &::= empty~ |~ [[G , x : A]]
\end{align*}

\begin{align*}
&DisjSpec & A * B & ::= \forall C, [[C <: A]] \wedge [[C <: B]] \rightarrow  \rfloor [[C]] \lfloor
\end{align*}

\bigskip

\ottdefnsBottomLike

\ottdefnsDisjointness

\ottdefnsSubtyping

\ottdefnsTyping

\ottdefnsReduction


\end{document}
