\section{Discussion}
\label{sec:discussion}
In this section we discuss some extensions and provide a comparision with the approach
adopted in Ceylon. 

\subsection{A More General Subtyping Rule for Bottom  Types}
\label{sec:inter:refactoring}
As discussed in Section~\ref{sec:inter}, Ceylon includes the following subtyping rule:

\begin{center}
\drule[]{s-dis}
\end{center}

%One property of Ceylon programming language ~\cite{} is that all \emph{bottom-like} types are 
%subtype of $[[Bot]]$ type.
\noindent It is possible to support, and in fact generalize, such as rule in \name. The idea is to
employ our definition of lowest ordinary subtypes, and add the following rule 
%This property also holds in \cal presented in \Cref{sec:union}. But this property does not
%hold in \cal with intersection types presented in \Cref{sec:inter} because we updated
%disjointness definition. We add following novel subtyping rule in subtyping of
%\Cref{sec:inter} to have this property
in \cal with intersection types:

\begin{center}
\drule[]{s-disj}
\end{center}
\bruno{Be consistent and use the empty set notation instead of the empty list. Also, call it S-Los.}

\noindent \Rref{s-disj} is an interesting addition in subtyping of \cal with intersection types.
It says that if LOS returns the empty set for some type $[[A]]$, then $[[A]]$
is subtype of all types. In other words, such type behaves like \emph{bottom-like} type.
Such rule generalizes the rule employed in Ceylon, since when $[[A]]$ is an intersection type of two
disjoint types, we get the empty set.
%Since $[[findsubtypes Int /\ A -> B]]$ returns empty set,
%therefor \rref{s-disj} makes $[[Int /\ A -> B]]$ subtype of all types.
Moreover, adding \rref{s-disj} makes \rref{s-bot} redundant as well, since the LOS for
the bottom type is also the empty set. Therefore \rref{s-bot} can safely be dropped without any
any loss of expressiveness. It is trivial to prove a lemma which says that $[[Bot]]$ type is
subtype of all types. We drop \rref{s-bot} from the calculus discussed in \Cref{sec:inter}
and prove \Cref{lemma:discussion:bls} to show this property instead. It is because that $[[findsubtypes Bot]]$
returns empty set (\{\}) and \rref{s-disj} makes all such types with $[[findsubtypes A]]$ = \{\} as
subtype of all types.
\bruno{You should also show a lemma: $A * B$ implies $A \& B <: Bot$ and adjust the text}

\begin{lemma}[Bottom Type Least Subtype]
  $[[Bot <: A]]$.
\label{lemma:discussion:bls}
\end{lemma}

\noindent The use of \rref{s-disj} instead of \rref{s-dis} also has the advantage that it does not
create a mutual dependency between disjointness and subtyping. We can have a the definition
of disjointness, which depends only on subtyping and ordinary types, and the definition
of subtyping, which depends on LOS but not on disjointness.

We have formalized and proved all the metatheory, including type-soundness, transitivity of subtyping,
soundness and completeness of disjointness and determinism for a variant of \name with intersection types
and \rref{s-disj}.

\subsection{Subtyping Distributivity}
\label{sec:inter:dist}

\bruno{I think this is lacking some details here: we should show some examples
  that require distributivity, and explain those examples, which illustrate the
  distributivity rules with some care. Also say something about disjointness. In particular
  explain that distributivity does not affect disjointness and that the current formulations
remain sound and complete.}

Another interesting extension is to add distributive subtyping to \cal.
Ceylon~\cite{} employs an enriched distributive subtyping relation \cite{}.
Distributive subtyping is more expressive and accomodates more programs than what we discussed in
\Cref{sec:inter}. Distributivity is easy to add to \cal and, the resulting 
calculus preserves soundness, completeness, type-safety and determinism. 

\baber{perhaps an example to show expressiveness of DS.}

\Cref{fig:discussion:ds} shows declarative version of distributive subtyping for \cal with intersection types.
This subtyping relation is declarative as it includes axioms for subtyping reflexivity (\rref{ds-refl}) and
subtyping transitivity (\rref{ds-trans}). \Rref{ds-top, ds-btm, ds-int, ds-arrow, ds-ora} have been
discussed in \Cref{sec:union}. \Rref{ds-anda} has been discussed in \Cref{sec:inter}.
\Rref{ds-orb, ds-orc} state that sub-part of a union type is a subtype of whole union type.
\Rref{ds-andb, ds-andc} state that intersection type $[[A1 /\ A2]]$ is a subtype of both $[[A1]]$
and $[[A2]]$ separately.\bruno{Snow can't we drop some of the distributivity rules? Some of them
are redundant, right?}

\Rref{ds-distarr, ds-distarrrev}  distribute function types over intersection types.
They state that $[[(A -> B1) /\ (A -> B2)]]$ and $[[A -> (B1 /\ B2)]]$ are equivalent types.
\Rref{ds-distarru, ds-distarrurev} state that $[[(A1 -> B) /\ (A2 -> B)]]$ and $[[(A1 \/ A2) -> B]]$
are equivalent types.
\Rref{ds-distor, ds-distand} distribute intersection types over union types and vice versa.


\begin{figure}[t]
  %\centering
  \drules[ds]{$ [[A <<: B ]] $}{Declarative Subtyping}{refl, trans, top, btm, int, arrow, ora, orb, orc, anda, andb, andc, distarr, distarrrev, distarru, distarrurev, distor, distand}
  \caption{Declarative Subtyping}
  \label{fig:discussion:ds}
\end{figure}
