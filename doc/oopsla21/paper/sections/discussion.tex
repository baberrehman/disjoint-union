\section{Discussion}
\label{sec:discussion}
In this section we ...

\subsection{Refactoring Subtyping}
\label{sec:inter:refactoring}

\Rref{s-disj} is a significant and novel addition in subtyping of \cal 
with intersection types. It primarily
makes intersection of disjoint types subtype of all types.
Because two types are disjoint only if they do not share any common ordinary subtype.

\Rref{s-disj} is an interesting addition in subtyping of \cal with intersection types.
It says that if $FindSubTypes$ operation returns empty set against some type $[[A]]$, then $[[A]]$
is subtype of all types. In other words, such type behaves like \emph{bottom-like} type.
One particularly interesting case of \rref{s-disj} is when $[[A]]$ is an intersection type of two
disjoint types such as $[[Int /\ A -> B]]$. Since $[[findsubtypes Int /\ A -> B]]$ returns empty set,
therefor \rref{s-disj} makes $[[Int /\ A -> B]]$ subtype of all types.

\paragraph{Dropping Subtyping Rule S-BOT}
Another particular interesting case of \rref{s-disj} is that it makes \rref{s-bot} redundant 
and \rref{s-bot} can safely be dropped without any significant
impact on metatheory and implementation. It is trivial to prove a lemma which says $[[Bot]]$ type is
subtype of all types. We drop \rref{s-bot} from the calculus discussed in this section
and prove \Cref{lemma:inter:bls} to show this property instead. It is because that $[[findsubtypes Bot]]$
returns empty set \{\} and \rref{s-disj} makes all such types with $[[findsubtypes Bot]]$ = \{\} as
subtype of all types.

\begin{lemma}[Bottom Type Least Subtype]
  $[[Bot <: A]]$.
\label{lemma:inter:bls}
\end{lemma}