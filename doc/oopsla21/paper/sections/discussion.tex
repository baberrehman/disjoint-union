\section{Discussion}
\label{sec:discussion}
In this section we ...
\baber{and then there would be some discussion section, 
where we would discuss the variant with the more general rule for bottom subtyping (with findtypes), 
and maybe some other things, like function disjointness and distributivity.
we would connect this discussion (the new bottom rule) with Ceylon and the disjointness rule that they use there}

\baber{I think that we can avoid mentioning the algorithmic system.
For the paper we are essentially not be mentioning those lemmas, or maybe just 
have a few quick sentences about this (since they are auxiliary lemmas anyway, 
and their proofs are not a contribution of this paper: such credit goes to @snow.
ok, so, yes, in Section 5 you’ll present the declarative subtyping rules.}

\subsection{Refactoring Subtyping}
\label{sec:inter:refactoring}

One property of Ceylon programming language ~\cite{} is that all \emph{bottom-like} types are 
subtype of $[[Bot]]$ type.
This property also holds in \cal presented in \Cref{sec:union}. But this property does not
hold in \cal with intersection types presented in \Cref{sec:inter} because we updated
disjointness definition. We add following novel subtyping rule in subtyping of
\Cref{sec:inter} to have this property in \cal with intersection types:

\begin{center}
\drule[]{s-disj}
\end{center}

\noindent \Rref{s-disj} is an interesting addition in subtyping of \cal with intersection types.
It says that if Lowest Ordinary Subtypes operation returns empty set against some type $[[A]]$, then $[[A]]$
is subtype of all types. In other words, such type behaves like \emph{bottom-like} type.
One particularly interesting case of \rref{s-disj} is when $[[A]]$ is an intersection type of two
disjoint types such as $[[Int /\ A -> B]]$. Since $[[findsubtypes Int /\ A -> B]]$ returns empty set,
therefor \rref{s-disj} makes $[[Int /\ A -> B]]$ subtype of all types.

\paragraph{Dropping Subtyping Rule S-BOT}
Another particular interesting case of \rref{s-disj} is that it makes \rref{s-bot} redundant 
and \rref{s-bot} can safely be dropped without any significant
impact on metatheory and implementation. It is trivial to prove a lemma which says $[[Bot]]$ type is
subtype of all types. We drop \rref{s-bot} from the calculus discussed in \Cref{sec:inter}
and prove \Cref{lemma:discussion:bls} to show this property instead. It is because that $[[findsubtypes Bot]]$
returns empty set (\{\}) and \rref{s-disj} makes all such types with $[[findsubtypes A]]$ = \{\} as
subtype of all types.

\begin{lemma}[Bottom Type Least Subtype]
  $[[Bot <: A]]$.
\label{lemma:discussion:bls}
\end{lemma}

\subsection{Subtyping Distributivity}
\label{sec:inter:dist}

\baber{Ceylon has distributivity ...., distributivity is more flexible than subtyping in section 4 ....
some examples with and without distributivity ....}

\begin{figure}[t]
  %\centering
  \drules[ds]{$ [[A <<: B ]] $}{Declarative Subtyping}{refl, trans, top, btm, int, arrow, ora, orb, orc, anda, andb, andc, distarr, distarrrev, distarru, distarrurev, distor, distand}
  \caption{Declarative Subtyping}
  \label{fig:discussion:subtyping}
\end{figure}