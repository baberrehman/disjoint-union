\section{Conclusion and Future Work}
\label{sec:conclusion}

This work develops the union calculus (\cal) with union types and a
type-based union elimination construct based on disjointness. We presented the
type-directed semantics of the calculus, and showed that it is
type-sound and deterministic. Disjointness plays a crutial role for
the determinism result, as it ensures that only one branch in the
switch elimination contruct can apply for any given value. A nice
aspect of the work was that we were able to adapt the notion of
disjointness used in disjoint intersection types to our variant of
\name with union types. We believe that this reinforces fundamental
connections between union and intersection types via duality.  The
addition of intersection types to \name lead to some interesting
discoveries.  In particular, it showed that the notion of disjointness
that we were able to formulate, inspired by the work on disjoint
intersection types, breaks.  This is not showing that the duality stops
working. Instead, it shows that the combination of intersections
and unions in the same system affects disjointness. None of the
previous works on disjoint intersection types included union
types. However, as discussed in Section~\ref{sec:related}, adding
union types to disjoint intersection types leads to a similar problem,
and the solution in \name can inspire solutions for adding union types to
disjoint intersection types.

We plan to extend \cal for practical programming languages with more
advanced features. An interesting line of research is adding polymorphism.
\emph{Disjoint polymorphism}~\cite{alpuimdisjoint} has been studied for
calculi with disjoint intersection
types, and we believe a similar notion can be adopted in calculi like \name.
Another research direction for \cal is to study
the addition of the merge operator, which calculi with disjoint
intersection types include. The main challenge is that types such as
$[[Int /\ Bool]]$ become inhabited. 
Therefore, special care has to be taken
to study \cal with a merge operator.
It could also be interesting to to study a variant of \name that uses
a best-match approach based on the dynamic type. This would relate
to the extensive line of research on \emph{multi-methods}~\cite{Chambers92}
and \emph{multiple dispatching}~\cite{CliftonLCM2000}. 


\begin{comment}
Another domain of future interest
is to embed dynamic dispatch in \cal by defining a weak and more
expressive disjointness. Dynamic dispatch will allow type checking
more programs.
\end{comment}
