\section{\name (\cal)}
\label{sec:union}

This section discusses a simple version of \name and without intersection types. We propose a type-safe,
coherent, sound, complete and deterministic calculus for \name in this section.

%%%%%%%%%%%%%%%%%%%%%
%% Syntax
%%%%%%%%%%%%%%%%%%%%%

\subsection{Syntax}
\label{sec:union:syntax}
\Cref{fig:union:syntax} shows the syntax for \cal. Metavariables $[[A]]$ and $[[B]]$ range over types.
Types include top ($[[Top]]$), bottom ($[[Bot]]$), function ($[[A -> B]]$) and union ($[[A \/ B]]$) types.
Metavariable $[[e]]$ ranges over program expressions. Expressions include variables ($[[x]]$), natural
numbers ($[[i]]$), type annotations ($[[e:A]]$), lambda abstractions ($[[\x.e]]$), applications 
($[[e1 e2]]$) and a novel typeof ($[[typeof e as A e1 B e2]]$) expression. \Typeof expression is 
like a case expression and evaluates a specific branch by matching the type. Details of \typeof 
expression will further be discusse in typing and operational semantics sections.

Metavariable $[[p]]$ ranges over partial expressions which are also referred as partial values.
It consists of natural numbers $[[i]]$ and annotated lambda expressions $[[\x.e : A -> B]]$.
Values are annotated partial expressions. Metavariable $[[v]]$ ranges over values. It is important to
note that $[[i]]$ is not a value in this calculus, instead $[[i:Int]]$ is a value.
Context ($[[G]]$) can either be empty or contains type bindings of variables and associated types. 

\begin{figure}[t]
  \begin{small}
    \centering
    \begin{tabular}{lrcl} \toprule
      Types & $[[A]], [[B]]$ & $\Coloneqq$ & $ [[Top]] \mid [[Bot]] \mid [[Int]] \mid [[A -> B]] \mid [[A \/ B]] $ \\
      Expr & $[[e]]$ & $\Coloneqq$ & $[[x]] \mid [[i]] \mid [[e:A]] \mid [[\x.e]] \mid [[e1 e2]] \mid [[typeof e as A e1 B e2]]$\\
      PExpr & $[[p]]$ & $\Coloneqq$ & $[[i]] \mid [[\x.e : A -> B]] $\\
      Value & $[[v]]$ & $\Coloneqq$ & $[[p:A]]$\\
      Context & $[[G]]$ & $\Coloneqq$ & $ \cdot \mid [[G , x : A]]$ \\
      \bottomrule
    \end{tabular}
  \end{small}
  \begin{small}
    \centering
    \drules[s]{$ [[A <: B ]] $}{Algorithmic Subtyping}{top, bot, int, arrow, ora, orb, orc}
  \end{small}
  \caption{Syntax and Algorithmic Subtyping for \cal.}
  \label{fig:union:syntax}
\end{figure}

%%%%%%%%%%%%%%%%%%%%%
%% Subtyping
%%%%%%%%%%%%%%%%%%%%%
\subsection{Subtyping}
\label{sec:union:sub}
Algorithmic subtyping rules for \cal are shown in \Cref{fig:union:syntax}. Subtyping rules are standard
for a system with union types.
\Rref{s-top} states that all types are subtypes of $[[Top]]$ type. \Rref{s-bot} states that $[[Bot]]$ type 
is subtype of all types. \Rref{s-int, s-arrow} are standard rules for integers and functions respectively.
Functions are contravariant in input types and covariant in output types. \Rref{s-ora, s-orb, s-orc} are
standard subtyping rules for union types. Union type of two types $A1$ and $A2$ is a subtype of another
type $A$ if both $A1$ and $A2$ are subtype of $A$, as stated in \rref{s-ora}. \Rref{s-orb, s-orc} state
that if a type is subtype of part of union type, then it is subtype of whole union type.
Subtyping relation for \cal is reflexive and transitive.

\begin{lemma}[Subtyping Reflexivity]
  $[[A <: A]]$.
\label{lemma:union:refl}
\end{lemma}

\begin{comment}
\begin{proof}
  By induction on type A. All cases are trivial to prove.
\end{proof}
\end{comment}

\begin{lemma}[Subtyping Transitivity]
  If \ $[[A <: B]]$ \ and \ $[[B <: C]]$ \ then: \ $[[A <: C]]$.
  \label{lemma:union:trans}
\end{lemma}

\begin{comment}
\begin{proof}
  By induction on type B.
  \begin{itemize}
    \item Cases $[[Top]]$, $[[Bot]]$ and $[[Int]]$ are trivial to prove.
    \item Case $[[A -> B]]$ requires double induction on type $[[C]]$
          and $[[A]]$.
    \item Case $[[A \/ B]]$ requires \Cref{lemma:union:sub-or}
  \end{itemize}
\end{proof}

\begin{lemma}[Subtyping Union Inversion]
\label{lemma:union:sub-or}
  If \ $[[A \/ B <: C]]$ then:
  \begin{enumerate}
    \item $[[A <: C]]$ and
    \item $[[B <: C]]$
  \end{enumerate}
\end{lemma}
\end{comment}


%%%%%%%%%%%%%%%%%%%%%%%
%% Disjointness
%%%%%%%%%%%%%%%%%%%%%%%
\subsection{Disjointness}
\label{sec:union:disj}
\baber{Mainly focused on technical details in this section. Story and benefits of disjointness may be discussed in another section.}
In this section we discuss in detail the notion of disjointness for union types and case expression
for \cal.
Disjointness for \cal is dual to disjointness for $\lambda_i$ \cite{oliveira2016disjoint}.
Two types in $\lambda_i$ are disjoint if they do not share any common \emph{supertype} 
which is not \emph{top-like}.
Whereas, two types in \cal are disjoint if they do not share any common \emph{subtype} which is not
\emph{bottom-like}. We emphasize the significance of \emph{supertypes} and \emph{subtypes} in
$\lambda_i$ and \cal respectively.

\paragraph{Bottom-Like Types}
\emph{Bottom-like} types are integral part of disjoitness in \cal like \emph{top-like} in 
$\lambda_i$ \cite{oliveira2016disjoint}. Therefore,
it is important to understand the notion of \emph{bottom-like} types before diving into the details 
of disjointness. Intuitively, a \emph{bottom-like} type is a type which behaves like $[[Bot]]$ type. 
\emph{Bottom-like} types are shown at the top in \Cref{fig:union:disj-typ}. Type
$[[Bot]]$ is a \emph{bottom-like} type (\rref{bl-bot}) and union type of two \emph{bottom-like} 
types is also a \emph{bottom-like} type (\rref{bl-or}).
It is trivial to conclude that a union type is \emph{bottom-like} only if all 
the primitive types in union are $[[Bot]]$. \emph{Bottom-like} types hold following properties:

\begin{lemma}[Bottom-Like Soundness]
  If \ $[[botlike A]]$ \ then: \ $[[A <: B]]$.
\label{lemma:union:bl-soundness}
\end{lemma}

\begin{comment}
\begin{proof}
  By induction on bottom-like relation.
  \begin{itemize}
    \item All cases are trivial to prove.
  \end{itemize}
\end{proof}
\end{comment}

\begin{lemma}[Bottom-Like Completeness]
  If \ $[[A <: B]]$ \ then: \ $[[botlike A]]$.
\label{lemma:union:bl-completeness}
\end{lemma}

\begin{comment}
\begin{proof}
  By induction on type $[[A]]$.
  \begin{itemize}
    \item Cases $[[Top]]$, $[[Bot]]$, $[[Int]]$ and $[[A -> B]]$ are trivial to prove.
    \item Case $[[A \/ B]]$ requires \Cref{lemma:union:sub-or}.
  \end{itemize}
\end{proof}
\end{comment}

\paragraph{Declarative Disjointness}
Recall that two types in \cal are disjoint if they do not share any common subtype which is not 
\emph{bottom-like}. The formal definition of disjoint specifications for this calculus is:

\begin{definition}
  A $*_s$ B $\Coloneqq$ $\forall$ C, $[[C <: A]]$ $\wedge$ $[[C <: B]]$ $\rightarrow$ $[[botlike C]]$
\label{def:union:disj}
\end{definition}

\noindent Reader may think at this point that $[[Bot]]$ type can simply be used in \Cref{def:union:disj}
instead of $[[botlike C]]$ in the conclusion. Answer to this question is 
union type with $[[Bot]]$ as all primitive types is also a least subtype in \cal. 
$[[botlike C]]$ also handles this case.
We illustrate specifications given in \Cref{def:union:disj} on few simple examples:

\begin{enumerate}
  \item $\boldsymbol{A = [[Int]], \ B = \ [[A -> B]]:}$ \\
        $[[Int]]$ and $[[A -> B]]$ are disjoint types. Common subtypes of $[[Int]]$ and $[[A -> B]]$ 
        include $[[Bot]]$ and union of $[[Bot]]$ types. $[[Bot]]$ and union of $[[Bot]]$ types are
        bottom-like types which make $[[Int]]$ and $[[A -> B]]$ disjoint types as per \Cref{def:union:disj}.
  \item $\boldsymbol{A = [[Int]], \ B = \ [[Bot]]:}$ \\
        $[[Int]]$ and $[[Bot]]$ are disjoint types. Common subtypes of $[[Int]]$ and $[[Bot]]$ 
        include $[[Bot]]$ and union of $[[Bot]]$ types. $[[Bot]]$ and union of $[[Bot]]$ types are
        bottom-like types which make $[[Int]]$ and $[[Bot]]$ disjoint types as per \Cref{def:union:disj}.
        In general, type $[[Bot]]$ is disjoint to all types.
  \item $\boldsymbol{A = [[Int]], \ B = \ [[Int]]:}$ \\ 
        $[[Int]]$ and $[[Int]]$ are not disjoint types because they share a common subtype $[[Int]]$ which
        is not \emph{bottom-like}.
  \item $\boldsymbol{A = [[Int]], \ B = \ [[Top]]:}$ \\
        $[[Int]]$ and $[[Top]]$ are not disjoint types because they share a common
        subtype $[[Int]]$ which is not \emph{bottom-like}. 
\end{enumerate}

\begin{comment}
\begin{figure}[t]
  \begin{small}
    \centering
    \drules[ad]{$[[A * B]]$}{Algorithmic Disjointness}{btmr, btml, intl, intr, orl, orr}
  \end{small}
  \caption{Algorithmic Disjointness for \cal.}
  \label{fig:union:ad}
\end{figure}
\end{comment}

\paragraph{Algorithmic Disjointness}
Middle part of \Cref{fig:union:disj-typ} shows an algorithmic version of disjointness.
\Rref{ad-btmr, ad-btml} state that $[[Bot]]$ type is disjoint to all types.
\Rref{ad-intl, intr} state that $[[Int]]$ and $[[A -> B]]$ are disjoint types.
Algorithmic disjointness can further be scaled to more primitive disjoint types such as
$Bool$ and $String$ by adding more rules similar to \rref{ad-intl, intr} for additional primitive types.
\Rref{ad-orl, ad-orr} are two symmetric rules for union types. Any type $[[C]]$ is disjoint
to union type $[[A \/ B]]$ if $[[C]]$ is disjoint to both $[[A]]$ and $[[B]]$.
Algorithmic disjointness is sound and complete with respect to \Cref{def:union:disj}:

\begin{lemma}[Disjointness Soundness]
  If \ $[[A * B]]$ \ then: \ $[[A *s B]]$.
\label{lemma:union:disj-sound}
\end{lemma}

\begin{comment}
\begin{proof}
  By induction on algorithmic disjointness relation.
  \begin{itemize}
    \item Cases \rref{ad-btmr, ad-btml, ad-orl, ad-orr} require induction on hypothesis
          and \Cref{lemma:union:sub-or}.
    \item Cases \rref{ad-intl, ad-intr} require induction on type and \Cref{lemma:union:sub-or}.
  \end{itemize}
\end{proof}
\end{comment}

\begin{lemma}[Disjointness Completeness]
  If \ $[[A *s B]]$ \ then \ $[[A * B]]$.
\label{lemma:union:disj-complete}
\end{lemma}

\begin{comment}
\begin{proof}
  By induction on type A.
  \begin{itemize}
    \item Case $[[Top]]$ requires \Cref{lemma:union:bl-disj}.
    \item Case $[[Bot]]$ is trivial to prove.
    \item Case $[[Int]]$ requires induction on type B and
          \Cref{lemma:union:bl-disj,lemma:union:disj-sym}.
    \item Case $[[A -> B]]$ requires induction on type B and \Cref{lemma:union:disj-sym}.
    \item Case $[[A \/ B]]$ follows directly from inductive hypothesis.
  \end{itemize}
\end{proof}

\begin{lemma}[Bottom-like Types Disjoint]
\label{lemma:union:bl-disj}
  If \ $[[botlike A]]$ \ then: \ $[[A * B]]$.
\end{lemma}

\begin{lemma}[Disjointness Symmetry]
\label{lemma:union:disj-sym}
  If \ $[[A * B]]$ \ then: \ $[[B * A]]$.
\end{lemma}
\end{comment}

\begin{lemma}[Bottom-Like Disjoint]
  If \ $[[botlike A]]$ \ then: \ $[[A * B]]$.
\label{lemma:union:bl-disjoint}
\end{lemma}

\begin{lemma}[Disjointness Symmetry]
  If \ $[[A * B]]$ \ then: \ $[[B * A]]$.
\label{lemma:union:disj-sym}
\end{lemma}

\baber{Should we add some other disjointness properties? For example symmetry and
bottom-like types are disjoint to all other types?}

\begin{figure}[t]
  \begin{small}
    \centering
    \drules[bl]{$[[botlike A]]$}{Bottom-Like Types}{bot, or}
  \end{small}
  \begin{small}
    \centering
    \drules[ad]{$[[A * B]]$}{Algorithmic Disjointness}{btmr, btml, intl, intr, orl, orr}
  \end{small}
  \begin{small}
    \centering
    \drules[typ]{$ [[G |- e dirflag A]] $}{Bidirectional Typing}{int, var, ann, app, sub, abs, typeof}
  \end{small}
  \caption{Bottom-Like types, Algorithmic Disjointness and Typing for \cal.}
  \label{fig:union:disj-typ}
\end{figure}


%%%%%%%%%%%%%%%%%%%%%
%% Typing
%%%%%%%%%%%%%%%%%%%%%
\subsection{Typing}
\label{sec:union:typ}
Endmost part of \Cref{fig:union:disj-typ} shows typing rules for \cal.
We adopt bidirectional typing in our calculus.
Two typing modes are opted in bidirectional typing. Inference mode ($[[=>]]$) and check mode ($[[<=]]$).
In inference mode, type of an expression $[[e]]$ is inferred or calculated based upon certain information
available in the given context $[[G]]$. 
While in check mode, an expression $[[e]]$ is checked against a given type $[[A]]$.
Typing rules are standard except of \rref{typ-typeof}.
An integer expression $[[i]]$ infers type $[[Int]]$ as stated in \rref{typ-int}.
\Rref{typ-var} states that a variable $[[x]]$ infers type $[[A]]$ if $[[x]]$ has type $[[A]]$ in the 
given context. \Rref{typ-ann} states if an expression $[[e]]$ checks against type $[[A]]$, then the
annotated expression $[[e:A]]$ infers type $[[A]]$.

\Rref{typ-app} type checks a function application and it is the elimination rule for lambda expressions.
Expression $[[e1]]$
has to be a function expression and expression $[[e2]]$ has to check against input type of $[[e1]]$.
An important point to notice in \rref{typ-app} is $[[e1]]$ infers type $[[A -> B]]$. This may look weird
at the very first glance because lamda expressions ($[[\x.e]]$) are not annotated in program expressions 
$[[e]]$ and it seems not possible for lambda expression to infer its type.
To answer this question, we emphasize the use of partial expressions $[[p]]$ and values $[[v]]$.
Lambda expression is annotated in $[[p]]$ and so in $[[v]]$ because values are defined as annotated
partial expressions.

\Rref{typ-sub} is the subsumption rule. It states that an expression $[[e]]$ can be checked 
against any supertype of its inferred type. 
\Rref{typ-abs} is the standard introduction rule for lambda expressions. To check a
lambda expression $[[\x.e]]$ against type $[[A -> B]]$, it is sufficient to check lambda body $[[e]]$
against the output type $[[B]]$ in an extended context with parameter $[[x]]$ of input type $[[A]]$.
\Rref{typ-typeof} is novel typing rule to type check \typeof expressions. Four conditions are necessary
for typing \typeof expressions. Very last condition ($[[A *s B]]$) ensures that branches in case 
expressions are always of disjoint types. Overlapping branches in case expressions lead to non-deterministic
and incoherent results. Disjointness guarrantees determinism and coherence for case expressions. 
Rest of the conditions are standard for a calculus with union types and case expression and have been
studied in various calculi (\baber{reference to calculi}).
Very first condition ($[[G |- e <= A \/ B]]$) ensures that case expression $[[e]]$ is well-typed and checks
against type $[[A \/ B]]$.
Next two conditions ensure that branches of case expression are well-typed and check against some
type $[[C]]$. An important point in these two conditions is that variable $[[x]]$ is of type $[[A]]$
in first branch and of type $[[B]]$ in second branch in the extended context.
And from the last condition $[[A *s B]]$, we guarantee that $[[A]]$ and $[[B]]$ are disjoint types.
Since all the branches check against $[[C]]$, the whole \typeof expression checks against $[[C]]$.

\begin{comment}
\begin{figure}[t]
  \begin{small}
    \centering
    \drules[typ]{$ [[G |- e dirflag A]] $}{Bidirectional Typing}{int, var, ann, app, sub, abs, typeof}
  \end{small}
  \caption{Typing for \cal.}
  \label{fig:union:typ}
\end{figure}
\end{comment}

%%%%%%%%%%%%%%%%%%%%%%%
%% Operations Semantics
%%%%%%%%%%%%%%%%%%%%%%%
\subsection{Operational Semantics}
\label{sec:union:os}
Dynamics of \cal are defined by small-step operational semantics. 
\Cref{fig:union:os} shows
operational semantics of \cal. Before going into the details of operational semantics, it is important
to recall partial expressions and values. An integer expression $[[i]]$ or a lambda expression $[[\x.e]]$
are not values unless annotated. Lambda expression has to be dually annotated to fulfill the criteria to
be a value. Operational semantics follow call-by-value evaluation strategy.
We empahsize the use of another interesting and novel technique which is type directed operational
semantics \baber{reference here}. Types are preserved by type annotations during substitution
in beta-reduction and \typeof cases.

\Rref{step-int} annotates integer expressions and makes them values. \Rref{step-appl} reduces left expression
of an application unless it becomes a value. \Rref{step-appr} works only if left expression of an
application is already reduced to value. \Rref{step-appr} then reduces right expression of
application to a value.
\Rref{step-beta} is the beta-reduction. It applies dually annotated lambda expression 
$[[\x.e : A1 -> B1 : A2 -> B2]]$ to input value $[[p:C]]$. Substitution replaces free occurances of variable
$[[x]]$ with value $[[p:A1]]$. Note that annotation of partial value $[[p]]$ has changed to $[[A1]]$ from 
$[[C]]$ during substitution.
This is to keep the most specific type in annotation. Substituted expression is also
annotated with both of the output types from annotated lambda expression. \baber{More details.}

\Rref{step-absbeta} deals with the simple lambda expressions of the form $[[\x.e]]$ as input to a function.
It follows same as \rref{step-beta} except that both of the input types are kept with $[[\x.e]]$ during
substitution i.e $[[\x.e:A2:A1]]$. \Rref{step-ann} reduces an annotated expression only if it is not a value
and $[[e]]$ reduces to some $[[e']]$. \Rref{step-rmann} drops inner annotation. \Rref{step-lamann} adds one
more type annotation with lambda expressions having single type annotation to make them values.
\Rref{step-typeof, step-typeofl, step-typeofr} deal with the reduction of \typeof expressions.

\Rref{step-typeof} reduces the case expression $[[e]]$ unless it becomes a value of the form $[[p:D]]$.
\Rref{step-typeofl} evaluates left branch of the \typeof expression if type of $[[e]]$ is subtype
of type of left branch.
\Rref{step-typeofr} evaluates right branch of the \typeof expression if type of $[[e]]$ is subtype
of type of right branch.
Subtyping condition in \rref{step-typeofl, step-typeofr} is interesting and important to consider.
It gives a freedom of various subtypes of $[[A]]$ and $[[B]]$ for corresponding branch instead of only
type $[[A]]$ and type $[[B]]$.
It is important to note that types of left and right branches of \typeof expression cannot overlap
because of disjoitness contraint in typing. Programs with overlapping types in branches of \typeof
expression will not type check in \cal.
A natural property of \cal is if type $[[A]]$ and type $[[B]]$ are two
disjoint types, then subtypes of $[[A]]$ are also disjoint to subtypes of $[[B]]$.

\paragraph{FindType Relation} FindType relation calculates the type of partial expressions
and is shown in lower part of \Cref{fig:union:os}. It takes $[[p]]$ as input and returns its type.
Partial expressions in \cal are $[[i]]$ and $[[\x.e: A -> B]]$.
FindType returns $[[Int]]$ when $[[p]]$ is an integer $[[i]]$ (\rref{findtype-int}). On the other hand,
FindType returns $[[A -> B]]$ against the partial expression $[[\x.e : A->B]]$ (\rref{findtype-arrow}).

\begin{figure}[t]
  \begin{small}
    \centering
    \drules[step]{$[[e --> e']]$}{Operational Semantics}{int, appl, appr, beta, absbeta, ann, rmann, lamann, typeof, typeofl, typeofr}
  \end{small}
  \begin{small}
    \centering
    \drules[findtype]{$[[findtype p A]]$}{FindType}{int, arrow}
  \end{small}
  \caption{Operational semantics for \cal and FindType.}
  \label{fig:union:os}
\end{figure}


%%%%%%%%%%%%%%%%%%%%%%%%%%%%%%
%% Type Safety and Determinism
%%%%%%%%%%%%%%%%%%%%%%%%%%%%%%
\subsection{Type Safety and Determinism}
\label{sec:union:safety}
\cal is type safe and deterministic. In this section we discuss the
proofs of type safety and determinism for \cal. Type safety is usually considered as composition
of type preservation and progress lemma. Type preservation (\Cref{lemma:union:preservation}) 
states that types are preserved during
reduction. Progress (\Cref{lemma:union:progress}) states well typed programs never get stuck. 
A well typed expression $[[e]]$ is 
either a value or it can take step to some $[[e']]$. Type of $[[e']]$ is preserved following the
type preservation property. Therefore, preservation and progress together ensure type safety.
We add one more choice in the conslusion of prorgress lemma to handle non-annotated lambda expressions 
($[[\x.e]]$). 
Determinism (\Cref{lemma:union:determinism}) is also a critical property for a calculus.
Non-deterministic calculi lead to incoherent programs.
Determinism ensures that a program will always produce the same unique result.

\begin{lemma}[Type Preservation]
\label{lemma:union:preservation}
  If \ $[[G |- e dirflag A]]$ and $[[e --> e']]$ then: $[[G |- e' dirflag A]]$.
\end{lemma}

\begin{comment}
\begin{proof}
  By induction on typing relation and subsequent inverting reduction relation.
  \begin{itemize}
    \item Cases \rref{typ-int, typ-var, typ-sub, typ-abs} are trivial to prove.
    \item Case \rref{typ-ann} requires helping \cref{lemma:union:check-pexpr-ann}.
    \item Case \rref{typ-app} requires helping \cref{lemma:union:pexpr-check-sub}
          and substitution \cref{lemma:union:substitution} for beta reduction.
    \item Case \rref{typ-typeof} requires substitution \cref{lemma:union:substitution}.
  \end{itemize}
\end{proof}

\baber{ToDo: change name of helping lemmas.}

\begin{lemma}[check-pexpr-ann]
\label{lemma:union:check-pexpr-ann}
  If \ $[[G |- p:C <= A]]$ \ then \ $[[G |- p <= A]]$.
\end{lemma}

\begin{lemma}[pexpr-check-sub]
\label{lemma:union:pexpr-check-sub}
  If \ $[[G |- p <= A]]$ \ and \ $[[A <: B]]$ \ then \ $[[G |- p <= B]]$.
\end{lemma}
\end{comment}

\begin{lemma}[Substitution]
\label{lemma:union:substitution}
  If \ $[[G, x:B , G1 |- e dirflag A]]$ \ and \ $[[G |- e' => B]]$
  then \ $[[G, G1 |- e [ x ~> e' ] dirflag A]]$
\end{lemma}

\begin{lemma}[Progress]
\label{lemma:union:progress}
If \ $[[ [] |- e dirflag A]]$ then:
 \begin{enumerate}
  \item either $[[e]]$ is a value.
  \item or $[[e]]$ can take a step to $[[e']]$.
  \item or $[[e]]$ = $[[\x.e]]$ and $[[dirflag]]$ = $[[<=]]$
  \end{enumerate}
\end{lemma}

\begin{comment}
\begin{proof}
By induction on typing relation.
  \begin{itemize}
    \item Cases \rref{typ-int, typ-var, typ-app, typ-sub, typ-abs} are trivial to prove.
    \item Case \rref{typ-anno} requires \cref{lemma:union:value-not-value}.
    \item Case \rref{typ-typeof} requires 
    \cref{lemma:union:check-pexpr-ann,lemma:union:check-or-typ,lemma:union:pexpr-inf-typ}.
  \end{itemize}
\end{proof}

\begin{lemma}[Value Decidability]
\label{lemma:union:value-not-value}
$\forall$ $[[e]]$, \ value \ $[[e]]$ \ $\vee$ \ $\neg$ value \ $[[e]]$.
\end{lemma}

\begin{lemma}[check-or-typ]
\label{lemma:union:check-or-typ}
If \ $[[A *s B]]$ \ and \ $[[G |- p <= A \/ B]]$ \ then:
  \begin{enumerate}
    \item either \ $[[G |- p <= A]]$
    \item or \ $[[G |- p <= B]]$
  \end{enumerate}
\end{lemma}

\begin{lemma}[pexpr-inf-typ]
\label{lemma:union:pexpr-inf-typ}
If \ $[[G |- p <= A]]$ \ then:
  \begin{enumerate}
  \item $\exists$ $[[B]]$, \ $[[B <: A]]$
  \item and \ $[[G |- p => B]]$
  \end{enumerate}
\end{lemma}
\end{comment}

\begin{lemma}[Determinism]
\label{lemma:union:determinism}
  If \ $[[G |- e dirflag A]]$
  \begin{enumerate}
  \item and \ $[[e --> e1]]$
  \item and \ $[[e --> e2]]$ \\
  then: $[[e1]]$ = $[[e2]]$.
  \end{enumerate}
\end{lemma}

\begin{comment}
\begin{proof}
  By induction on first reduction relation and inverting second reduction relation subsequently.
  All cases are trivial to solve by simple inversions except:
  \begin{itemize}
    \item Case \rref{typ-typeof} requires \cref{lemma:union:check-both-disj-false}.
  \end{itemize}
\end{proof}

\begin{lemma}[check-both-disj-false]
\label{lemma:union:check-both-disj-false}
If \ $[[A *s B]]$ \ and \ $[[G |- p <= A]]$ \ and \ $[[G |- p <= B]]$ \ then \ False.
\end{lemma}
\end{comment}

\baber{Again, should we show the property that a term cannot be checked against two disjoint types?}


%%%%%%%%%%%%%%%%%%%%%%%%%%%%%%
%% Discussion on Disjointness
%%%%%%%%%%%%%%%%%%%%%%%%%%%%%%
\subsection{Discussion on Disjointness}
\label{sec:union:discussion}
Interaction of intersection and union types is a well-known problem in programming languages.
In this section we discuss limited expressive power of our current disjoint specifications 
(\cref{def:union:disj}) and propose a robust and scalable definition for disjointness.
\Cref{def:union:disj} works fine in simple calculus as of now. But it adds complexities as soon as we add
intersection types to make the system more expressive.

\paragraph{Updated Disjointness}
An updated and robust declerative disjointness is shown in \Cref{fig:union:ord} as \Cref{def:union:disj1}.
It is important to establish the fact that \Cref{def:union:disj1} works as expected.
We prove equivalence of \Cref{def:union:disj} and \Cref{def:union:disj1} to establish this fact.
Examples illustrated in \Cref{sec:union:disj} also work fine with the updated disjointness definition.
\Cref{def:union:disj1} will be discussed in detail in \Cref{sec:inter} along with corresponding
algorithmic version.

\begin{lemma}[Disjointness Equivalence]
A $*_{s}$ B $\longleftrightarrow$ A $*_{s1}$ B.
\end{lemma}

\begin{figure}
    \begin{small}
    \centering
    \drules[ord]{$[[ordinary A]]$}{Ordinary Types}{top, int, arrow}
  \end{small}
  \medskip
  \begin{definition}
    \centering
    A $*_{s1}$ B $\Coloneqq$ $\forall$ C, $[[ordinary C]]$ $\rightarrow$ $\neg$ ($[[C <: A]]$ $\wedge$ $[[C <: B]]$)
    \label{def:union:disj1}
  \end{definition}
  \caption{Updated disjointness and ordinary types for \cal.}
  \label{fig:union:ord}
\end{figure}
