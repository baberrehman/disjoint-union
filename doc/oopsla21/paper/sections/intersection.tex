\section{\cal with Intersection Types: The Challenge of Disjointness}
\label{sec:inter}

In this section we add intersection types to \cal.
%The study of an
%extension of \cal with intersection types is motivated by the fact
%that most languages support both union and intersection types.
%Therefore it is important to understand whether intersection types
%can be easily added or whether there are some challenges.
The addition of intersection types poses a challenge,
since the notion of disjointness inspired from disjoint intersection types~\citep{oliveira2016disjoint}
no longer works. Fortunately, the alternative specification presented in
\Cref{sec:union:discussion} comes to the rescue. Furthermore, we show how to obtain
an algorithmic formulation of disjointness based on a novel notion
called \emph{lowest ordinary subtypes}. All the properties, including
type soundness and determinism are preserved in the extension of \cal
with intersection types.

%%%%%%%%%%%%%%%%%%%%%%%
%% Syntax and Semantics
%%%%%%%%%%%%%%%%%%%%%%%
\subsection{Syntax, Subtyping and Ordinary Types}
\label{sec:inter:system}
The syntax and subtyping for this section mostly follows
from \Cref{sec:union}.  The additional types and subtyping rules are shown in
\Cref{fig:inter:system}.
%This system can trivially be extended with
%more primitive types. We also have $Bool$ and $String$ types in our
%Coq formalization.
The most significant difference and novelty in this section
is the addition of intersection types $[[A/\ B]]$.
Expressions $[[e]]$, pre-values $[[p]]$, annotated values $[[w]]$,
values $[[v]]$, and context $[[G]]$ stay the same as in \Cref{sec:union}.

\subsubsection*{Subtyping}
The new rules for subtyping are shown in middle part of
\Cref{fig:inter:system}.  \Rref{s-anda,s-andb,s-andc} are for newly
added intersection types. \Rref{s-anda} states that a type $[[A]]$ is
a subtype of intersection of two types $[[B]]$ and $[[C]]$ only if
$[[A]]$ is a subtype of both $[[B]]$ and $[[C]]$. \Rref{s-andb,s-andc} 
state that an intersection type $[[A /\ B]]$ is a subtype
of some type $[[C]]$ if either of its component types ($[[A]]$ or
$[[B]]$) is a subtype of $[[C]]$. The subtyping relation is reflexive
and transitive.

\begin{figure}[t]
    \centering
    \begin{tabular}{lrcl} \toprule
      Types & $[[A]], [[B]]$ & $\Coloneqq$ & $ ... \mid [[A /\ B]] $ \\
      \bottomrule
    \end{tabular}
    \centering
    \drules[s]{$ [[A <: B ]] $}{Additional Subtyping Rules}{anda, andb, andc}
    \centering
    \drules[ord]{$[[ordinary A]]$}{Ordinary Types}{int, arrow}
%    \centering
%    {\renewcommand{\arraystretch}{1.2}
%    \begin{tabular}{|lcl|}
%      \multicolumn{3}{c}{Lowest Ordinary Subtypes (LOS) $[[findsubtypes A]]$} \\
%      \hline
%     $[[findsubtypes Top]]$ & = & \{$ [[Int]], [[Top -> Bot]]$\}  \\
%     $[[findsubtypes Bot]]$ & = & \{\}  \\
%     $[[findsubtypes Int]]$ & = & \{$ [[Int]] $\}  \\
%     $[[findsubtypes A -> B]]$ & = & \{$ [[Top -> Bot]] $\}  \\
%     $[[findsubtypes A \/ B]]$ & = & $ [[findsubtypes A]] \cup [[findsubtypes B]] $\\
%     $[[findsubtypes A /\ B]]$ & = & $ [[findsubtypes A]] \cap [[findsubtypes B]] $\\
%      \hline
%    \end{tabular} }
  \caption{Additional Types, Subtyping and Ordinary Types for \cal with intersection types.}
  \label{fig:inter:system}
\end{figure}

\subsubsection*{Ordinary Types}
%Declarative disjointness is now defined in terms of ordinary types
%instead of \emph{bottom-like} types.
Lower part of
\Cref{fig:inter:system} shows ordinary types~\cite{davies2000intersection}, 
which is the same as given in \Cref{sec:union:discussion}.
Ordinary types are essentially primitive types. In \cal they
are integers and functions. 
%$[[Top]]$, $[[Int]]$ and $[[A -> B]]$ types are considered as ordinary
%types in \cal with intersection types.

%%%%%%%%%%%%%%%%%%%%%%%
%% Disjointness
%%%%%%%%%%%%%%%%%%%%%%%
\subsection{Disjointness Specification}
\label{sec:inter:disj}
Disjointness is the most interesting aspect of the extension of \cal with
intersection types. Unfortunatelly, \Cref{def:union:disj} does not work with intersection
types. In what follows, we first explain why \Cref{def:union:disj} does not work, and then
we show how to introduce disjointness in the presence of intersection types. 

%We employ updated disjointness for \cal with intersection types as
%discussed in \Cref{sec:union:discussion} and discuss the updated
%disjointness in detail in this section.

\subsubsection*{Bottom-like Types, Intersection Types and Disjointness}
\noindent %Bottom-like types behave like $[[Bot]]$ type and are discussed in \Cref{sec:union:disj}.
Recall that disjointness in \Cref{sec:union} (\Cref{def:union:disj}) depends
upon bottom-like types, where two types are disjoint only if all their common
subtypes are bottom-like. However, this definition no longer works with the
addition of intersection types. Specifically, according to the subtyping rule
for intersection types, any two types have their intersection as one common subtype.
For non-bottom-like types, their intersection is also not bottom-like. For
example, type $[[Int]]$ and type $[[Bool]]$ now has a subtype $[[Int /\ Bool]]$,
which is not bottom-like. Therefore, the disjointness definition fails, since we
can always find a common non-bottom-like subtype for any two (non-bottom-like)
types.

\begin{comment}
Reader may think at this point to add intersection of non-overlapping types such as $[[Int /\ Bool]]$
in bottom-like types to solve the problem. A trivial and intuitive approach to think of is:

\begin{center}
\drule[]{bl-andsub}
\end{center}

\noindent \Rref{bl-andsub} states that if two types $[[A]]$ and $[[B]]$
are not subtypes of each other (i.e. non-overlapping) then intersection of
such types $[[A /\ B]]$ is bottom-like.
\Rref{bl-andsub} works for simple cases such as $[[Int]]$ and $[[Bool]]$.
But it fails if  $[[A]]$ = $[[Int /\ Bool]]$ and 
$[[B]]$ = $[[Int /\ Bool]]$.
Because $[[A]]$ ($[[Int /\ Bool]]$) and $[[B]]$ ($[[Int /\ Bool]]$) are subtypes
of each other and are not bottom-like as per \rref{bl-andsub}.
So, naive addition of \rref{bl-andsub} skips potential bottem-like types.
Another alternative may be:

\begin{center}
\drule[]{bl-anddisj}
\end{center}

\noindent \Rref{bl-anddisj} states that if two types $[[A]]$ and $[[B]]$ are disjoint,
then intersection of such types $[[A /\ B]]$ is bottom-like.
But \rref{bl-anddisj} imposes additional complexities of mutually
dependent definitions among disjointness and bottom-like.
This makes completeness challenging or even impossible to prove.
\end{comment}
\subsubsection*{A possible solution: the Ceylon approach}
A possible solution for this issue is to add a subtyping rule which makes intersection of
disjoint types to be subtype of $[[Bot]]$.

\begin{center}
\drule[]{s-disj}
\end{center}
%\bruno{Please use disjointness spec instead of disjoitness algorithm in the rule.}

\noindent This rule is adopted by the Ceylon language~\cite{muehlboeck2018empowering}:
\Rref{s-disj} states that the insersection type $[[A /\ B]]$
of two disjoint types $[[A]]$ and $[[B]]$ is a subtype of the $[[Bot]]$ type.
In other words, now the type $[[Int /\ Bool]]$ would be a bottom-like type, and the
definition of disjointness used in Section~\ref{sec:union} could still work.
The logic behind this rule is that if we interpret types as sets of values,
and intersection as set intersection, then intersecting disjoint sets
is the empty set. In other words, we would get a type that has no inhabitants (i.e.
a bottom-like type).
For instance the set of all integers is disjoint to the set of all booleans.
However we do not adopt the Ceylon solution here two reasons:

\begin{enumerate}

\item \Rref{s-disj} complicates the system because
  it adds a mutual dependency between subtyping and disjointness:
  disjointness is defined in terms of subtyping, and subtyping now
  uses disjointness as well in \rref{s-disj}. This creates important
  challenges for the metatheory. In particular, the completeness proof
  for disjointness becomes quite challenging.

\item Additionaly, the assumption that intersections of disjoint types
  are equivalent to $[[Bot]]$ is too strong for some calculi with intersection
  types. If a merge operator~\cite{reynolds1988preliminary} is allowed in the calculus, 
  intersection types
  with disjoint types can be inhabited with values (for example, in 
  \cite{oliveira2016disjoint},
  the type $[[Int /\ Bool]]$ is inhabited by $[[1]] ,, \mathsf{True}$). Considering those
  types bottom-like would lead to a problematic definition of
  subtyping, since some bottom-like types (those based on disjoint types) would
  be inhabited.

\end{enumerate}

For those reasons we adopt a different approach in \name.
Nevertheless, in Section~\ref{sec:inter:refactoring} we will see that it is possible to
create an extension of \name that includes (and in fact generalizes)
the Ceylon-style \rref{s-disj}.

\subsubsection*{Disjointness based on Ordinary Types to the rescue}
To solve the problem with the disjointness specification, we resort to
the alternative definition of disjointness presented in \Cref{sec:union:discussion}.

\begin{definition}
\label{def:inter:disj}
  A $*_s$ B $\Coloneqq$ $\forall$ C, \ $[[ordinary C]]$ \ $\rightarrow$ \ $\neg$ \ ($[[C <: A]]$ and $[[C <: B]]$).
\end{definition}

Interestingly, while in Section~\ref{sec:union} such definition was
equivalent to the definition using bottom-like types, this is no
longer the case for \name with intersection types. To see why,
consider again the types $[[Int]]$ and $[[Bool]]$.  $[[Int]]$ and
$[[Bool]]$ do not share any common ordinary subtype. Therefore,
$[[Int]]$ and $[[Bool]]$ are disjoint types according to
\Cref{def:inter:disj}.
\begin{comment}
We extend our previous example of type $[[Int]]$ and type $[[Bool]]$ and show how
disjointness based upon ordinary types categorize them as disjoint types.
An important observation at this point is common subtypes of type $[[Int]]$ and type $[[Bool]]$
cannot include either $[[Int]]$ or $[[Bool]]$. Problematic types are the intersection types
such as $[[Int /\ Bool]]$. We empahsize the point that ordinary types in \cal does not contain
intersection types. Further, all ordinary types are non-overlapping in \cal.
Therefore, now we say that
two types are disjoint if they do not have any common ordinary subtype. $[[Int]]$ and $[[Bool]]$
do not share any common ordinary subtype. Therefore, $[[Int]]$ and $[[Bool]]$ are disjoint types.
\Cref{def:inter:disj} shows the declarative disjointness for \cal with intersection types:


\noindent Two types $[[A]]$ and $[[B]]$ are
disjoint if the two types $[[A /\ B]]$ do
not have any common ordinary subtype. For example, $[[Int]]$ and $[[A -> B]]$
are disjoint types because there is no ordinary type that is a subtype
of both types ($[[Int]]$ and $[[A -> B]]$).
\Cref{def:inter:disj} is the same as 
\Cref{def:union:disj1}. However, while the \Cref{def:union:disj1} in \Cref{sec:union:discussion}
is equivalent the definition of disjointness using bottom-like types (\Cref{def:union:disj}),
in the calculus with intersection types that is no longer the case.
\end{comment}
We further illustrate  
\Cref{def:inter:disj} with a few concrete examples:

\begin{enumerate}
  \item $\boldsymbol{A = [[Int]], \ B = \ [[Int -> Bool]]:}$ \\
        Since there is no ordinary type that is subtype of both $[[Int]]$ and $[[Int -> Bool]]$,
        the two types are disjoint.
  \item $\boldsymbol{A = [[Int \/ Bool]], \ B = \ [[Bot]]:}$ \\
    Since there is no ordinary type that is subtype of both $[[Int \/ Bool]]$ and $[[Bot]]$,
    $[[Int \/ Bool]]$ and $[[Bot]]$ are disjoint types.
    In general, $[[Bot]]$ type is disjoint to all types because $[[Bot]]$ type does not
    have any ordinary subtype.
  \item $\boldsymbol{A = [[Int /\ (Int -> Bool)]], \ B = \ [[Int]]:}$ \\
        There is no ordinary type that is subtype of both $[[Int /\ (Int -> Bool)]]$ and $[[Int]]$.
        Therefore, $[[Int /\ A -> B]]$ and $[[Int]]$ are disjoint types.
        In general, an intersection of two disjoint types ($[[Int /\ A -> B]]$ in this case)
        is always disjoint to all types.
  \item $\boldsymbol{A = [[Int /\ Bool]], \ B = \ [[Int \/ Bool]]:}$ \\
        There is no ordinary type that is subtype of both $[[Int /\ Bool]]$ and $[[Int \/ Bool]]$.
        Therefore, $[[Int /\ Bool]]$ and $[[Int \/ Bool]]$ are disjoint types.
        This follows from the description in example 3 that type $[[Int /\ Bool]]$ will always be
        disjoint to all types.
  \item $\boldsymbol{A = [[Int]], \ B = \ [[Top]]:}$ \\
        In this case, $[[Int]]$ and $[[Top]]$ share a common ordinary subtype which is $[[Int]]$.
        Therefore, $[[Int]]$ and $[[Top]]$ are not disjoint types.
\end{enumerate}

%The
%reason we did not write intersection in \Cref{def:union:disj1} is
%because we do not have intersection types in \cal discussed in
%\Cref{sec:union}.
%Updated disjointness specifications are now
%represented as $[[A *s B]]$ and not $[[A]]$ $*_{s1}$ $[[B]]$.

\subsection{Algorithmic Disjointness}

The change in the disjointness specification has a significant impact on an
algorithmic formulation. In particular, it is not obvious at all how to adapt
the algorithmic formulation in \Cref{fig:union:disj-typ}. To obtain a
sound, complete and decidable formulation of disjointness, we employ the novel
notion of \emph{lowest ordinary subtypes}.

\begin{figure}[t]
    \centering
    {\renewcommand{\arraystretch}{1.2}
    \begin{tabular}{|lcl|}
      \multicolumn{3}{c}{Lowest Ordinary Subtypes (LOS) $[[findsubtypes A]]$} \\
      \hline
     $[[findsubtypes Top]]$ & = & $\{ [[Int]], [[Top -> Bot]]\}$  \\
     $[[findsubtypes Bot]]$ & = & $\{\}$  \\
     $[[findsubtypes Int]]$ & = & $\{ [[Int]] \}$  \\
     $[[findsubtypes A -> B]]$ & = & $\{ [[Top -> Bot]] \}$  \\
     $[[findsubtypes A \/ B]]$ & = & $ [[findsubtypes A]] \cup [[findsubtypes B]] $\\
     $[[findsubtypes A /\ B]]$ & = & $ [[findsubtypes A]] \cap [[findsubtypes B]] $\\
      \hline
    \end{tabular} }
  \caption{Lowest Ordinary Subtypes function for \cal with intersection types.}
  \label{fig:inter:los}
\end{figure}


\subsubsection*{Lowest Ordinary Subtypes ($[[findsubtypes A]]$)}
\Cref{fig:inter:los} shows the definition of
\emph{lowest ordinary subtypes} (LOS) ($[[findsubtypes A]]$).
LOS is defined as a function which
returns a set of ordinary subtypes of the given input type. 
No ordinary type is a subtype of $[[Bot]]$. The only ordinary
subtype of $[[Int]]$ is $[[Int]]$ itself. The function case is
interesting. Since no two functions are disjoint in the calculus
proposed in this paper, the case for function types $[[A -> B]]$ returns $[[Top
    -> Bot]]$. This type is the least ordinary function type, which is a subtype
of all function types.
Ordinary
subtypes of $[[Top]]$ are $[[Int]]$ and $[[Top -> Bot]]$.
In the case of union types $[[A \/ B]]$, the
algorithm is collects the LOS of $[[A]]$ and $[[B]]$ and returns the union of the
two sets. For intersection types $[[A
    /\ B]]$ the algorithm collects the LOS of $[[A]]$ and $[[B]]$
and returns the intersection of the two sets.
Note that LOS is defined as a structurally recursive function and therefore
its decidability is immediate.

\subsubsection*{Algorithmic Disjointness}
With LOS, it is straightforward to give an algorithmic formulation of
disjointness:
%Since we gave a new definition for declarative disjointness.
%Therefore, we define a new algorithm for disjointness as shown in \Cref{def:inter:ad}.

\begin{definition}
\label{def:inter:ad}
%\texttt{`abc \textasciigrave abc}
  A $*_a$ B $\Coloneqq$  $ [[findsubtypes A]] \cap [[findsubtypes B]] $ = $\{\}$.
\end{definition}

\noindent The algorithmic formulation of disjointness in
\Cref{def:inter:ad} states that two
types $[[A]]$ and $[[B]]$ are disjoint
if they do not have any common lowest ordinary subtypes. That is the
set intersection of $[[findsubtypes A]]$ and $[[findsubtypes B]]$ is the empty set.
%In simple words,
%two types $[[A]]$ and $[[B]]$ are disjoint if they do not share any common ordinary subtype because
%$FindSubTypes$ ($[[findsubtypes A]]$) returns a set of ordinary subtypes.
Note that this algorithm is naturally very close to \Cref{def:inter:disj}.
\begin{comment}
We illustrate \Cref{def:inter:ad} with a few examples:

\begin{enumerate}
  \item $\boldsymbol{A = [[Int]], \ B = \ [[A -> B]]:}$ \\
        $[[findsubtypes Int]]$ returns \{$[[Int]]$\} and $[[findsubtypes A -> B]]$ returns
        \{$[[Top -> Bot]]$\}. Set intersection of \{$[[Int]]$\} and \{$[[Top -> Bot]]$\} is
        empty set \{\}. Therefore, $[[Int]]$ and $[[A -> B]]$ are disjoint types.
  \item $\boldsymbol{A = [[Int]], \ B = \ [[Bot]]:}$ \\
        $[[findsubtypes Int]]$ returns \{$[[Int]]$\} and $[[findsubtypes Bot]]$ returns
        \{\}. Set intersection of \{$[[Int]]$\} and \{\} is
        empty set \{\}. Therefore, $[[Int]]$ and $[[Bot]]$ are disjoint types.
        In general, type $[[Bot]]$ is disjoint to all types because $[[findsubtypes Bot]]$
        will always return \{\} and intersection of \{\} with all other sets is \{\}.
  \item $\boldsymbol{A = [[Int /\ A -> B]], \ B = \ [[Int]]:}$ \\
        $[[findsubtypes Int /\ A -> B]]$ returns \{\} and $[[findsubtypes Int]]$ returns
        \{$[[Int]]$\}. Set intersection of \{\} and \{$[[Int]]$\} is
        empty set \{\}. Therefore, $[[Int /\ A -> B]]$ and $[[Int]]$ are disjoint types.
        In general, intersection type of two disjoint types which is $[[Int /\ A -> B]]$ in this case,
        is always disjoint to all types.
  \item $\boldsymbol{A = [[Int]], \ B = \ [[Top]]:}$ \\
        $[[findsubtypes Int]]$ returns \{$[[Int]]$\} and $[[findsubtypes Top]]$ returns
        \{$[[Int]]$, $[[Top -> Bot]]$\}.
        Set intersection of \{$[[Int]]$\} and \{$[[Int]]$, $[[Top -> Bot]]$\} is
        set \{$[[Int]]$\}. Therefore, $[[Int]]$ and $[[Top]]$ are not disjoint types.
\end{enumerate}
\end{comment}

\subsubsection*{Soundness and Completeness}

We prove that the algorithmic disjointness is sound and complete with respect to
the declarative specification.

\begin{lemma}[Disjointness Soundness]
  If \ $[[A * B]]$ \ then \ $[[A *s B]]$.
\label{lemma:inter:disj-sound}
\end{lemma}

\begin{lemma}[Disjointness Completeness]
  If \ $[[A *s B]]$ \ then \ $[[A * B]]$.
\label{lemma:inter:disj-complete}
\end{lemma}

\subsection{Typing, Semantics and Metatheory}

The typing and operational semantics stay the same as in \Cref{sec:union}. Note
that while those rules stay the same, as types and subtyping rules are extended
with intersection type.
We prove that \cal with intersection types preserves type soundness and determinism.
%\bruno{add determinism}

\begin{lemma}[Type Preservation]
\label{lemma:inter:preservation}
  If \ $[[G |- e dirflag A]]$ and $[[e --> e']]$ then $[[G |- e' dirflag A]]$.
\end{lemma}

\begin{lemma}[Progress]
\label{lemma:inter:progress}
If \ $[[ [] |- e dirflag A]]$ then
 \begin{enumerate}
  \item either $[[e]]$ is a value.
  \item or $[[e]]$ can take a step to $[[e']]$.
  \end{enumerate}
\end{lemma}

\begin{theorem}[Determinism]
\label{lemma:inter:determinism}
  If \ $[[G |- e dirflag A]]$ and \ $[[e --> e1]]$ and \ $[[e --> e2]]$ then $[[e1]]$ = $[[e2]]$.
\end{theorem}

\begin{comment}
\begin{lemma}[Substitution]
\label{lemma:inter:substitution}
  If \ $[[G, x:B , G1 |- e dirflag A]]$ \ and \ $[[G |- e' => B]]$
  then \ $[[G, G1 |- e [ x ~> e' ] dirflag A]]$
\end{lemma}
\end{comment}

%%% Local Variables:
%%% mode: latex
%%% TeX-master: "../paper"
%%% org-ref-default-bibliography: "../paper.bib"
%%% End:
