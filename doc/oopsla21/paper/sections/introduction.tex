\section{Introduction}
\label{sec:intro}

Most programming languages support some mechanism to express terms
with alternative types. Algol 68~\cite{} included a form of
\emph{tagged} unions for this purpose. With tagged unions
an explicit tag distinguishes between different cases in the
union type.
Such an approach has been adopted by functional languages, like Haskell, ML, or
OCaml, which allow tagged unions (or sum types~\cite{}), typically via
either \emph{algebraic datatypes}~\cite{} or \emph{variant types}~\cite{}.
Languages like C or C++ support \emph{untagged} union types where
values of the alternative types are simply stored at the same memory
location. However, there is no checking of types when accessing values of
such untagged types. Thus, it is up to the programmer to ensure that the proper
values are accessed correctly in different contexts. Otherwise the
program may produce errors by accessing the value at the incorrect type.

Modern OOP languages, such as Scala~\cite{}, Flow~\cite{},
TypeScript~\cite{} or Ceylon~\cite{} support a form
of untagged union types.
In such languages a union type $[[A \/ B]]$ denotes expressions which can have type
$[[A]]$ or type $[[B]]$. Union types can be useful in many situations.
For instance, union types provide an alternative to some forms
of overloading. The idea is that a function that takes an argument
with a union type acts similarly to an overloaded function.
Furthermore, union types have other uses, such as modelling error handling.
If a function returning a number may fail due to some
error, the union type $[[String \/ Int]]$ can be returned: an integer
is returned upon successful computation of the result; and
a string (an error message) is returned if an error happens.

To safely access values with union types, some form of
\emph{elimination construct} is needed. In many programming languages,
this is typically done by employing a language construct that checks
the types of the values at runtime, and acts as the elimination
construct. Several elimination constructs for (untagged) union types
have also been studied in the research literature~\cite{}. Typically,
such constructs take the form of a type-based case analysis
expression.

A complication is that the presence of subtyping introduces the
possibility of \emph{overlapping types}. For instance we may have a
\lstinline{Student} and a \lstinline{Person}, where every student is a person (but not
vice-versa). If we try to eliminate a union using such types we can
run into situations where the type in one branch can cover a type in a
different branch (for instance \lstinline{Person} can also cover
\lstinline{Student}). More generally, types can \emph{partially overlap}
and for some values two branches with such types can apply, whereas
for some other values only one branch applies.
Therefore the design of such elimination constructs has to
consider what to do in situations where overlapping types arise.  A
first possibility is to have a \emph{non-deterministic semantics},
where any of the branches that matches can be taken. However, in
practice determinism is a desirable property,
so this option is not practical. A second possibility, which is
commonly used for overloading, is to employ a \emph{best-match
  semantics}, where we attempt to find the case with the type that
best matches the value. Yet another option is to employ a
\emph{first-match semantics}, which employs the order of the branches
in the case. Various existing elimination constructs for unions~\cite{}
employ a first-match approach. All of these three options have been explored
and studied in the literature. 

The Ceylon language is a JVM-based language that aims to provide an
alternative to Java. The type system is interesting
in that it departs from existing language designs, in particular
with respect to union types and method overloading.
The Ceylon designers had a few different
reasons for this. They wanted to have a fairly rich type system
supporting, among others: \emph{subtyping}; \emph{generics with bounded
quantification}; \emph{union and intersection types}; and \emph{type-inference}.
The aim was to support most features that are also already available
in Java, as well as a few new ones. However they wanted to do this in
a principled way, where all the features interacted nicely.  A
stumbling block towards this goal was Java-style method
overloading\footnote{\url{https://github.com/ceylon/ceylon-spec/issues/73}}.
The interaction of overloading with other
features was found to be quite challenging. Additionally, the presence of
overloaded methods with overlapping types
makes reasoning about the code hard 
for both tools and humans. Algorithms for finding the best match for an
overloaded method in the presence of rich type system features (such as
those in Ceylon) are challenging, and not necessarally well-studied in the
existing literature. Moreover allowing overlapping methods can make
the code harder to reason for humans: without a clear knowledge of how
overloading resolution works, programmers may incorrectly assume that
a different overloaded method is invoked. Or worse, overloading can
happen silently, by simply reusing the same name for a new
method. These problems can lead to subtle bugs.
For these reasons, the Ceylon designers decided not to support
Java-style method overloading.

To counter the absence of overloading, the Ceylon designers turned to
union types instead. However, they did so in a way that differs from
existing approaches. Ceylon includes a type-based 
\emph{switch construct} where all the cases must be \emph{disjoint}.  If
two types are found to be overlapping, then the program is statically
rejected. Many common cases of method overloading, which are clearly
not ambiguous, can be modelled using union types and disjoint switches.
By using an approach based on disjointness, some use cases for
overloading that involve Java-style overloading with
overlapping types are forbidden. However,
programmers can still resort to creating non-overloaded methods in
such a case, which arguably results in code easier to reason about.
Disjointness ensures that it is always
clear which implementation is selected for an ``overloaded'' method,
and only in such cases overloading is allowed\footnote{Ceylon does
  allow dynamic type tests, similar to \lstinline{instanceof} in Java, which can be used
  in combination with the switch construct to simulate some overlapping.}.
In the switch construct,
the order of the cases does not matter and reordering the cases has no
impact on the semantics, which can also aid program undertanding and
refactoring.
Additionally, union types have other applications besides overloading,
so we can get other interesting functionality as well. Finally, from
the language design point of view, it would be strange to support two
mechanisms (method overloading and union types), which greatly overlap
in terms of functionality.

While implemented in the Ceylon language, such construct
with disjoint switches has not been well studied formally.
The work by \citet{muehlboeck2018empowering} is the only work that we are aware of,
where the Ceylon's switch construct
and disjointness are briefly mentioned. However, the focus 
of that work is on algorithmic formulations of subtyping
with unions types, intersection types and various distributivity
rules for subtyping. No semantics of the switch construct is given.
Disjointness is informally defined in various sentences in the
Ceylon documentation. It involves a set of 14 rules described in English\footnote{\url{http://web.mit.edu/ceylon_v1.3.3/ceylon-1.3.3/doc/en/spec/html_single}}. Some of the rules are relatively generic, while
others are quite language specific. 
Interestingly, a notion
of disjointness has been already studied in the literature
for intersection types~\cite{oliveira2016disjoint}. That line of work studies calculi
with intersection types and a \textit{merge operator}~\cite{reynolds1988preliminary}. Disjointness
is used to prevent ambiguity in merges, which can create
values with types such as $[[Int /\ Bool]]$. Only values
with disjoint types can be used in a merge.

In this paper, we study union types with disjoint switches formally
and in a language independent way. We present the \emph{union
  calculus} (\cal), which includes disjoint switches and union types.
The notion of disjointness in \cal is interesting in that it is
the dual notion of disjointness for intersection types.
We prove several results, including \emph{type soundness}, \emph{determinism}
and the \emph{soundness} and \emph{completeness} of algorithmic formulations
of disjointness.
We also study several extensions of \cal. In particular,
the first extension (discussed in Section~\ref{sec:inter}) adds intersection
types to \cal. It turns out such extension is non-trivial, as it reveals
a challenge that arises for disjointness when combining
union and intersection types:
the notion of dual notion disjointness borrowed from
disjoint intersection types no longer works, and we must employ
a novel, more general, notion instead. Such change also has an impact
on the algorithmic formulation of disjointness, which must change as
well. We also study several other extensions, including intersection
types and distributive subtyping. We prove that all the extensions retain
the original properties of \cal. Moreover, all the results about \cal and its
extensions have been formalized in the Coq theorem prover.

In summary, the contributions of this paper are:

\begin{itemize}
\item {\bf The \name calculus:} We present a simple calculus with union
  types, and a disjoint switch construct. The calculus is type sound and
  deterministic. 
\item {\bf Specifications and algorithmic formulations of disjointness:}
  We present two formulations of disjointness, which are general and
  language independent. The first formulation is directly derived from
  the the existing notion of disjointness for disjoint intersection types,
  but it only works for a calculus with union types. The second formulation
  is novel and more general, and can be used in a calculus that includes
  intersection types as well.
\item {\bf Extensions:} We study several extensions of \name, including the
  addition of intersection types, distributive subtyping and
  subtyping rule for a class of types that represent empty types.
\item {\bf Mechanical formalization:}
  All the results about \cal and its
  extensions have been formalized in the Coq theorem prover.
  \footnote{{\bf Note to reviewers:} The Coq formalizations can be found in the
  supplementary material for this paper.}
\end{itemize}



%\begin{align*}
%&Isomorphic & A \sim B & ::= [[A <: B]] \wedge [[B <: A]]
%\end{align*}


%\begin{align*}
%&BottomLikeSpec & C & ::= (\forall A ~ B, ~ [[A /\ B]] \sim C \rightarrow \neg ( [[ A <: B ]] ) \wedge \neg ( [[ B <: A ]] )) \vee ([[C <: Bot]])
%\end{align*}

%\begin{align*}
%&DisjSpec & A * B & ::= \forall C, [[C <: A]] \wedge [[C <: B]] \rightarrow  \rfloor [[C]] \lfloor
%\end{align*}

%%% Local Variables:
%%% mode: latex
%%% TeX-master: "../paper"
%%% org-ref-default-bibliography: "../paper.bib"
%%% End:
