\section{Overview}
\label{sec:overview}

\bruno{I'm sketching a possible structure, but it can be changed ofcourse.}

\subsection{Union Types}

Brief intro to union types and one or 2 simple examples that show that we
can automatically lift values into union type without the need of some
tag/constructor. Something like:

\begin{verbatim}
... x : String | Int = 1 ...
\end{verbatim}

\subsection{Eliminating Union Types}

Motivate the need for having a construct that can eliminate union types,
perhaps trying to use an example where the String above would be a kind
of exception, and the Int would be regular computation. Alternatively
find an existing example from the literature.

Discussing existing approaches for eliminating union types;
point out that elimination constructs based on types (our focus) vs
elimination constructs based on tags (which are used in algebraic
datatypes or polymorphic variants (like in OCaml)).
The focus should be on type-based approaches.

Try to identify some limitations/problems. For instance how to deal
with ambiguity (just use order? restrict the construct somehow ...).

\subsection{Eliminating Union Types in Ceylon}

Introduce the Ceylon approach with examples; introduce the idea of
disjointness informally; contrast with existing constructs.
Try to motivate the adapotion of disjointness (this should be discussed
in the Ceylon documentation to some extent).

Here it is important to give representative examples (look at documentation
and/or other online resources), including examples
that demonstrate how Ceylon uses union types to model overloaded functions.

\snow{
  Known problem in Ceylon:
  (Int|String) & (Int|Char) & (Char|String) is not treated as a bottom-like type

  Potential reason:
  It defines bottom-like as:
  Bottom-like A&B ::= A * B

  Possible consequences:
  - [harmless] we can write a case branch for such type.
  - The definition of disjointness might be not so presice
}


\subsection{Our work}

\snow{Some concerns: does the addition of intersection types and distributivity
  really increase the expressiveness of the system?
  Currently the only interesting intersection type has the form of (A->B) \& (C->D),
  like (int->bool)\&(bool->int).
  But there is no way to construct a term of such type.
  The closest thing is $\lambda$ x.x, which has (int->int)\&(bool->bool).}

Introduce our work, setting the goal to study the construct formally.
Connect with the work on disjoint intersection types, which also
employs a notion of disjointness, but for intersection. Explain that
what is needed is a dual notion of disjointness.

Introduce the first calculus, and explain that it is directly inspired
by a dual notion of discjointness.

Introduce the second calculus and identify a technical challenge with
disjointness: the addition of intersection types breaks the previous
notion of disjointness. Introduce the novel way to find disjoint types.

Summarize/mention key results: type-safety; soundness/completeness of
disjointness; determinism.

Perhaps here it is also useful to identity, together with Baber, what
were the most challenging aspects in the formalization, and maybe
highlight these.

\baber{This section will give an overview.}
