\section{Overview}
\label{sec:overview}

This section provides some background on union types, and some common approaches
to eliminate union types. Then it describes the Ceylon approach to Union types.
Finally, it presents the key ideas and challenges in our work and \name.

\ningning{We may want to be consistent with the syntax used in examples. Now
  there are three things going on: examples before introducing Ceylon; examples
  in Ceylon; examples in our calculi. Will it look confusing as we use three
  different syntax rules?}
\bruno{Three different syntaxes is definitly too much. I think we may be able
  to get by with two syntaxes: our calculus and Ceylon. Alternatevely, we can
  use just the syntax in our calculus here, and later, perhaps in the discussion
  have examples in Ceylon, when comparing to Ceylon. }

\subsection{Tagged Union Types}

% Brief intro to union types and one or 2 simple examples that show that we
% can automatically lift values into union type without the need of some
% tag/constructor. Something like:

We start with a brief introduction to union types. An expression has a union
type $[[A\/B]]$, if it can be considered to have either type $[[A]]$ \textit{or}
type $[[B]]$. Many systems model \textit{tagged union types} (also called
\textit{sum types} or \textit{variants types}), where explicit \textit{tags}
are used to construct terms with union types. Languages with algebraic datatypes~\cite{hope}
or (polymorphic) variants~\cite{garrigue98} support tagged union types. 
In their basic form, there are two introduction forms:
$\mathsf{inj_1} :: [[A -> A \/ B]]$ turn the type of an expression of type
$[[A]]$ into type $[[A \/ B]]$; and $\mathsf{inj_2} :: [[B -> A \/ B]]$
turns the type of an expressions of type $[[B]]$ into type $[[A \/ B]]$.
For example, we can have:

\begin{lstlisting}
inj1 "foo": String | Int
inj2 1 : String | Int
\end{lstlisting}

\noindent Using tagged union types, we can implement a safe integer
division function, as:

\begin{lstlisting}
function safediv (x : Int) (y : Int) : String | Int =
  if (y == 0) then inj1 "Divided by zero"
  else inj2 (x / y)
\end{lstlisting}

\noindent Here the intention is to have a safe (integer) division operation that detects
division by zero errors, and requires clients of this function to handle
such errors. The return type \lstinline{String | Int} denotes that the function
can either return an error message (a string), or an integer, when division
is performed without errors.

\paragraph{Eliminating tagged union types.}
Tagged union types are eliminated by some form of case analysis.
For consistency with the rest of the paper, we use a syntactic form with
\lstinline{switch} expressions for such case analysis. For example,
the following program \lstinline{tostring} has different behaviors depending on the
tag of \lstinline{x}, where \lstinline{show} takes an \lstinline{Int} and
returns back its string representation.

\begin{lstlisting}
function tostring (x: String | Int) : String =
  switch (x)
    inj1 str -> str
    inj2 num -> show num
\end{lstlisting}

Combining union type construction in \lstinline{safediv} and its elimination in
\lstinline{tostring}, we can easily implement an interface which returns the
result of safe division as one \lstinline{String}.

\begin{lstlisting}
> tostring (safediv 42 2)
"21"
> tostring (safe 42 0)
"Divided by zero"
\end{lstlisting}


\subsection{Type-directed Elimination of Union Types}\label{subsec:elimination}

% Motivate the need for having a construct that can eliminate union types,
% perhaps trying to use an example where the String above would be a kind
% of exception, and the Int would be regular computation. Alternatively
% find an existing example from the literature.

% Discussing existing approaches for eliminating union types;
% point out that elimination constructs based on types (our focus) vs
% elimination constructs based on tags (which are used in algebraic
% datatypes or polymorphic variants (like in OCaml)).
% The focus should be on type-based approaches.

% Try to identify some limitations/problems. For instance how to deal
% with ambiguity (just use order? restrict the construct somehow ...).


While tags are useful to make it explicit which type a value belongs to, they
also add clutter in the programs, and usually require extra space to store the
tags. On the other hand, in systems with subtyping for union types
\cite{dunfield2014elaborating,pierce1991programming,muehlboeck2018empowering},
explicit tags are replaced by implicit coercions represented by the two
subtyping rules $[[A <: A \/ B]]$ and $[[B <: A \/ B]]$. We call such types
\textit{untagged union types}, or simply \textit{union types}. In those systems,
a term of type $[[A]]$ or $[[B]]$ can be directly used as if it had type $[[A \/
B]]$, and thus we can model safe division as

\begin{lstlisting}
function safediv2 (x : Int) (y : Int) : String | Int =
  if (y == 0) then "Divided by zero"
  else (x / y)
\end{lstlisting}

\noindent However, now elimination of union types cannot rely on tags anymore, and
different systems implement elimination differently. We review some of the
existing approaches next.

\paragraph{Single branch elimination.}

A possible approach is to use an elimination of an expression with a union type
$[[A \/ B]]$ supports only one branch, and the branch needs to have the same
type when the expression has type $[[A]]$, or type $[[B]]$. This approach is
adopted, for example by \citet{pierce1991programming},
\citet{barbanera1995intersection}, and more recently,
\citet{dunfield2014elaborating}. For better illustration, we adapt their syntax
using our switch notation in the examples, while preserving their semantics. For
example, in \lstinline{tostring2}, the expression \lstinline{show x} must return
\lstinline{String} when \lstinline{y : String}, and when \lstinline{y : Int},
which means that \lstinline{show} must be overloaded (in
\citet{pierce1991programming,dunfield2014elaborating,barbanera1995intersection},
this can be implemented by requiring \lstinline{show} to have an
\textit{intersection type}).

\begin{lstlisting}
function tostring2 (x: String | Int) : String =
  switch (x)
    y -> show y
\end{lstlisting}

This implementation is concise, but it is also restrictive as it can no longer
support multiple branches according to the different representations of
\lstinline{x}. Furthermore it relies on the language also supporting overloaded
functions. Without overloaded functions the construct would not be very useful.
%\bruno{Perhaps, somewhere in this section we need to comment on the non-deterministic semantics.
%  Maybe we can do that here and point out, for instance, that Dunfield's approach has a
%  non-deterministic semantics (since it actually allows, for instance two overloaded
%  implementations of show with integer arguments)}

\paragraph{Type-directed elimination.}

On the other hand, some
systems~\cite{castagna:settheoretic,muehlboeck2018empowering} support
\textit{type-directed} elimination of union types. For instance,
\lstinline{tostring3} has different behaviors depending on the \textit{type} of
\lstinline{x}.

\begin{lstlisting}
function tostring3 (x: String | Int) : String =
  switch (x)
    String -> x
    Int -> show x
\end{lstlisting}

\snow{To be consistent with our syntax, each branch should name a variable, like
\lstinline{tostring2}. Same applies to \lstinline{isstudent2}.}

\noindent Note here \lstinline{show} is directly applied to \lstinline{x}, and
it does not need to be overloaded, as the type-directed elimination
\textit{refines} the type of \lstinline{x} from \lstinline{String | Int} to
\lstinline{Int} in the second branch (such type refinements are basically a form of
\textit{occurrence typing} \cite{Tobin:typedscheme}).

However, compared to tag-directed elimination, extra care must to taken with
type-directed elimination. In particular, while we can easily distinguish tags,
in type-directed elimination, ambiguity may arise when types in a union type
overlap. For example, consider the type \lstinline{Person | Student}, where we
assume \lstinline{Student} is a subtype of \lstinline{Person}. With tag-directed
elimination, we can write the following function:

\begin{lstlisting}
function isstudent (x: Person | Student) : Bool =
  switch (x)
    inj1 person -> False
    inj2 student -> True
\end{lstlisting}

But if we transform this function straightforwardly to type-directed
elimination, we will get:

\begin{lstlisting}
function isstudent2 (x: Person | Student) : Bool =
  switch (x)
    Person  -> False
    Student -> True
\end{lstlisting}

\noindent Now it is unclear what would happen if we apply \lstinline{isstudent2}
to a term of type \lstinline{Student}, as its type matches both branches. In
some calculi~\citep{dunfield2014elaborating}, the choice is not determined in
the semantics, in the sense that either branch can be chosen. This leads to a
non-deterministic semantics. In some other languages or
calculi~\citep{castagna:settheoretic}, branches are inspected from top to
bottom, and the first one that matches the type gets chosen. However, in those
systems, as \lstinline{Person} is a supertype of \lstinline{Student}, the first
branch subsumes the second one and will always get chosen, and so the second
branch will never get evaluated! This could be unintentional, and similar
programs being accepted could lead to subtle bugs. Even if a warning could be
given to alert programmers that a case can never be executed, there could be
other situations where two cases overlap, but neither case subsumes the other.
For instance we could have \lstinline{Student} and \lstinline{Worker} as
subtypes of \lstinline{Person}. For a person that is both a student and a
worker, a switch statement that discriminates between workers and students could
potentially choose either branch. However for persons that are only students or
only workers, only one branch can be chosen.

Some languages support form of typed-based union elimination via overloading,
which usually finds in all branches the one that best describes the type. Such
approach is used in, for example, resolving overloading in Java \cite{javadoc}.
Then, if we apply \lstinline{isstudent2} to a term of type \lstinline{Student},
the second branch is chosen, as \lstinline{Student} is the best type describing
\lstinline{x}. However, even with the best-match strategy, the semantics can be
confusing when the system features implicit upcasting (e.g., by subtyping). For
example, suppose the type \lstinline{Pegasus} is a subtype of both type
\lstinline{Bird} and type \lstinline{Horse}. If we apply a switch expression on
the type \lstinline{Bird | Horse} to a term of type \lstinline{Pegasus}, which
branch should we choose, the one with \lstinline{Bird}, or the one with
\lstinline{Horse}? In such case, the semantics is non-deterministic again, and
depends on the particular implementation of the compiler.


\subsection{Eliminating Union Types in Ceylon}

The Ceylon language~\cite{} supports type-directed union elimination by a
switch expression with multiple branches. The following program is an example
with union types using Ceylon's syntax:

\begin{lstlisting}
	void print(String|Integer|Float x) {
		switch (x)
		case (is String) { print("String: ``x``"); }
		case (is Integer|Float) { print("Number: ``x``"); }
	}
\end{lstlisting}
%

For the switch expression, Ceylon enforces static type checking with two
guarantees: \textit{exhaustiveness}, and \textit{disjointness}. First, Ceylon
ensures that all cases in a switch expression are exhaustive. In the above
example, \lstinline{x} can be either a string, an integer or a floating point
number. The types used in the cases do not have to exactly match with the types
of \lstinline{x}. Nevertheless, the combination of all cases must be able to
handle all possibilities. That means, if the last case only dealt with
\lstinline{Integer} (instead of \lstinline{Integer|Float}), then the program
would be statically rejected, since no case would deal with a floating point
number.

Second, Ceylon enforces that all cases in a switch expression are
\textit{disjoint}. That is, unlike the approaches described in
Section~\ref{subsec:elimination}, in Ceylon, it is impossible to have two
branches that match with the input at the same time. For instance, if the first
case used the type \lstinline{String | Float} instead of \lstinline{String}, the
program would be rejected statically with a type error. Indeed, if the program
were to be accepted, then the call \lstinline{print(3.0)} would be ambiguous,
since two branches could deal with the floating point number. Note that, since
the cases in a switch cannot overlap, their order is irrelevant to the program's
behavior and its evaluation result.

\subsubsection*{Union Types as an Alternative to Overloading}
A motivation of such type-directed union elimination in Ceylon is to
model a form of function overloading.
The following example, which is adapted from TypeScript's documentation\footnote{https://www.typescriptlang.org/docs/handbook/unions-and-intersections.html},
demonstrates how to define an ``overloaded'' function \lstinline{padLeft},
which adds some padding to a string. The idea is that there can be two versions
of \lstinline{padLeft}: one where the second argument is a string; and
the other where the second argument is an integer:

\begin{lstlisting}
String space(Integer n){
	if (n==0) {
		return "";
	} else {
		return " "+space(n-1);
	}
}

String padLeft(String v, String|Integer x){
	switch (x)
	case (is String) { return x+v; }
	case (is Integer) { return space(x)+v; }
}

print( padLeft("?", 5) ); // "     ?"
print( padLeft("World", "Hello ") ); // "Hello World"
\end{lstlisting}
%

In the two cases of the switch construct, there are two different implementations
of the \lstinline{padLeft} function: one that appends a string to the left of \lstinline{v},
and the other that calls function \lstinline{space} to generate a string with \lstinline{x} spaces,
and then append that to \lstinline{v}.
Although statically \lstinline{x} has type \lstinline{String|Integer}, as a concrete value 
it can only be a string or an integer.
As such, when values with such types are passed to the function,
the corresponding branch is chosen and executed.

\subsubsection*{Other Applications of Union Types}
Besides being used for overloading, union types can be used for other purposes to.
For instance, we can easily encode the \lstinline{safediv} function in Section~\ref{subsec:elimination}
in Ceylon:
\bruno{Ceylon code for safediv and to string here, and some short piece of text about the
example}
%
\begin{lstlisting}
String | Integer safediv3 (Integer x, Integer y){
	if (y==0) {
		return "Divided by zero";
	} else{
		return (x/y);
	}
}
\end{lstlisting}
%
The return value can be a string or an integer.
No tag is needed for any of them to be a \lstinline{String|Integer},
since union types can be implicitly introduced.
For the declared return type of the function, as long as it is
a supertype of all possible return values, it is valid
in Ceylon.


Furthermore, another interesting application
of union types in Ceylon is to encode nullable types (or optional types)
in a type-safe way.
A similar approach to nullable types has also been recently proposed for Scala~\citep{nieto20nulls}.
The \lstinline{null} value is inhabited by the type \lstinline{Null}.
For type \lstinline{A} to be nullable, one can write \lstinline{A?},
which stands for \lstinline{A|Null}.
If we eliminate a value of type \lstinline{A|Null}, there has to be a branch
to handle the null value.
Ceylon define the top type \lstinline{Anything} as an enumerated class.
\lstinline{Null} is a case of it.
It differs to the bottom type \lstinline{Nothing} in the sense
that it is inhabited.
\lstinline{Anything} has another distinct case called \lstinline{Object}.
It is the root of primitive types, function types, all interfaces and any user-defined class.
\lstinline{Null} is disjoint to \lstinline{Object}, and therefore, to all
these types.
The following diagram presents their subtyping relation briefly.

\begin{tikzcd}
	& \begin{smallmatrix}     0 & 0 & 0 \\      & 0 &            \end{smallmatrix}  \arrow[rd] &  & \begin{smallmatrix}     Anything \\      &  &            \end{smallmatrix}  \arrow[rd] &  &  \\
	&  & \begin{smallmatrix}     Null \\      &  &            \end{smallmatrix}  \arrow[rd] \arrow[ru] &  & \begin{smallmatrix}     Object \\      & &            \end{smallmatrix}  \arrow[rd] &  \\
	& \begin{smallmatrix}     0 & 0 & 0 \\      & 1 &            \end{smallmatrix}  \arrow[ru] \arrow[rd] &  & \begin{smallmatrix}     Char \\      &  &            \end{smallmatrix}  \arrow[ru] \arrow[rd] &  & \begin{smallmatrix}     Integer \\      &  &            \end{smallmatrix}  \\
	\begin{smallmatrix}     1 & 1 & 1 \\      & 1 &            \end{smallmatrix}  \arrow[ru] &  & \begin{smallmatrix}     0 & 0 & 0 \\      & 1 &            \end{smallmatrix}  \arrow[ru] &  & \begin{smallmatrix}     Nothing \\      & &            \end{smallmatrix}  \arrow[ru] &
\end{tikzcd}

\bruno{Say more about this, maybe show a (simplified)
  picture of the subtyping lattice?
The point is that Null is not a subtype of Object, right? so, by default object
types have no null value. }
%Similarly, one can use enumerated types to denote various special cases, or easily
%add another type to the argument type of an existing function when considering
%more possible inputs, to improve the program's robustness.
%Next, we will see how Ceylon's static type checking helps programmers.

\begin{comment}
\subsubsection*{Exhaustiveness}

Ceylon checks the exhaustiveness of a switch by comparing the union of all
cases and the switched term's type.
For the switch to be accepted, the former must be a supertype of the later.
%Recall that adding more components to a union type make it a supertype of the
%initial type, like \lstinline{Integer|Float} to \lstinline{Integer}.
That is to say a switch construct must have enough cases to handle all
possible runtime types of the term.
%
For example, here we define \lstinline{Node} by enumerating its subtypes.
It can be viewed as the union of \lstinline{Leaf} and \lstinline{Branch}.
Then in the function \lstinline{printTree} that takes a \lstinline{Node},
both cases are taken into consideration.
%
\begin{comment}
------------- THE OTHER EXAMPLE -----------------
Adding more subtype in it causes error in exhaustiveness checking in switch.
\begin{lstlisting}
interface Resource of File | Directory | Link { }
interface File satisfies Resource {}
interface Directory satisfies Resource {}
interface Link satisfies Resource {}

void printType(Resource resource){
	switch (resource) 
	case (is File) { print("File"); }
	case (is Directory) { print("Directory"); }
	case (is Link) { print("Link"); }
}
\end{lstlisting}

%
\begin{lstlisting}
abstract class Node() of Leaf | Branch {}

class Leaf(shared Object element) 
extends Node() {}

class Branch(shared Node left, shared Node right) 
extends Node() {}

void printTree(Node node) {
	switch (node)
	case (is Leaf) {
		print("Found a leaf: ``node.element``!");
	}
	case (is Branch) {
		printTree(node.left);
		printTree(node.right);
	}
}

printTree(Branch(Branch(Leaf("aap"), Leaf("noot")), Leaf("mies")));
\end{lstlisting}
% https://ceylon-lang.org/documentation/1.3/tour/types/
%
We can allow the input to be \lstinline{null} by replacing the argument type \lstinline{Node} by \lstinline{Node?}.
Ceylon uses union types to encode nullable types (or the optional types) in a 
type-safe way.
The null value is inhabited in \lstinline{Null}, and \lstinline{A?} stands for \lstinline{A|Null}.
If the switched term has a nullable type, the exhaustive checking makes sure there
is a branch to handle it.
%
Similarly, one can use enumerated types to denote various special cases, or easily
add another type to the argument type of an existing function when considering
more possible inputs, to improve the program's robustness.
%
If we can add more subtype in the declaration of \lstinline{Node} without
adapting the function definition, an compiling error will be raised:
case types must cover all cases of the switch type.
Such checking reminds programmers to keep consistent when changing the
related code and avoid potential runtime errors.

\bruno{The following example is better, and I guess useful to illustrate exhaustiveness.}
\begin{lstlisting}
	void printAfterPlusOne(Integer|String x) {
		switch (x)
		case (is Integer|Float) { print(x+1); }
		case (is String) { print("String:"+x); }
	}
\end{lstlisting}
%
Exhaustiveness checking does not prevent the cases in a switch to 
accept more than necessary.
For example, in the above function, even though \lstinline{x} cannot be
a \lstinline{Float}, it is not harmful for the first branch to expect
a term of \lstinline{Integer|Float}, since \lstinline{Integer|Float|String}
still covers \lstinline{Integer|String}.


\subsubsection*{Disjointness}
Ceylon prevents any pair of cases in a switch from overlapping.
And disjointness is introduced to describe such relation between two types.
Being disjoint means the non-existence of a value which can be assigned to
both types.
With the existence of subtyping, a branch in switch/case expression
can take a term of a subtype of its expected type.
Therefore disjointness roughly equals to no common subtype.
However, this does not directly lead to an algorithm. So Ceylon
provides a set of rules: 1) distinct cases in an enumerated type is thought to
be disjoint, and their subtypes are disjoint too; 2) two classes, if they are
not subclass of each other, are disjoint; and etc.
Especially, for a union type $[[A1\/A2]]$, being disjoint with another
type $B$ requires both $[[A1]]$ and $[[A2]]$ to be disjoint with $B$.
Eventually, after decomposing and examining subtyping relation, it is decidable
that two types are disjoint or not.

% https://github.com/ceylon/ceylon-spec/issues/50
% https://github.com/ceylon/ceylon-spec/issues/65

Disjointness is the foundation of
Ceylon's deterministic and order-irrelevant semantics of the switch construct.
Forcing all cases to be disjoint eliminates ambiguity, and avoid subtle bugs that may arize from overlapping cases.
%
People may argue that the programmer should be free to use overlapping cases
and arrange their order intentionally. The following example provides a scenario
where the programmer may not notice the dangerous overlapping.
%
\begin{lstlisting}
	alias Number => Integer | Null;
	String getNumber(Number n) {
		switch (n)
		case (is Integer) { return "``n``"; }
		case (is Null) { return "infinity"; }
		
	}
	void  printAge(Number|Null x) {
		switch (x)
		case (is Number) { print("Age is ``getNumber(x)``".); }
		case (is Null) { print("Age is not provided."); }
	}
\end{lstlisting}
%
Assume \lstinline{Number} is already defined where \lstinline{null} stands for
infinity. A client uses it to store age information. 
Meanwhile, the client use \lstinline{null} to denote missing data.
Without disjointness checking, the above code will be accepted.
But in \lstinline{printAge}, the second branch is always shadowed by the first one.
Any missing data will be interpreted as infinity.
%
Note that swapping the two cases cannot prevent all unexpected behaviors.
In that case the meaning of infinity is hidden.
On contrary, disjoint cases are always distinguishable, and the user, when
using an existing definition, is guaranteed that there is no hidden conflicts.

\subsubsection*{The Bottom Type}
To dig deeper for disjointness, it often comes to the the concept
of \emph{bottom type}.
Bottom type is a subtype of all types, in contrast to the top type.
Certainly the bottom type has no value.
In Ceylon, it is called \lstinline{Nothing}, representing the empty set.
%
For two disjoint types $[[A]]$ and $[[B]]$, their intersection $[[A/\B]]$
is naturally a common subtype, and therefore must be equivalent to
the bottom type \lstinline{Nothing}.
\end{comment}

\begin{comment}
\subsubsection*{Existing Problems in Ceylon}
In general, a term of type $A$ is always assignable to any supertype of $A$.
But in Ceylon, the checking of assignability is not complete to
subtyping.
Although the subtyping relation holds between \lstinline{v}'s
(declarative) type and \lstinline{Integer}, \lstinline{v}
is not assignable to \lstinline{Integer}, and the following program
cannot be accepted by Ceylon's compiler.
% https://try.ceylon-lang.org/#

\begin{lstlisting}
	< Character | Integer > & < String | Integer > v = 100;
	switch (v)
	case (is Integer) { print("Integer: ``v``"); }
\end{lstlisting}
\end{comment}

\subsection{Our Methodology}

The switch construct in our calculus \cal is similar to Ceylon's.
Its typing rule guarantees that cases are disjoint and exhaustive.
Reduction preserves types and produces deterministic result in the
runtime, with the help of annotated values.
Here we give an overview of our design and discuss some challenges
we met for the two calculi in the paper.

\subsubsection*{Disjointness, Interacted with Intersection Types}
After seeing the connection between bottom-like types and disjointness,
it is intuitive to formally define disjointness via bottom-like types.

\begin{definition}\label{def:disjointness}
	A $*_s$ B $\Coloneqq$ $\forall$ C, $[[C <: A]]$ $\wedge$ $[[C <: B]]$ $\rightarrow$ $[[botlike C]]$
	\label{def:union:disj}
\end{definition}

\snow{Here I copy the definition from union.tex. Maybe we can introduce it
earlier.}

Two types are disjoint if and only if all of their common subtypes are bottom-like.
That is to say, there does not exist any term that is assignable to both of
them. Specially, a bottom-like type is disjoint to any types, while the top type
is only disjoint to bottom-like types.
For the detailed discussion and an algorithmic formalization of
disjointness, please refer to Section~\ref{sec:union:disj}.

In literature, there exist a very different definition of disjointness
in calculi with intersection types that also serves for disambiguity
purpose~\cite{oliveira2016disjoint}.
In contrary to subtypes, it restricts the common supertype, or the lowest
upper bound, of two disjoint types to be \emph{top-like}.
Similarly, a top-like type is equivalent to type top, which is the greatest
upper bound of all types.
The difference comes from the subtyping rules for intersections.
Any component of an intersection type is a supertype of it.
If two intersection types share a part, e.g. $[[Int/\Char]]$ and $[[String/\Int]]$
both contains $[[Int]]$, they cannot pass the disjointness checking.
Moreover, assuming $[[Odd]]$ and $[[Even]]$ denote odd numbers and even numbers,
they are both subtype of $[[Int]]$, and therefore are not disjoint.
The application of this definition is dual to our definition:
given a type that is not top-like, consider a scenario where we are looking for
a term that is assignable to the type. If all candidates' types are disjoint, 
at most one term can be chosen from them.
\begin{verbatim}
 (\x. x+1 : Int->Int) (1 ,, True) --> (1+1):Int
\end{verbatim}

We alter the definition of disjointness (see Section~\ref{sec:inter:disj})
in our second calculus, which extends \cal by intersection types.
Any two types $[[A]]$ $[[B]]$ have a trivial subtype $[[A/\B]]$.
Generally such an intersection type is not bottom-like.
A new representation for types with no inhabited values is then needed.
At the end, we come up with a concept that is more like the opposite to it:
we use \emph{ordinary} to denote types that have corresponding values.
%
If two types have a common subtype that is ordinary, any inhabited
value of the ordinary type is assignable to them.
Thurs, any ordinary types cannot be a subtype of two disjoint types at
the same time.
%
Literal types are ordinary, like $[[Int]]$.
All function types are viewed as ordinary for simplification.
Compound types like intersection or union types are not included in ordinary
types because they must have subtype that is a literal type if they have
inhabited values.
%
Then we can prove $[[Int/\String\/Bool]]$ is disjoint with $[[Int/\String\/Char]]$.
Although they have subtype $[[Int/\String]]$, it is not ordinary.
And $[[Int\/String]]$ is not disjoint with $[[Bool\/String]]$, because of the
subtype $[[String]]$.
Especially, types corresponding to no values, like $[[Int/\String]]$, are
disjoint with any types.

We cannot enumerate all subtypes of two types to check disjointness.
Instead we have an algorithmic definition of disjointness (Figure~\ref{fig:union:disj-typ}) for \cal.
These rules structurally examine two given types, decomposing unions
and comparing their literal components.
For example, $[[Int\/String]]$ is disjoint to $[[Char]]$ because both types
in it is disjoint to $[[Char]]$.
%
However, this approach cannot be directly employed to the intersection
extension. 
Like we said above, some intersection is disjoint to other types
because itself is uninhabited.
So even neither $[[Int]]$ or $[[String]]$ is disjoint with $[[Int\/String]]$,
together $[[Int/\String]]$ is disjoint with it. 
Alternatively, for a given type, we use a set of ordinary types to denote
all possible values of it. For instance, \verb"{Int,Char}" represent the
union type $[[Int\/Char]]$.
The set is calculated by function \lstinline{LOS} (\emph{lowest ordinary
subtypes})in Section~\ref{}.
To check whether two types are disjoint, we calculate the intersection
of their sets and see if it is empty.
From another angle, we normalize and simplify types by \lstinline{LOS},
and then are able to directly detect uninhabited ones.

The set representation is justified by the distributivity in subtyping (Figure~\ref{}).
Every type in the set denotes a component in a union type.
That is to say, any inhabited type must have an equivalent union type,
except for literal types.
Only with distributivity, we have $[[(A\/B)/\C]]$ equivalent to 
$[[(A/\C)\/(B/\C)]]$.
The soundness and completeness of the algorithmic definition is proved
with respect to the declarative definition.
Same applies to the two definitions for \cal without intersection.


\subsubsection*{Typing and Exhaustiveness}
In \cal, a switch expression has two branches. For multiple cases,
one can write nested switch expression.
We assume the two branches expect $[[A]]$ and $[[B]]$.
To make sure they exhaust all possible types of the switched term $[[e]]$,
there is a premise that $[[e]]$ can be checked by $[[A\/B]]$.
\verb|<=| stands for checking mode in bidirectional typing,
on contrary to inference mode.
In other words, the inferred type of $[[e]]$ should be a subtype of $[[A\/B]]$,
like $[[Int]]$ to $[[Int\/Char]]$.
%
\begin{mathpar}
	\ottdruletypXXswitch{}
\end{mathpar}
%
Another premise requires the two cases to be disjoint.
Besides, the two branches are typed under different assumption of the bound
variable. Although the same type $[[C]]$ is used for both of them in the rule,
it does not prevent them to return different types.
Assuming the inferred type of $[[e1]]$ is $[[C1]]$ and the inferred type of 
$[[e2]]$ is $[[C2]]$, we can make $[[C]]$ to be $[[C1\/C2]]$.

\subsubsection*{Reduction and Annotated Values}
Our reduction preserves type.
So a term of union type can only evaluates to a value of an union type.
We use type annotations to construct such values.
For example, $[[2:Int\/Bool]]$ is a value, and it is the result of
$[[(switch 1 Int (x p1) Bool (neg y)):Int\/Bool]]$.

\snow{I think we might be able to relax preservation by allowing subtyping.
For which precise reason we failed in that variant? But I guess it is ok
to omit that unless the reviewers ask}

Lambdas can be annotated. But if a lambda has inferred type, it must
have two annotation: the outer one is for the preservation purpose,
while the inner one records its principal type.
%
An example is $[[\x.x:Int->Int:Int->Int\/Bool]]$. 

\snow{I don't know why we need to keep a lambda's original type. We
don't need to distinguish two arrow types in switch.}

Deciding to take which branch in the runtime requires knowing the
precise type of the switched term.
Instead of the annotation, we need to look into the wrapped value
(which is called as \emph{pre-value}) for its principal type.
Given $[[2:Int\/Bool]]$, we still know that it is an integer inside.
We then look for a branch that expects a supertype of $[[Int]]$.
As previously discussed, the exhaustiveness checking promises at least
one branch matches.
And the disjointness restriction ensures it can only be one branch.
Therefore the reduction is deterministic.
If the two branch types are both supertype of the value type, they violates
the definition, no matter the Definition~\ref{def:disjointness} or the
improved one.
In the following example, the first branch will be chosen, just like when
we passing $[[1]]$ to it.

\begin{verbatim}
	(switch (2:Int\/Bool) Int (x p1) Bool (neg y)) 
	--> (2:Int p1) :Int\/Bool
	--> 3 : Int \/ Bool
\end{verbatim}

Wrapping values with annotation also helps us to unify the related reduction rules.
\snow{I guess so?}



\bruno{
1) use of annotatted values for both reduction (the semantics must be type directed); and preservation;
2) Disjointness based on disjoint intersection, and later novel notion of disjointness in the presence of intersections;
3) LOS for algorithmic disjointness;
4) Dealing with exhaustiveness and reduction in the switch construct
}

\begin{comment}
\bruno{For the following text, this is the level of detail that I expect
  for the introduction, but not for the overview! The overview should
  identify key ideas and challenges in more detail. So the following text
  is not what I'm hoping for here. For an example of what I'm hoping for here,
  see for instance Section 2.3 and 2.4 in ``Disjoint Intersection Types'' or
  Section 2.3, 2.4 and 2.5 in "A Type-Directed Operational Semantics for a
  Calculus with a Merge Operator''
}
In this paper, we follow type-based union elimination similar to Ceylon.
We enforce disjointness constraint on the types of branches of a switch
expression. On the contrary to Ceylon, we provide a formal definition
of disjointness. Our disjointness definition is inspired by $\lambda_{i}$ 
\cite{oliveira2016disjoint}. $\lambda_{i}$ formally defines disjointness
for intersection types and merge operator.
Intersection of two types $[[A]]$ and $[[B]]$ is dijoint in $\lambda_{i}$
if $[[A]]$ and $[[B]]$ do not share a common supertype which is not
\emph{top-like}. So-called \emph{top-like} types are defined in 
\cite{oliveira2016disjoint} and are such types which are \emph{supertypes}
of all other types.

Union types are usually considered to be dual of intersection types.
Therefore, we propose a dual definition of \emph{bottom-like} types.
In contrast to \emph{top-like} types, \emph{bottom-like} types are
subtypes of all other types. Further, in contrast to the disjointness
definition in $\lambda_{i}$, two types $[[A]]$ and $[[B]]$ in the simplest 
caluclus studied in this paper are disjoint if $[[A]]$ and $[[B]]$
do not share any common \emph{subtype} which is not \emph{bottom-like}.
Note that $\lambda_{i}$ proposed \emph{top-like} with \emph{supertypes}.
We propose \emph{bottom-like} with \emph{subtypes}. This will be discussed
in detail in \Cref{sec:union}.

Adding intersection types in our calculus
together with union types poses non-trivial challenges on disjointness.
Specifically, it makes it impossible to define a complete disjointness
definition. Therefore, the simple disjointness definition which is dual
to $\lambda_{i}$ no longer works. We define a notion of ordinary types
and define disjointness based upon ordinary types to overcome the
challenge in completeness. Informally, updated disjointness definition states
that two types $[[A]]$ and $[[B]]$ are disjoint if they do not share any
common ordinary subtype. This will further be discussed in detail in
\Cref{sec:inter}.

We propose a sound and complete disjointness definition. Caclulus studied
in this paper is type-safe and deterministic.

\bruno{
	Introduce our work, setting the goal to study the construct formally.
	Connect with the work on disjoint intersection types, which also
	employs a notion of disjointness, but for intersection. Explain that
	what is needed is a dual notion of disjointness.\\
	Introduce the first calculus, and explain that it is directly inspired
	by a dual notion of discjointness.\\
	Introduce the second calculus and identify a technical challenge with
	disjointness: the addition of intersection types breaks the previous
	notion of disjointness. Introduce the novel way to find disjoint types.\\
	Summarize/mention key results: type-safety; soundness/completeness of
	disjointness; determinism.\\
	Perhaps here it is also useful to identity, together with Baber, what
	were the most challenging aspects in the formalization, and maybe
	highlight these.
}

\end{comment}
%%% Local Variables:
%%% mode: latex
%%% TeX-master: "../paper"
%%% org-ref-default-bibliography: "../paper.bib"
%%% End:
