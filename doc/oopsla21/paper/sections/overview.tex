\section{Overview}
\label{sec:overview}

\ningning{We may want to be consistent with the syntax used in examples. Now
  there are three things going on: examples before introducing Ceylon; examples
  in Ceylon; examples in our calculi. Will it look confusing as we use three
  different syntax rules?}
\bruno{Three different syntaxes is definitly too much. I think we may be able
  to get by with two syntaxes: our calculus and Ceylon. Alternatevely, we can
  use just the syntax in our calculus here, and later, perhaps in the discussion
  have examples in Ceylon, when comparing to Ceylon. }

\subsection{Tagged Union Types}

% Brief intro to union types and one or 2 simple examples that show that we
% can automatically lift values into union type without the need of some
% tag/constructor. Something like:

We start with a brief introduction to union types. An expression has a union
type $[[A\/B]]$, if it can be considered to have either type $[[A]]$ \textit{or}
type $[[B]]$. Many systems model \textit{tagged union types} (also called
\textit{sum types} or \textit{variants types}), where explicit \textit{tags}
are used to construct terms with union types. Languages with algebraic datatypes~\cite{}
or (polymorphic) variants~\cite{} support tagged union types. 
In their basic form, there are two introduction forms:
$\mathsf{inj_1} :: [[A -> A \/ B]]$ turn the type of an expression of type
$[[A]]$ into type $[[A \/ B]]$; and $\mathsf{inj_2} :: [[B -> A \/ B]]$
turns the type of an expressions of type $[[B]]$ into type $[[A \/ B]]$.
For example, we can have:

\begin{lstlisting}
inj1 "foo": String | Int
inj2 1 : String | Int
\end{lstlisting}

\noindent Using tagged union types, we can implement safe division as:

\begin{lstlisting}
function safediv (x : Int) (y : Int) : String | Int =
  if (y == 0) then inj1 "Divided by zero"
  else inj2 (x / y)
\end{lstlisting}

\noindent Here the intention is to have a safe (integer) division operation that detects
division by zero errors, and requires clients of this function to handle
such errors. The return type \lstinline{String | Int} denotes that the function
can either return an error message (a string), or an integer, when division
is performed without errors.

\paragraph{Eliminating tagged union types.}
Tagged union types are eliminated by some form of case analysis.
For consistency with the rest of the paper, we use
\lstinline{switch} statements here for such case analysis. For example,
the following program \lstinline{tostring} has different behaviors depending on the
tag of \lstinline{x}, where \lstinline{show} takes an \lstinline{Int} and
returns back its string representation.

\begin{lstlisting}
function tostring (x: String | Int) : String =
  switch x of
    inj1 str -> str
    inj2 num -> show num
\end{lstlisting}

Combining union type construction in \lstinline{safediv} and its elimination in
\lstinline{tostring}, we can easily implement an interface which returns the
result of safe division as one \lstinline{String}.

\begin{lstlisting}
> tostring (safediv 42 2)
"21"
> tostring (safe 42 0)
"Divided by zero"
\end{lstlisting}


\subsection{Type-directed Elimination of Union Types}

% Motivate the need for having a construct that can eliminate union types,
% perhaps trying to use an example where the String above would be a kind
% of exception, and the Int would be regular computation. Alternatively
% find an existing example from the literature.

% Discussing existing approaches for eliminating union types;
% point out that elimination constructs based on types (our focus) vs
% elimination constructs based on tags (which are used in algebraic
% datatypes or polymorphic variants (like in OCaml)).
% The focus should be on type-based approaches.

% Try to identify some limitations/problems. For instance how to deal
% with ambiguity (just use order? restrict the construct somehow ...).


While tags are useful to make it explicit which type a value belongs to, they
also add clutter in the programs, and usually require extra space to store the
tags. On the other hand, in systems with subtyping for union types
\cite{dunfield2014elaborating,pierce1991programming,muehlboeck2018empowering},
explicit tags are replaced by implicit coercions represented by the two
subtyping rules $[[A <: A \/ B]]$ and $[[B <: A \/ B]]$. We call such types
\textit{untagged union types}, or simply \textit{union types}. In those systems,
a term of type $[[A]]$ or $[[B]]$ can be directly used as if it had type $[[A \/
B]]$, and thus we can model safe division as

\begin{lstlisting}
function safediv2 (x : Int) (y : Int) : String | Int =
  if (y == 0) then "Divided by zero"
  else (x / y)
\end{lstlisting}

\noindent However, now elimination of union types cannot rely on tags anymore, and
different systems implement elimination differently. We review some of the
existing approaches next.

\paragraph{Single branch elimination.}

In \citet{pierce1991programming,dunfield2014elaborating}, elimination of an
expression with a union type $[[A \/ B]]$ supports only one branch, and the
branch needs to have the same type when the expression has type $[[A]]$, or type
$[[B]]$. For example\footnote{For better illustration, we adapt their syntax
  using our notations in the examples, while preserving their semantics.}, in
\lstinline{tostring2}, the expression \lstinline{show x} must return
\lstinline{String} when \lstinline{y : String}, and when \lstinline{y : Int},
which means \lstinline{show} must be overloaded (in
\citet{pierce1991programming,dunfield2014elaborating}, this can be implemented
by requiring \lstinline{show} to have an \textit{intersection type}).

\begin{lstlisting}
function tostring2 (x: String | Int) : String =
  switch x of
    y -> show y
\end{lstlisting}

This implementation is concise, but it is also restrictive as it can no longer
support multiple branches according to the different representations of
\lstinline{x}. Furthermore it relies on the language also supporting overloaded
functions. Without overloaded functions the construct would not be very useful.

\paragraph{Type-directed elimination.}

On the other hand, some systems [?] support \textit{type-directed} elimination
of union types. For instance, \lstinline{tostring3} have different behaviors
depending on the \textit{type} of \lstinline{x}.

\begin{lstlisting}
function tostring3 (x: String | Int) : String =
  switch x of
    String -> x
    Int -> show x
\end{lstlisting}

\noindent Note here \lstinline{show} is directly applied to \lstinline{x}, and
it does not need to be overloaded, as the type-directed elimination
\textit{refines} the type of \lstinline{x} from \lstinline{String | Int} to
\lstinline{Int} in the second branch (such type refinements are also used in
\textit{occurrence typing} \cite{Tobin:typedscheme}).

However, compared to tag-directed elimination, extra care must to taken with
type-directed elimination. In particular, while we can easily distinguish tags,
in type-directed elimination, ambiguity may arise when types in a union type
overlap. For example, consider the type \lstinline{Person | Student}, where we
assume \lstinline{Student} is a subtype of \lstinline{Person}. With tag-directed
elimination, we can write the following function:

\begin{lstlisting}
function isstudent (x: Person | Student) : Bool =
  switch x of
    inj1 person -> False
    inj2 student -> True
\end{lstlisting}

But if we transform this function straightforwardly to type-directed
elimination, we will get:

\begin{lstlisting}
function isstudent2 (x: Person | Student) : Bool =
  switch x of
    Person  -> False
    Student -> True
\end{lstlisting}

\noindent Now it is unclear what would happen if we apply \lstinline{isstudent2}
to a term of type \lstinline{Student}, as its type matches both branches. In some
calculi [?], the choice is not determined in the semantics, in the sense either
branch can be chosen, which leads to a non-deterministic semantics. In some
other languages or calculi [?], branches are inspected from top to bottom, and the first one
that matches the type gets chosen. However, in those systems, as
\lstinline{Person} is a supertype of \lstinline{Student}, the first branch
subsumes the second one and will always get chosen, and so the second branch
will never get evaluated! This could be unintentional, and similar programs  
being accepted could lead to subtle bugs. Even if a warning could be given
to alert programmers that a case can never be executed, there could be
other situations where two cases overlap, but neither case subsumes the other.
For instance we could have \lstinline{Student} and \lstinline{Worker} as
subtypes of \lstinline{Person}. For a person that is both a student and a worker,
a switch statement that discriminates between workers and students could
potentially choose either branch. However for persons that are only students
or only workers, only one branch can be chosen.

\subsection{Eliminating Union Types in Ceylon}
Ceylon supports type-directed union elimination via a switch expression with
multiple branches. Each branch matches on a type.
In the runtime, one branch is chosen according to the input's type.
\begin{lstlisting}
	void print(String|Integer|Float x) {
		switch (x)
		case (is String) { print("String: ``x``"); }
		case (is Integer|Float) { print("Number: ``x``"); }
	}
\end{lstlisting}
%
Static type checking for the switch construct ensures two things: the 
list of cases is exhaustive, and any two cases are disjoint.
In the above example, \lstinline{x} can be either a string or an integer,
or a floating number. There does not have to be exact three cases correspondingly,
but the combination of all cases must be able to handle all possibilities.
On the other hand, unlike the previous discussion, it is impossible to have two
branches matched with one input at the same time, thanks to the disjointness
restriction.

\subsubsection*{As an Alternative to Overloading}
A motivation of such type-directed union elimination in Ceylon is to
model a form of function overloading.
The following example demonstrate how to define a function \lstinline{padLeft}.
It takes two arguments. The second argument can be either a string or
an integer.

\bruno{I suggested using interesting examples from the documentation of Ceylon, but
  I think we only give artificial examples here. Please write a couple of more
  interesting examples (I gave quite a few pointers and examples from the documentation
  on slack) here!}

\begin{lstlisting}
String space(Integer n){
	if (n==0) {
		return "";
	} else {
		return " "+space(n-1);
	}
}

String padLeft(String v, String|Integer x){
	switch (x)
	case (is String) { return x+v; }
	case (is Integer) { return space(x)+v; }
}

print( padLeft("?", 5) );
print( padLeft("World", "Hello ") );
\end{lstlisting}
%
In the two cases of the switch construct, there are two different implementations
for the \lstinline{padLeft} function: one add a string to the left of \lstinline{v},
the other calls function \lstinline{space} to generate an string of \lstinline{x} spaces,
and then pad it to of \lstinline{v}.
Although statically \lstinline{x} has type \lstinline{String|Integer}, as a concrete value 
it can only be a string or an integer.
Then the corresponding branch is chosen and executed.
%
Besides, Ceylon uses union types to encode nullable types (or the optional types).
The null value is inhabited in \lstinline{A?}, which stands for \lstinline{A|Null}.
If the switched term has a nullable type, there has to be a branch to handle it.
Similarly, one can use enumerated types to denote various special cases, or easily
add another type to the argument type of an existing function when considering
more possible inputs, to improve the program's robustness.
Next, we will see how Ceylon's static type checking helps programmers.

\subsubsection*{Exhaustiveness}
Ceylon checks the exhaustiveness of a switch by compare the union of all cases and
the switched term's type. For the switch to be accepted, the former must completely
\emph{cover} the later. In other words, the switch term can be safely converted into
the union type. 
Here we define \lstinline{Node} by enumerating its subtypes. It can be viewed as the
union of \lstinline{Leaf} and \lstinline{Branch}.
\begin{comment}
------------- THE OTHER EXAMPLE -----------------
Adding more subtype in it causes error in exhaustiveness checking in switch.
\begin{lstlisting}
interface Resource of File | Directory | Link { }
interface File satisfies Resource {}
interface Directory satisfies Resource {}
interface Link satisfies Resource {}

void printType(Resource resource){
	switch (resource) 
	case (is File) { print("File"); }
	case (is Directory) { print("Directory"); }
	case (is Link) { print("Link"); }
}
\end{lstlisting}
\end{comment}
%
\begin{lstlisting}
abstract class Node() of Leaf | Branch {}

class Leaf(shared Object element) 
extends Node() {}

class Branch(shared Node left, shared Node right) 
extends Node() {}

void printTree(Node node) {
	switch (node)
	case (is Leaf) {
		print("Found a leaf: ``node.element``!");
	}
	case (is Branch) {
		printTree(node.left);
		printTree(node.right);
	}
}

printTree(Branch(Branch(Leaf("aap"), Leaf("noot")), Leaf("mies")));
\end{lstlisting}
% https://ceylon-lang.org/documentation/1.3/tour/types/
If we replace the argument type \lstinline{Node} in the function \lstinline{printTree} by
\lstinline{Node?},
or we add more subtype in the declaration of \lstinline{Node}, an compiling error
will be raised: case types must cover all cases of the switch type.
Such checking reminds programmers to keep consistent when changing the related code
and avoid runtime errors.

\bruno{The following example is better, and I guess useful to illustrate exhaustiveness.}
\begin{lstlisting}
	void printAfterPlusOne(Integer|String x) {
		switch (x)
		case (is Integer|Float) { print(x+1); }
		case (is String) { print("String:"+x); }
	}
\end{lstlisting}
%
Although the list of cases must be exhaustive, it does need to
strictly follow the input's type.
For example, in the above function, \lstinline{x} can only be an integer
or a string. But it is not harmful for the first branch to expect
a term of \lstinline{Integer|Float}, since \lstinline{Integer|Float|String}
still covers \lstinline{Integer|String}.

\subsubsection*{Disjointness}
Every pair of cases in a switch cannot overlap. Therefore, their order is
irrelevant to the program's behavior and its evaluation result.

	
\bruno{Please organize the section text using paragraphs, to highlight key topics.
  I think the flow of the text is quite bad and I can't quite figure out what points
  you are trying to make in different parts of the text. In fact it seems that you keep
  discussing related points in different parts of the text. Give the text structure
  and a flow by stating upfront (in a paragraph title) what is the point that you are trying
  to make in that paragraph. I can think of the following as themes/titles for paragraphs:
  1) Union types as an alternative to overloading; (motivate a little bit why Ceylon has
  such a construct).
  2) Disjointness; 3) Exhaustiveness 4) Problems with Ceylon's approach. 
}

Ceylon supports type-directed union elimination via a switch expression with
multiple branches. Each branch matches on a type.
\bruno{The following text is making some implicit assumptions: firstly it is not obvious
  why you are calling this an overloaded function: you would need to first explain that
  Ceylon uses union types as an alternative to model overloaded functions. I think you want
  to show the function declaration, rather than just a body. Then it is important to show
  the type of $v$ (and also why this function can be seen as an alternative to overloading).
  Without that it is not very clear what is being done here.}
Together the overloaded function is applied to a input.
The input has an union type statically.
But its runtime type can only fit in one branch.
That makes it different to the previous discussion. Why?
Because the types in all branches must be disjoint.

\begin{lstlisting}
  void print(String|Integer|Float x) {
    switch (x)
      case (is String) { print("String: ``x``"); }
      case (is Integer) { print("Integer: ``x``"); }
      case (is Float) { print("Float: ``x``"); }
    }
\end{lstlisting}
% https://ceylon-lang.org/documentation/1.3/tour/types/
% case (is Object) { print(v); }
% case (is Nothing) { print("nothing"); }

Being disjoint means it is impossible to find a value that matches both types.
The three types \lstinline{Integer}, \lstinline{Char}, and \lstinline{String} 
are disjoint, since no value can have more than one of the types. 
% Ensured by Ceylon's type system, \lstinline{v} of type \lstinline{Void} can 
% either be an \lstinline{Object} or \lstinline{Nothing}. 
With disjointness the order of branches does not matter, and
programmers can reorder the cases without affecting the semantics
of the program. More importantly, disjointness can avoid subtle bugs
that may arize from overlapping cases.

\bruno{I'm not sure that that what we want to say is what you say next.
  A reader may argue what if we want to model such an example. Actually
  Ceylon does allow you todo the example with an extension of the case construct.
  But I think what we want to illustrate here is that even with a disjoint construct
  we can do interesting things. This is why showing more interesting examples
  from the documentation and other ones that we discussed on slack is important. 
}
Take the previous function \lstinline{isstudent2} as an example.
For its definition to be accepted by Ceylon,
\lstinline{Person} and \lstinline{Student} cannot be subclass of
each other, assuming they are classes.
Since Ceylon does not allow multiple inheritance on classes,
the restriction is sufficient to prevent the incoherent scenario.
In the runtime, type of the input must be a subtype of either \lstinline{Person} 
or \lstinline{Student}. According to the subtyping relation,
one branch will be chosen and the input term will be cast
and substituted in.
%
%Besides primitive types and classes, 
%distinct members in an enumerated type are also viewed as disjoint.
% switch/case also handles other types as long as they are disjoint.
% Some primitive types like \lstinline{Integer}, \lstinline{Char}, and 
% \lstinline{String} are disjoint to others.
%
Forcing all cases in one switch/case expression to be disjoint makes them
interchangeable.
That could reduce confusion of program readers and eliminate a potential
source of bugs.

\bruno{Examples with polymorphism are not good because we don't cover polymorphism
  in this paper, nor we have disjointness constraints either. Also, I'm not quite
  sure about the point that you're trying to make with this example. You seem to keep
  going at overlapping. Are you trying to mention why the Ceylon team has decided
  to have a construct with disjoint cases? If so, be clear about that, and give references
  to the discussions that have motivated it! If the reason to write this example was
  to describe the motivations of the Ceylon team, then maybe it is ok to have the polymorphic
code. }
\begin{lstlisting}
	void <A> printA(A|Null x) {
		switch (x)
		case (is A) { print("an A"); }
		case (is Null) { print("nothing"); }
	}
\end{lstlisting}

For instance, one might expect the second branch to be executed if the above
function takes \lstinline{null} as input.
However, without the disjointness constraint, \lstinline{A} might be nullable itself.
An example is using \lstinline{Integer|Null} to denote a number or infinity
as a division result.
In that case the second branch is always shadowed by the first one.
Note that swapping the two cases cannot prevent all unexpected behaviors.
When the outside \lstinline{Null} has a different meaning other than infinity,
they are not distinguishable, and the branch designed for it will take them all.
% And instead of \lstinline{print}, one might call a function that takes care
% of \lstinline{Nothing} and the other component in \lstinline{A}.
This problem is even more serious when the language implements some syntax
sugar or encode other features via the switch/case construct.
It could be convenient to have a simple construct for unwrapping \lstinline{A|Null} without writing down the full switch/case expression.
But once the details are hidden, users may not be able to image that how does
the order affects the program, as we discussed above.  
On the other hand, with some requirements on the argument and return types, it is 
straightforward to support function overloading by combining them in a switch/case expression.
But it will be hard to find a fair criteria to decide which one
should be prioritized, if two functions can overlap.
% https://github.com/ceylon/ceylon-spec/issues/50
% https://github.com/ceylon/ceylon-spec/issues/65

\subsubsection*{Existing Problems in Ceylon}

In general, a term of type $A$ is always assignable to any supertype of $A$.
\bruno{The next text seems to switch to another point, which is issues with the
  Ceylon approach. So, what you should do is to start a new paragraph that clearly
identifies the change of topic. }
But in Ceylon, the checking of assignability is not complete to
subtyping.
Although the subtyping relation holds between \lstinline{v}'s
(declarative) type and \lstinline{Integer}, \lstinline{v}
is not assignable to \lstinline{Integer}, and the following program
cannot be accepted by Ceylon's compiler.
% https://try.ceylon-lang.org/#

\begin{lstlisting}
	< Character | Integer > & < String | Integer > v = 100;
	switch (v)
	case (is Integer) { print("Integer: ``v``"); }
\end{lstlisting}

\bruno{Now you come back to disjointness having spoken about disjointness earlier
  in this section. Don't jump around topics like this. If you have a paragraph
  on disjointness, then talk about everything that relates to disjointness under that
  paragraph. Or maybe have another paragraph called ``Bottom-like Types just after
disjointness, since that's clearly related and should follow.}
To dig deeper for disjointness, it often comes to the the concept
of \emph{bottom type}.
Bottom type is a subtype of all types, in contrast to the top type.
Certainly the bottom type has no value.
In Ceylon, it is called \lstinline{Nothing}, representing the empty set.
%
With the existence of subtyping, a branch in switch/case expression
can take a term of a subtype of its expected type.
If the term's type is also a subtype of the other branch's expected
type, type information will be insufficient to disambiguate.
To prevent this, disjoint types cannot have any common subtype which has
inhabited values.
For two disjoint types $[[A]]$ and $[[B]]$, their intersection $[[A/\B]]$
is naturally a common subtype, and therefore must be equivalent to
the bottom type \lstinline{Nothing}.

\begin{comment}
Let us call them \emph{bottom-like types}.
% Ceylon treat such types specially.
In a switch construct, every case has a type annotation.
In Ceylon, this type annotation is not allowed to be bottom-like.
The compiler uses this type to narrow the static type of the input term
in switch/case, and rejects the program if the resultant type is equivalent
to \lstinline{Nothing}.
\end{comment}

\bruno{
	Introduce the Ceylon approach with examples; introduce the idea of
	disjointness informally; contrast with existing constructs.
	Try to motivate the adapotion of disjointness (this should be discussed
	in the Ceylon documentation to some extent).\\
	Here it is important to give representative examples (look at documentation
	and/or other online resources), including examples
	that demonstrate how Ceylon uses union types to model overloaded functions.
}

\subsection{Our Methodology}

The switch construct in our calculus \cal is similar to Ceylon's.
Its typing rule guarantees that cases are disjoint and exhaustive.
Reduction preserves types and produces deterministic result with
the help of annotated values.
Here we give an overview of our design and discuss some challenges
we met for the two calculi in the paper.

\subsubsection*{Disjointness, Interacted with Intersection Types}
After seeing the connection between bottom-like types and disjointness,
it is intuitive to formally define disjointness via bottom-like types.

\begin{definition}\label{def:disjointness}
	A $*_s$ B $\Coloneqq$ $\forall$ C, $[[C <: A]]$ $\wedge$ $[[C <: B]]$ $\rightarrow$ $[[botlike C]]$
	\label{def:union:disj}
\end{definition}

\snow{Here I copy the definition from union.tex. Maybe we can introduce it
earlier.}

Two types are disjoint if and only if all of their common subtypes are bottom-like.
That is to say, there does not exist any term that is assignable to both of
them.
For the detailed discussion and an algorithmic formalization of
disjointness, please refer to Section~\ref{sec:union:disj}.

In literature, there exist a very different definition of disjointness
in calculi with intersection types that also serves for disambiguity
purpose~\cite{oliveira2016disjoint}.
In contrary to subtypes, it restricts the common supertype, or the lowest
upper bound, of two disjoint types to be \emph{top-like}.
Similarly, a top-like type is equivalent to type top, which is the greatest
upper bound of all types.
The difference comes from the subtyping rules for intersections.
Any component of an intersection type is a supertype of it.
If two intersection types share a part, e.g. $[[Int/\Char]]$ and $[[String/\Int]]$
both contains $[[Int]]$, they cannot pass the disjointness checking.
Moreover, assuming $[[Odd]]$ and $[[Even]]$ denote odd numbers and even numbers,
they are both subtype of $[[Int]]$, and therefore are not disjoint.
The application of this definition is dual to our definition:
given a type that is not top-like, consider a scenario where we are looking for
a term that is assignable to the type. If all candidates' types are disjoint, 
at most one term can be chosen from them.
\begin{verbatim}
 (\x. x+1 : Int->Int) (1 ,, True) --> (1+1):Int
\end{verbatim}


In our second calculus , Definition~\ref{def:disjointness} is adapted
for intersection types.
We are forced to do the change because with the existence of intersection types,
any two types $[[A]]$ $[[B]]$ have a trivial subtype $[[A/\B]]$. Generally such an intersection type is not bottom-like.
Our alternative definition of disjointness is presented in Section~\ref{sec:inter:disj}.
...


\snow{Next of the next: intersection types and distributivity}
\bruno{We don't need to say much about this, but maybe a few sentences are ok.}

LOS

LOS(top) = {Int, Bot -> Top}
LOS(Int -> Int) = ….

\subsubsection*{Typing and Exhaustiveness}

\snow{Next: How do we ensure exhaustive? Show the typing rule of switch/case.}
\bruno{Yes please: explain how exhaustiveness is ensured.}

\subsubsection*{Annotated Values}
\begin{verbatim}
	switch (1) {x:Int -> x+1, y:Bool -> not y} --> (1+1):Int
\end{verbatim}

in lambdai the return type must be different
\begin{verbatim}
	(\x. x+1 : Int->Int ,, \y. not y : Bool->Bool):Int->Int 1 --> (1+1):Int
\end{verbatim}

\bruno{
1) use of annotatted values for both reduction (the semantics must be type directed); and preservation;
2) Disjointness based on disjoint intersection, and later novel notion of disjointness in the presence of intersections;
3) LOS for algorithmic disjointness;
4) Dealing with exhaustiveness and reduction in the switch construct
}



\snow{Some concerns: does the addition of intersection types and distributivity
	really increase the expressiveness of the system?
	Currently the only interesting intersection type has the form of (A->B) \& (C->D),
	like (int->bool)\&(bool->int).
	But there is no way to construct a term of such type.
	The closest thing is $\lambda$ x.x, which has (int->int)\&(bool->bool).}
\bruno{Why don't you check examples from Ceylon? In any case, I don't think we need
to talk much about distributivity here: that is not our main focus.}

\bruno{For the following text, this is the level of detail that I expect
  for the introduction, but not for the overview! The overview should
  identify key ideas and challenges in more detail. So the following text
  is not what I'm hoping for here. For an example of what I'm hoping for here,
  see for instance Section 2.3 and 2.4 in ``Disjoint Intersection Types'' or
  Section 2.3, 2.4 and 2.5 in "A Type-Directed Operational Semantics for a
  Calculus with a Merge Operator''
}
In this paper, we follow type-based union elimination similar to Ceylon.
We enforce disjointness constraint on the types of branches of a switch
expression. On the contrary to Ceylon, we provide a formal definition
of disjointness. Our disjointness definition is inspired by $\lambda_{i}$ 
\cite{oliveira2016disjoint}. $\lambda_{i}$ formally defines disjointness
for intersection types and merge operator.
Intersection of two types $[[A]]$ and $[[B]]$ is dijoint in $\lambda_{i}$
if $[[A]]$ and $[[B]]$ do not share a common supertype which is not
\emph{top-like}. So-called \emph{top-like} types are defined in 
\cite{oliveira2016disjoint} and are such types which are \emph{supertypes}
of all other types.

Union types are usually considered to be dual of intersection types.
Therefore, we propose a dual definition of \emph{bottom-like} types.
In contrast to \emph{top-like} types, \emph{bottom-like} types are
subtypes of all other types. Further, in contrast to the disjointness
definition in $\lambda_{i}$, two types $[[A]]$ and $[[B]]$ in the simplest 
caluclus studied in this paper are disjoint if $[[A]]$ and $[[B]]$
do not share any common \emph{subtype} which is not \emph{bottom-like}.
Note that $\lambda_{i}$ proposed \emph{top-like} with \emph{supertypes}.
We propose \emph{bottom-like} with \emph{subtypes}. This will be discussed
in detail in \Cref{sec:union}.

Adding intersection types in our calculus
together with union types poses non-trivial challenges on disjointness.
Specifically, it makes it impossible to define a complete disjointness
definition. Therefore, the simple disjointness definition which is dual
to $\lambda_{i}$ no longer works. We define a notion of ordinary types
and define disjointness based upon ordinary types to overcome the
challenge in completeness. Informally, updated disjointness definition states
that two types $[[A]]$ and $[[B]]$ are disjoint if they do not share any
common ordinary subtype. This will further be discussed in detail in
\Cref{sec:inter}.

We propose a sound and complete disjointness definition. Caclulus studied
in this paper is type-safe and deterministic.

\bruno{
	Introduce our work, setting the goal to study the construct formally.
	Connect with the work on disjoint intersection types, which also
	employs a notion of disjointness, but for intersection. Explain that
	what is needed is a dual notion of disjointness.\\
	Introduce the first calculus, and explain that it is directly inspired
	by a dual notion of discjointness.\\
	Introduce the second calculus and identify a technical challenge with
	disjointness: the addition of intersection types breaks the previous
	notion of disjointness. Introduce the novel way to find disjoint types.\\
	Summarize/mention key results: type-safety; soundness/completeness of
	disjointness; determinism.\\
	Perhaps here it is also useful to identity, together with Baber, what
	were the most challenging aspects in the formalization, and maybe
	highlight these.
}

%%% Local Variables:
%%% mode: latex
%%% TeX-master: "../paper"
%%% org-ref-default-bibliography: "../paper.bib"
%%% End:
