\section{Related Work}
\label{sec:related}

\begin{comment}
\begin{itemize}
	\item {Union Types    (discussing work on union types and elimination constructs for union types) <-- Need to coordinate with Ningning/Snow so that there’s not much overlap there. If the Overview discusses some of these in detail, the RW does not need to have much text in it.}
	\item {Overloading (more general work on overloading, including type classes, approaches to overloading based on intersection types, etc…)}
	\item{Disjointness and Disjoint Intersection Types}
\end{itemize}
\end{comment}

\paragraph{Union types}
\begin{comment}
Set-theoretic unions have sound theory and extensively studied in
mathematics. Set-theoretic unions correspond to union types or
disjoint union types in programming languages. Disjoint union types
are also called sum types or variants.  Constructors are explicitly
labeled in disjoint union types and expressions are manipulated using
corresponding labels. Few other interesting calculi (and this paper)
do not use labels and provide type-based union elimination.
%Developing
%a sound and deterministic union elimination construct for type-based
%union elimination have been a challenge in research community.
\end{comment}
The union type constructor was first introduced by \citet{macqueen1984ideal}.
They proposed a typing rule that eliminates union implicitly.
The rule breaks type preservation under
the conventional reduction strategy of lambda calculus.
\citet{barbanera1995intersection} solved the problem by reducing all
copies of the same redex in parallel.
\citet{dunfield2003type} took another approach to support
mutable references. They restricted the typing rules to only allow single
occurence of the subterm of union type in the typing expression.
%Also, \citet{macqueen1984ideal} did not define dynamic
%semantics.
\snow{I think the following discussion can be more precise. I feel
  dunfield2003type is very similar to dunfield2014elaborating.}
Later, union types have been studied in the literature in various
contexts. As discussed in \Cref{subsec:elimination},
\citet{pierce1991programming} and \cite{dunfield2014elaborating} support
single-branch elimination. However, as both calculi support intersection types,
with single-branch elimination the system can also be used to support form of
overloading. \citet{pierce1991programming} does not give the dynamic semantics.
On the other hand, \cite{dunfield2014elaborating} provides
a nondeterministic dynamic semantics, which can give different
results when eliminating union types.
\begin{comment}
Moreover, while \citet{dunfield2014elaborating} shows that
subtyping is not necessary for elaboration, it is not obvious how to generalize
elaboration to support subtyping relations such as
\lstinline{Student <: Person} without using the subtyping rule. If the
elaboration were generalized further to support such a subtyping relation, then
a student with type \lstinline{Student | Person} can also be tagged
non-deterministically.
\end{comment}


% \citet{barbanera1995intersection} proposed
% two approaches for type preservation in a calculus with intersection and union
% types.
%However, their union elimination is same as last two and follow
%unrestricted union elimination.

\begin{comment}
\citet{freeman1991refinement} studied union types along with intersection types
in setting of refinement types for ML. Main focus of their work is to infer
more precise types of expressions, which they call
refinement types. Their work is targeted to contribute for the types
in ML and not for the expressions. Therefore, they did not define
expressions and dynamic semantics. On the contrary, our work provides
a complete calculus with type sound dynamic semantics.
\bruno{and ... Do they have an elimination construct for unions?
  If so how is that related to our work?}.
\baber{I added few more lines in Freeman citation.}
\end{comment}
Union types and elimination constructs based on types are
widely used in the context of XML processing languages~\cite{hosoya2003xduce,benzaken2003cduce}.
Generally speaking the elimination constructs in those
languages offer a first-match semantics,
where cases can overlap and reordering the cases may change the semantics of the program.
This is in contrast to our disjointness based approach.
\citet{hosoya2003xduce} studied union types
in the XDuce programming language. XDuce offers a novel feature of
so-called regular expression types. Pattern matching can be on
expressions and types in XDuce.  Expressions are considered as special
cases of types.  CDuce
\cite{benzaken2003cduce} is an extension of XDuce. CDuce has improved
pattern matching but it also
follows first-match semantics. Work on the more foundational aspects
of CDuce, and in particular on \emph{semantic subtyping}~\cite{frisch2002semantic}
and set-theoretic types,
also employs a form of first-match semantics elimination construct, though in a different form.
In particular, work by \citet{castagna2005gentle,castagna2017gradual}
proposes a conditional construct that can test whether a value matches a type.
If it matches then the first branch is executed and the type for the value is refined.
Otherwise, the second branch is executed and the type of the value is refined to be
the negation of the type (expressing that the value does not have such type).
\begin{comment}
\citet{fallside2001xml} studied union
types in markup language, but with a restriction of disjoint top-level
labels which is different from our work on
disjointness on types.\bruno{In other words, for the last work, the unions
are labelled (i.e. the cases are not based on types)?}

\cite{frisch2002semantic} studied union types and intersection types
with semantic subtyping and provide the theoretical basis for CDuce programming language.
However, their case expression (they call it \emph{match}) uses explicit
tags to select a particular branch. Whereas, we propose dynamic type-based
case construct.
\cite{castagna2005gentle} extended this work with type-based case construct
in the context of XDuce programming language. Main motivation of their
work is to elaborate the significance of semantics subtyping.
However, their underlying technique to select the branch in case construct
relies on negation types. We, on the other hand, use disjointness
and static type of the expression. \baber{they also use an extra variable.}
%\cite{frisch2008semantic}.
%\baber{more text.}
\citet{castagna2017gradual} studied
set-theoretic union types, intersection types and negation types
in gradual typing setting. They also propose a dynamic type-based
construct for set-theoretic union elimination which is closely
related to our work.
However, their underlying technique to select branches in case
expression is different from ours.
In particular, negation types play fundamental role in their
typing construct for the case expressions.
Whereas, our work relies on disjointness along with a notion of
static types to to select a particular branch.
They also do not provide dynamic semantics for their source language.
Whereas, we provide a type sound dynamic semantics for \cal.
\baber{no variables in their case construct?}

\baber{below text will be removed for Castagna's work.}
\bruno{Castagna did alot of work
  on union types before this one, so I'm not sure why we are mentioning only this work
  of Castagna, instead of talking about previous work where the ideas were introduced.}
\baber{Castagna's work includes CDuce, which is mentioned. I added semantic subtyping
as well now. Should we add more Castagna's work? I think his above mentioned work
is relevant to our work.}
They give a dynamic type-based cast for union elimination. But they do not
have disjointness constraint on case branches and they check the type
only for first case branch\bruno{Not a very good explanation: you get negated types for the branches}.
\end{comment}

\citet{muehlboeck2018empowering} give a general framework for subtyping
with intersection and union types. They illustrate the significance of
their framework using the Ceylon programming language.
The main objective of their work is to define a generic framework for
deriving subtyping algorithms for
intersection and union types in the presence of various distributive subtyping rules.
For instance, their framework could be useful to derive an algorithmic
formulation for the subtyping relation presented in Figure~\ref{fig:discussion:ds}.
They also briefly cover disjointness in their work. As part of their framework, they
can also check disjointness given some disjointness axioms. For instance,
for \name, such axioms could be similar to \rref{ad-mtmr} or \rref{ad-intl}
in Figure~\ref{fig:union:disj-typ}.
However, they do not have a formal
specification of disjointness. Instead they assume that some sound specification
exists and that the axioms respect such specification.
If some unsound axioms are given to their framework (say $[[Int * Int]]$) this
would lead to a problematic algorithm for checking disjointness.
In our work we provide specifications for disjointness together
with sound and complete algorithmic formulations.
In addition, unlike us,
they do not study the semantics of disjoint switch expressions,
although they mention some Ceylon examples using such expressions.

\paragraph{Occurrence Typing}
Occurence typing or flow typing \cite{tobin2008design} specializes or refines
the type of variable in a type test. For example, with occurrence typing we
could have:

\begin{lstlisting}
Int occurence(Int | String val) {
  if (val is Int) { return val+1; }
  else { return toInt(val)+2; }
}
\end{lstlisting}

\noindent In such code, \lstinline{val} initially has type $[[Int \/ String]]$.
The conditional checks if the \lstinline{val} is of type $[[Int]]$.
If the condition succeeds, it is safe to assume that \lstinline{val} is of type $[[Int]]$,
and the type of \lstinline{val} is refined in the branch to be \lstinline{Int}.
Otherwise, it is safe to assume that \lstinline{val} is of type $[[String]]$, in the
other branch (and the type is refined accordingly).
The motivation to study occurrence typing was to introduce typing in dynamically
typed languages.
Occurrence typing was further studied by \cite{tobin2010logical},
which resulted into the development of Typed Racket.
Variants of occurrence typing are nowadays employed in mainstream languages
such as TypeScript or Flow.
\cite{castagna2019revisiting} proposed a more general formulation of
occurence typing. He extended occurrence typing to refine the type of
generic expressions, not just variables. He also studied the combination
with gradual typing. Occurrence typing provides an alternative
means to eliminate union types using a first-match semantics. That is the
order of the type tests determines the priority that is given to the various
types being tested.

\paragraph{Disjoint Intersection Types}
\citet{pottinger1980type} and \citet{coppo1981functional} initially
studied intersection types in programming languages
literature. Forsythe~\cite{reynolds1988preliminary} is the first
programming language to have intersection types, but it did not
have union types.  Disjoint intersection types were first
studied by \citet{oliveira2016disjoint} in the $\lambda_{i}$ calculus
to give a coherent calculus for intersection types with a merge
operator. The notion of disjointness used in \cal, discussed in \Cref{sec:union},
is inspired by the notion of disjointness of $\lambda_{i}$. In essence,
disjointness in \cal is the dual notion: while in $\lambda_{i}$ two types
are disjoint if they only have top-like supertypes, in \cal two types
are disjoint if they only have bottom-like supertypes.

None of calculi with disjoint intersection types~\cite{oliveira2016disjoint,bi_et_al:LIPIcs:2018:9227,alpuimdisjoint} in the literature
includes union types. One interesting discovery of our work is that the
presence of both intersections and unions in a calculus can affect disjointness.
In particular, as we have seen in Section~\ref{sec:inter}, adding intersection types
required us to change disjointness. The notion of disjointness that was
derived from $\lambda_{i}$ stops working in the presence of intersection types.
Interestingly, a similar issue happens when union types are added to
a calculus with disjoint intersection types. If disjointness of two types \lstinline{A}
and \lstinline{B} is defined to be that such types can only have top-like types,
then adding union types immediately breaks such definition.
For example, the types $[[Int]]$ and $[[Bool]]$ are disjoint but, with union
types, $[[Int \/ Bool]]$ is a common supertype that is not top-like.
We conjecture that, to add union types to disjoint intersection types,
we can use the following definition of disjointness:

\begin{definition}
\label{def:related:disj}
  A $*_s$ B $\Coloneqq$ $\forall$ C, \ $[[ordinary C]]$ \ $\rightarrow$ \ $\neg$ \ ($[[A <: C]]$ and $[[B <: C]]$).
\end{definition}

\noindent which is, in essence, the dual notion of the definition presented in
Section~\ref{sec:inter}. Under this definition $[[Int]]$ and $[[Bool]]$ would
be disjoint since we cannot find an common ordinary supertype (and $[[Int \/ Bool]]$
is a supertype, but it is not ordinary). Furthermore, there should be a
dual notion to LOS, capturing the greatest ordinary supertypes. Moreover,
if a calculus includes both disjoint switches and a merge operator,
then the two notions of disjointness must coexist in the calculus.
This will be an interesting path of exploration for future work.

\paragraph{Overloading}
Union types also provide a form of function overloading or ad-hoc
polymorphism using the switch and type-based case analysis. A programmer
may define the argument type to a union type. By
using type-based case analysis, it is possible to execute different code
for each specific type of input.  Intersection types have also been
studied for function overloading. For example a function with type
$[[Int -> Int /\ Bool -> Bool]]$ can take input values either of type
$[[Int]]$ or $[[Bool]]$.  In such case, we return either $[[Int]]$ or $[[Bool]]$
depending upon the input type.  Function overloading has been studied
in detail in the literature by \cite{castagna1995calculus},
\cite{cardelli1985understanding}, \cite{stuckey2005theory} among
others.  \citet{wadler1989make} studied type classes as an alternative way
to provide overloading based on parametric polymorphism.

%%% Local Variables:
%%% mode: latex
%%% TeX-master: "../paper"
%%% org-ref-default-bibliography: "../paper.bib"
%%% End:
