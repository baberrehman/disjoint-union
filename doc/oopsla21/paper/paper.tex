\documentclass[acmsmall,review,anonymous]{acmart}\settopmatter{printfolios=true,printccs=false,printacmref=false}

\setcopyright{acmcopyright}
\copyrightyear{2018}
\acmYear{2018}
\acmDOI{10.1145/1122445.1122456}

%% These commands are for a PROCEEDINGS abstract or paper.
\acmConference[Woodstock '18]{Woodstock '18: ACM Symposium on Neural
  Gaze Detection}{June 03--05, 2018}{Woodstock, NY}
\acmBooktitle{Woodstock '18: ACM Symposium on Neural Gaze Detection,
  June 03--05, 2018, Woodstock, NY}
\acmPrice{15.00}
\acmISBN{978-1-4503-XXXX-X/18/06}

%%%%%%%%%%%%%%%%%%%%%%%%%%%%%%%%%%%%%%%%%%%%%%%%%%%%%%%%%%%%%%%%%%%%%%%%%%%%%%%%
%\usepackage{geometry}
%\geometry{left=2.5cm,right=2.5cm,top=2.5cm,bottom=2.5cm}

% Basics
%\usepackage{fixltx2e}
%\usepackage{url}
%\usepackage{fancyvrb}
%\usepackage{mdwlist}  % Miscellaneous list-related commands
\usepackage{xspace}   % Smart spacing
%\usepackage{supertabular}


% https://www.nesono.com/?q=book/export/html/347
% Package for inserting TODO statements in nice colorful boxes - so that you
% won’t forget to fix/remove them. To add a todo statement, use something like
% \todo{Find better wording here}.
%\usepackage{todonotes}

%for backtick `
\usepackage{textcomp}

%% Math
%\usepackage{amsmath}
\usepackage{amsthm}
%\usepackage{amssymb}
%\usepackage{bm}       % Bold symbols in maths mode

%% Theoretical computer science
\usepackage{stmaryrd}
\usepackage{mathtools}  % For "::=" ( \Coloneqq )

%% Code listings
\usepackage{listings}
\usepackage{xcolor}

%for cref -- celever referencing
\usepackage{cleveref}

\usepackage{amsbsy}

\definecolor{codegreen}{rgb}{0,0.6,0}
\definecolor{codegray}{rgb}{0.5,0.5,0.5}
\definecolor{codepurple}{rgb}{0.58,0,0.82}
\definecolor{backcolour}{rgb}{0.95,0.95,0.92}

\lstdefinestyle{mystyle}{
    %backgroundcolor=\color{backcolour},
    commentstyle=\color{codegreen},
    keywordstyle=\color{magenta},
    numberstyle=\tiny\color{codegray},
    stringstyle=\color{codepurple},
    basicstyle=\ttfamily\footnotesize,
    breakatwhitespace=false,
    breaklines=true,
    captionpos=b,
    keepspaces=true,
    numbers=left,
    numbersep=5pt,
    showspaces=false,
    showstringspaces=false,
    showtabs=false,
    tabsize=2
}
\lstset{style=mystyle}

%% Font
\usepackage[euler-digits,euler-hat-accent]{eulervm}


\usepackage{ottalt}

\renewcommand\ottaltinferrule[4]{
  \inferrule*[right={#1}]
    {#3}
    {#4}
}

\newcommand\mynote[3]{\textcolor{#2}{#1: #3}}
\newcommand\bruno[1]{\mynote{Bruno}{red}{#1}}
\newcommand\baber[1]{\mynote{Baber}{blue}{#1}}
\newcommand\snow[1]{\mynote{Snow}{orange!80!black}{#1}}
\newcommand\ningning[1]{\mynote{NN}{purple}{#1}}

\newcommand{\name}{$\lambda_{u}$\xspace}
%\newcommand{\dut}{\textsf{Disjoint Union Types}\xspace}
\newcommand{\cal}{$\lambda_{u}$\xspace}
\newcommand{\typeof}{$typeof$\xspace}
\newcommand{\Typeof}{$Typeof$\xspace}

% Ott includes
\inputott{ott-rules}

%% Bibliography style
\bibliographystyle{plainurl}% the mandatory bibstyle

\title{Union Types with Disjoint Switches}

%%%%%%%%%%%%%%%%%%%%%%%%%%%%%%%%%%%%%%%%%%%%%%%%%%%%%%%%%%%%%%%%%%%%%%%%%%%%%%%%
%% \author{Xuejing Huang}
%% \orcid{0000-0002-8496-491X}
%% \affiliation{
%%   \institution{The University of Hong Kong}
%% }
%% \email{xjhuang@cs.hku.hk}

\author{Bruno C. d. S. Oliveira}
%% \authornote{with author1 note}          %% \authornote is optional;
                                        %% can be repeated if necessary
%% \orcid{}             %% \orcid is optional
\affiliation{
  %% \position{Position1}
  %% \department{Department1}              %% \department is recommended
  \institution{The University of Hong Kong}            %% \institution is required
  %% \streetaddress{Street1 Address1}
  %% \city{City1}
  %% \state{State1}
  %% \postcode{Post-Code1}
  %% \country{Country1}                    %% \country is recommended
}
\email{bruno@cs.hku.hk}          %% \email is recommended
%%%%%%%%%%%%%%%%%%%%%%%%%%%%%%%%%%%%%%%%%%%%%%%%%%%%%%%%%%%%%%%%%%%%%%%%%%%%%%%%

\begin{document}

\begin{abstract}
Union types allow expressing terms with alternative types. Several forms of
union types have been investigated in the research literature, and
union types are nowadays a common feature in many modern programming
languages.
%The Ceylon programming language is interesting in that it
%supports union types, with an innovative elimination construct that has
%not been previously studied in the rese
This paper investigates a particular formulation of union types,
inpired by the Ceylon programming language, with an elimination
construct that enables case analysis (or switches) on types.  The
interesting aspect of such construct is that each clause must operate
on \emph{disjoint} types. By using such elimination construct, it is
possible to ensure that all possible cases are
handled and that none of the cases overlap. In turn, this means that
the order of the cases does not matter and reordering the cases has no
impact on the semantics, which can aid program undertanding and
refactoring. While implemented in the Ceylon language, such construct
with disjoint switches has not been formally studied in the research
literature, although a related notion of disjointness has been studied
in the context of \emph{disjoint intersection types}.

We study union types with disjoint switches formally and in a language
independent way.  We present the \emph{union calculus} (\cal) that
includes disjoint switches and prove several results, including type
soundness and determinism. The notion of disjointness in \cal is
interesting in that it is in essence the dual notion of disjointness
for intersection types.  However, there are challenges that arize for
disjointness when combining union and intersection types.  We also
study several extensions, including intersection types and
distributive subtyping and show that such extensions retain all the
desired properties. All the results about \cal and its extensions have
been formalized in the Coq theorem prover.
\end{abstract}

\begin{comment}
    With advance types
such as intersection types and union types, it has become a challenge
to define a robust, type-safe, coherent and deterministic type
system. One often has to compromise on one property to attain
another. Union types with pattern matching on types add significant
expressive power in programming language. Function overloading can
simply be expressed in a single function with the help of union types
and case analysis on types. Intersection types incorporate many
interesting and advance features that are not easy to implement in
classical OOP model.  This study proposes a novel calculus with all
aforementioned properties for pattern matching with union types and
intersection types. The calculus proposed in this study is named as
\cal.  Outline idea in \cal is to allow only non-overlapping or
disjoint types in case expressions.
\end{comment}

\maketitle

%\ottdefnsOrdinary

%\ottdefnsBottomLike

%\ottdefnsDisjointness

%\ottdefnsSubtyping

%\ottdefnsTyping

%\ottdefnsReduction

\section{Introduction}
\label{sec:intro}

Most programming languages support some mechanism to express terms
with alternative types. Algol 68~\cite{van1969report,van2012revised} included a form of
\emph{tagged} unions for this purpose. With tagged unions
an explicit tag distinguishes between different cases in the
union type.
Such an approach has been adopted by functional languages, like Haskell, ML, or
OCaml, which allow tagged unions (or sum types~\cite{pierce2002types}), typically via
either \emph{algebraic datatypes}~\cite{hope} or \emph{variant types}~\cite{garrigue98}.
Languages like C or C++ support \emph{untagged} union types where
values of the alternative types are simply stored at the same memory
location. However, there is no checking of types when accessing values of
such untagged types. It is up to the programmer to ensure that the proper
values are accessed correctly in different contexts. Otherwise the
program may produce errors by accessing the value at the incorrect type.

Modern OOP languages, such as Scala~\cite{odersky2004overview}, Flow~\cite{chaudhuri2015flow},
TypeScript~\cite{bierman2014understanding} or Ceylon~\cite{king2013ceylon} support a form
of untagged union types.
In such languages a union type $[[A \/ B]]$ denotes expressions which can have type
$[[A]]$ or type $[[B]]$. Union types can be useful in many situations.
For instance, union types provide an alternative to some forms
of overloading. The idea is that a function that takes an argument
with a union type acts similarly to an overloaded function.
Furthermore, union types have other uses, such as modelling error handling.
If a function returning a number may fail due to some
error, the union type $[[String \/ Int]]$ can be returned: an integer
is returned upon successful computation of the result; and
a string (an error message) is returned if an error happens.

To safely access values with union types, some form of
\emph{elimination construct} is needed. Many programming languages often
employ a language construct that checks
the types of the values at runtime for this purpose.
Several elimination constructs for (untagged) union types
have also been studied in the research literature~
\cite{benzaken2003cduce,dunfield2014elaborating,castagna:settheoretic}.
Typically, such constructs take the form of a type-based case analysis
expression.

A complication is that the presence of subtyping introduces the
possibility of \emph{overlapping types}. For instance we may have a
\lstinline{Student} and a \lstinline{Person}, where every student is a person (but not
vice-versa). If we try to eliminate a union using such types we can
run into situations where the type in one branch can cover a type in a
different branch (for instance \lstinline{Person} can also cover
\lstinline{Student}). More generally, types can \emph{partially overlap}
and for some values two branches with such types can apply, whereas
for some other values only one branch applies.
Therefore the design of such elimination constructs has to
consider what to do in situations where overlapping types arise.  A
first possibility is to have a \emph{non-deterministic semantics},
where any of the branches that matches can be taken. However, in
practice determinism is a desirable property,
so this option is not practical. A second possibility, which is
commonly used for overloading, is to employ a \emph{best-match
  semantics}, where we attempt to find the case with the type that
best matches the value. Yet another option is to employ a
\emph{first-match semantics}, which employs the order of the branches
in the case. Various existing elimination constructs for unions
~\cite{benzaken2003cduce,castagna:settheoretic}
employ a first-match approach. All of these three options have been explored
and studied in the literature. 

The Ceylon language~\cite{king2013ceylon} is a JVM-based language that aims to provide an
alternative to Java. The type system is interesting
in that it departs from existing language designs, in particular
with respect to union types and method overloading.
The Ceylon designers had a few different
reasons for this. They wanted to have a fairly rich type system
supporting, among others: \emph{subtyping}; \emph{generics with bounded
quantification}; \emph{union and intersection types}; and \emph{type-inference}.
The aim was to support most features that are also already available
in Java, as well as a few new ones. However they wanted to do this in
a principled way, where all the features interacted nicely.  A
stumbling block towards this goal was Java-style method
overloading\footnote{\url{https://github.com/ceylon/ceylon-spec/issues/73}}.
The interaction of overloading with other
features was found to be quite challenging. Additionally, the presence of
overloaded methods with overlapping types
makes reasoning about the code hard 
for both tools and humans. Algorithms for finding the best match for an
overloaded method in the presence of rich type system features (such as
those in Ceylon) are challenging, and not necessarally well-studied in the
existing literature. Moreover allowing overlapping methods can make
the code harder to reason for humans: without a clear knowledge of how
overloading resolution works, programmers may incorrectly assume that
a different overloaded method is invoked. Or worse, overloading can
happen silently, by simply reusing the same name for a new
method. These problems can lead to subtle bugs.
For these reasons, the Ceylon designers decided not to support
Java-style method overloading.

To counter the absence of overloading, the Ceylon designers turned to
union types instead. However, they did so in a way that differs from
existing approaches. Ceylon includes a type-based 
\emph{switch construct} where all the cases must be \emph{disjoint}.  If
two types are found to be overlapping, then the program is statically
rejected. Many common cases of method overloading, which are clearly
not ambiguous, can be modelled using union types and disjoint switches.
By using an approach based on disjointness, some use cases for
overloading that involve Java-style overloading with
overlapping types are forbidden. However,
programmers can still resort to creating non-overloaded methods in
such a case, which arguably results in code easier to reason about.
Disjointness ensures that it is always
clear which implementation is selected for an ``overloaded'' method,
and only in such cases overloading is allowed\footnote{Ceylon does
  allow dynamic type tests, similar to \lstinline{instanceof} in Java, which can be used
  in combination with the switch construct to simulate some overlapping.}.
In the switch construct,
the order of the cases does not matter and reordering the cases has no
impact on the semantics, which can also aid program undertanding and
refactoring.
Additionally, union types have other applications besides overloading,
so we can get other interesting functionality as well. Finally, from
the language design point of view, it would be strange to support two
mechanisms (method overloading and union types), which greatly overlap
in terms of functionality.

While implemented in the Ceylon language,
disjoint switches have not been well studied formally.
The work by \citet{muehlboeck2018empowering} is the only work that we are aware of,
where the Ceylon's switch construct
and disjointness are briefly mentioned. However, the focus 
of that work is on algorithmic formulations of subtyping
with unions types, intersection types and various distributivity
rules for subtyping. No semantics of the switch construct is given.
Disjointness is informally defined in various sentences in the
Ceylon documentation. It involves a set of 14 rules described in English\footnote{\url{http://web.mit.edu/ceylon_v1.3.3/ceylon-1.3.3/doc/en/spec/html_single}}. Some of the rules are relatively generic, while
others are quite language specific. 
Interestingly, a notion
of disjointness has been already studied in the literature
for intersection types~\cite{oliveira2016disjoint}. That line of work studies calculi
with intersection types and a \textit{merge operator}~\cite{reynolds1988preliminary}. Disjointness
is used to prevent ambiguity in merges, which can create
values with types such as $[[Int /\ Bool]]$. Only values
with disjoint types can be used in a merge.

In this paper, we study union types with disjoint switches formally
and in a language independent way. We present the \emph{union
  calculus} (\cal), which includes disjoint switches and union types.
The notion of disjointness in \cal is interesting in that it is
the dual notion of disjointness for intersection types.
We prove several results, including \emph{type soundness}, \emph{determinism}
and the \emph{soundness} and \emph{completeness} of algorithmic formulations
of disjointness.
We also study several extensions of \cal. In particular,
the first extension (discussed in Section~\ref{sec:inter}) adds intersection
types to \cal. It turns out such extension is non-trivial, as it reveals
a challenge that arises for disjointness when combining
union and intersection types:
the notion of dual notion disjointness borrowed from
disjoint intersection types no longer works, and we must employ
a novel, more general, notion instead. Such change also has an impact
on the algorithmic formulation of disjointness, which must change as
well. We also study several other extensions, including intersection
types and distributive subtyping. We prove that all the extensions retain
the original properties of \cal. Moreover, all the results about \cal and its
extensions have been formalized in the Coq theorem prover.

In summary, the contributions of this paper are:

\begin{itemize}
\item {\bf The \name calculus:} We present a simple calculus with union
  types, and a disjoint switch construct. The calculus is type sound and
  deterministic. 
\item {\bf Specifications and algorithmic formulations of disjointness:}
  We present two formulations of disjointness, which are general and
  language independent. The first formulation is directly derived from
  the the existing notion of disjointness for disjoint intersection types,
  but it only works for a calculus with union types. The second formulation
  is novel and more general, and can be used in a calculus that includes
  intersection types as well.
\item {\bf Extensions:} We study several extensions of \name, including the
  addition of intersection types, distributive subtyping and
  subtyping rule for a class of types that represent empty types.
\item {\bf Mechanical formalization:}
  All the results about \cal and its
  extensions have been formalized in the Coq theorem prover.
  \footnote{{\bf Note to reviewers:} The Coq formalizations can be found in the
  supplementary material for this paper.}
\end{itemize}



%\begin{align*}
%&Isomorphic & A \sim B & ::= [[A <: B]] \wedge [[B <: A]]
%\end{align*}


%\begin{align*}
%&BottomLikeSpec & C & ::= (\forall A ~ B, ~ [[A /\ B]] \sim C \rightarrow \neg ( [[ A <: B ]] ) \wedge \neg ( [[ B <: A ]] )) \vee ([[C <: Bot]])
%\end{align*}

%\begin{align*}
%&DisjSpec & A * B & ::= \forall C, [[C <: A]] \wedge [[C <: B]] \rightarrow  \rfloor [[C]] \lfloor
%\end{align*}

%%% Local Variables:
%%% mode: latex
%%% TeX-master: "../paper"
%%% org-ref-default-bibliography: "../paper.bib"
%%% End:

\section{Overview}
\label{sec:overview}

This section provides some background on union types, and some common approaches
to eliminate union types. Then it describes the Ceylon approach to Union types.
Finally, it presents the key ideas and challenges in our work and \name.

\ningning{We may want to be consistent with the syntax used in examples. Now
  there are three things going on: examples before introducing Ceylon; examples
  in Ceylon; examples in our calculi. Will it look confusing as we use three
  different syntax rules?}
\bruno{Three different syntaxes is definitly too much. I think we may be able
  to get by with two syntaxes: our calculus and Ceylon. Alternatevely, we can
  use just the syntax in our calculus here, and later, perhaps in the discussion
  have examples in Ceylon, when comparing to Ceylon. }

\subsection{Tagged Union Types}

% Brief intro to union types and one or 2 simple examples that show that we
% can automatically lift values into union type without the need of some
% tag/constructor. Something like:

We start with a brief introduction to union types. An expression has a union
type $[[A\/B]]$, if it can be considered to have either type $[[A]]$ \textit{or}
type $[[B]]$. Many systems model \textit{tagged union types} (also called
\textit{sum types} or \textit{variants types}), where explicit \textit{tags}
are used to construct terms with union types. Languages with algebraic datatypes~\cite{hope}
or (polymorphic) variants~\cite{garrigue98} support tagged union types.
In their basic form, there are two introduction forms:
$\mathsf{inj_1} :: [[A -> A \/ B]]$ turn the type of an expression of type
$[[A]]$ into type $[[A \/ B]]$; and $\mathsf{inj_2} :: [[B -> A \/ B]]$
turns the type of an expressions of type $[[B]]$ into type $[[A \/ B]]$.
For example, we can have:

\begin{lstlisting}
inj1 "foo": String | Int
inj2 1 : String | Int
\end{lstlisting}

\noindent Using tagged union types, we can implement a safe integer
division function, as:

\begin{lstlisting}
function safediv (x : Int) (y : Int) : String | Int =
  if (y == 0) then inj1 "Divided by zero"
  else inj2 (x / y)
\end{lstlisting}

\noindent Here the intention is to have a safe (integer) division operation that detects
division by zero errors, and requires clients of this function to handle
such errors. The return type \lstinline{String | Int} denotes that the function
can either return an error message (a string), or an integer, when division
is performed without errors.

\paragraph{Eliminating tagged union types.}
Tagged union types are eliminated by some form of case analysis.
For consistency with the rest of the paper, we use a syntactic form with
\lstinline{switch} expressions for such case analysis. For example,
the following program \lstinline{tostring} has different behaviors depending on the
tag of \lstinline{x}, where \lstinline{show} takes an \lstinline{Int} and
returns back its string representation.

\begin{lstlisting}
function tostring (x: String | Int) : String =
  switch (x)
    inj1 str -> str
    inj2 num -> show num
\end{lstlisting}

Combining union type construction in \lstinline{safediv} and its elimination in
\lstinline{tostring}, we can easily implement an interface which returns the
result of safe division as one \lstinline{String}.

\begin{lstlisting}
> tostring (safediv 42 2)
"21"
> tostring (safe 42 0)
"Divided by zero"
\end{lstlisting}


\subsection{Type-directed Elimination of Union Types}\label{subsec:elimination}

% Motivate the need for having a construct that can eliminate union types,
% perhaps trying to use an example where the String above would be a kind
% of exception, and the Int would be regular computation. Alternatively
% find an existing example from the literature.

% Discussing existing approaches for eliminating union types;
% point out that elimination constructs based on types (our focus) vs
% elimination constructs based on tags (which are used in algebraic
% datatypes or polymorphic variants (like in OCaml)).
% The focus should be on type-based approaches.

% Try to identify some limitations/problems. For instance how to deal
% with ambiguity (just use order? restrict the construct somehow ...).


While tags are useful to make it explicit which type a value belongs to, they
also add clutter in the programs, and usually require extra space to store the
tags. On the other hand, in systems with subtyping for union types
\cite{dunfield2014elaborating,pierce1991programming,muehlboeck2018empowering},
explicit tags are replaced by implicit coercions represented by the two
subtyping rules $[[A <: A \/ B]]$ and $[[B <: A \/ B]]$. We call such types
\textit{untagged union types}, or simply \textit{union types}. In those systems,
a term of type $[[A]]$ or $[[B]]$ can be directly used as if it had type $[[A \/
B]]$, and thus we can model safe division as

\begin{lstlisting}
function safediv2 (x : Int) (y : Int) : String | Int =
  if (y == 0) then "Divided by zero"
  else (x / y)
\end{lstlisting}

\noindent However, now elimination of union types cannot rely on tags anymore, and
different systems implement elimination differently. We review some of the
existing approaches next.

\paragraph{Single branch elimination.}

A possible approach is to use an elimination of an expression with a union type
$[[A \/ B]]$ supports only one branch, and the branch needs to have the same
type when the expression has type $[[A]]$, or type $[[B]]$. This approach is
adopted, for example by \citet{pierce1991programming},
\citet{barbanera1995intersection}, and more recently,
\citet{dunfield2014elaborating}. For better illustration, we adapt their syntax
using our switch notation in the examples, while preserving their semantics. For
example, in \lstinline{tostring2}, the expression \lstinline{show x} must return
\lstinline{String} when \lstinline{y : String}, and when \lstinline{y : Int},
which means that \lstinline{show} must be overloaded (in
\citet{pierce1991programming,dunfield2014elaborating,barbanera1995intersection},
this can be implemented by requiring \lstinline{show} to have an
\textit{intersection type}).

\begin{lstlisting}
function tostring2 (x: String | Int) : String =
  switch (x)
    y -> show y
\end{lstlisting}

This implementation is concise, but it is also restrictive as it can no longer
support multiple branches according to the different representations of
\lstinline{x}. Furthermore it relies on the language also supporting overloaded
functions. Without overloaded functions the construct would not be very useful.
%\bruno{Perhaps, somewhere in this section we need to comment on the non-deterministic semantics.
%  Maybe we can do that here and point out, for instance, that Dunfield's approach has a
%  non-deterministic semantics (since it actually allows, for instance two overloaded
%  implementations of show with integer arguments)}

\paragraph{Type-directed elimination.}

On the other hand, some
systems~\cite{castagna:settheoretic} support
\textit{type-directed} elimination of union types. For instance,
\lstinline{tostring3} has different behaviors depending on the \textit{type} of
\lstinline{x}.

\begin{lstlisting}
function tostring3 (x: String | Int) : String =
  switch (x)
    (y : String) -> y
    (y : Int) -> show y
\end{lstlisting}

\snow{To be consistent with our syntax, each branch should name a variable, like
\lstinline{tostring2}. Same applies to \lstinline{isstudent2}.}
\ningning{I added variables. Then there is no type refinement anymore.. But I
  guess it's fine as the reference to occurrence typing is not important anyway.}

\noindent Note here \lstinline{show} is directly applied to \lstinline{x}, and
it does not need to be overloaded, as the type-directed elimination
\textit{turns} the variable \lstinline{x} of type \lstinline{String | Int} into
the variable \lstinline{y} of type \lstinline{Int}.

However, compared to tag-directed elimination, extra care must to taken with
type-directed elimination. In particular, while we can easily distinguish tags,
in type-directed elimination, ambiguity may arise when types in a union type
overlap. For example, consider the type \lstinline{Person | Student}, where we
assume \lstinline{Student} is a subtype of \lstinline{Person}. With tag-directed
elimination, we can write the following function:

\begin{lstlisting}
function isstudent (x: Person | Student) : Bool =
  switch (x)
    inj1 person -> False
    inj2 student -> True
\end{lstlisting}

But if we transform this function straightforwardly to type-directed
elimination, we will get:

\begin{lstlisting}
function isstudent2 (x: Person | Student) : Bool =
  switch (x)
    (y : Person)  -> False
    (y : Student) -> True
\end{lstlisting}

\noindent Now it is unclear what would happen if we apply \lstinline{isstudent2}
to a term of type \lstinline{Student}, as its type matches both branches. In
some calculi~\citep{dunfield2014elaborating}, the choice is not determined in
the semantics, in the sense that either branch can be chosen. This leads to a
non-deterministic semantics. In some other languages or
calculi~\citep{castagna:settheoretic}, branches are inspected from top to
bottom, and the first one that matches the type gets chosen. However, in those
systems, as \lstinline{Person} is a supertype of \lstinline{Student}, the first
branch subsumes the second one and will always get chosen, and so the second
branch will never get evaluated! This could be unintentional, and similar
programs being accepted could lead to subtle bugs. Even if a warning could be
given to alert programmers that a case can never be executed, there could be
other situations where two cases overlap, but neither case subsumes the other.
For instance we could have \lstinline{Student} and \lstinline{Worker} as
subtypes of \lstinline{Person}. For a person that is both a student and a
worker, a switch statement that discriminates between workers and students could
potentially choose either branch. However for persons that are only students or
only workers, only one branch can be chosen.

\paragraph{Best-match and overloading.}

Some languages support form of typed-based union elimination via method
overloading. Such form is used in, for example, Java \cite{javadoc}. In Java, we
can encode \lstinline{isstudent3} as a function, which has different
behaviors when the type of the argument differs.

\begin{lstlisting}
boolean isstudent3 (Person x) {
  return False;
}

boolean isstudent3 (Student x) {
  return True;
}
\end{lstlisting}

In Java, overloading is resolved by finding in all method implementations the
one with the \textit{best} type signature that describes the argument. That
means, if we apply \lstinline{isstudent3} to a term of type \lstinline{Student},
the second implementation is chosen, as \lstinline{Student} is the best type
describing the argument. As we can see, such best-match strategy gets rid of the
order-sensitive problem. In this case, we can have the two implementations of
\lstinline{isstudent3} in any order, and Java will always try to find the
best-match one.

However, the best-match strategy can also be confusing, especially when the
system features implicit upcasting (e.g., by subtyping). For example, suppose
the type \lstinline{Pegasus} is a subtype of both type \lstinline{Bird} and type
\lstinline{Horse}. If a method \lstinline{isbird} is overloaded for
\lstinline{Bird} and \lstinline{Horse}, then which method implementation should
we choose when we apply \lstinline{isbird} to a term of type
\lstinline{Pegasus}, the one for \lstinline{Bird}, or the one for
\lstinline{Horse}? In such case, the semantics is non-deterministic again, and
the behavior of the program depends on the particular implementation of the
compiler.


\subsection{Eliminating Union Types in Ceylon}

The Ceylon language~\cite{} supports type-directed union elimination by a
switch expression with multiple branches. The following program is an example
with union types using Ceylon's syntax:

\begin{lstlisting}
	void print(String|Integer|Float x) {
		switch (x)
		case (is String) { print("String: ``x``"); }
		case (is Integer|Float) { print("Number: ``x``"); }
	}
\end{lstlisting}
%

For the switch expression, Ceylon enforces static type checking with two
guarantees: \textit{exhaustiveness}, and \textit{disjointness}. First, Ceylon
ensures that all cases in a switch expression are exhaustive. In the above
example, \lstinline{x} can be either a string, an integer or a floating point
number. The types used in the cases do not have to exactly match with the types
of \lstinline{x}. Nevertheless, the combination of all cases must be able to
handle all possibilities. That means, if the last case only dealt with
\lstinline{Integer} (instead of \lstinline{Integer|Float}), then the program
would be statically rejected, since no case would deal with a floating point
number.

Second, Ceylon enforces that all cases in a switch expression are
\textit{disjoint}. That is, unlike the approaches described in
Section~\ref{subsec:elimination}, in Ceylon, it is impossible to have two
branches that match with the input at the same time. For instance, if the first
case used the type \lstinline{String | Float} instead of \lstinline{String}, the
program would be rejected statically with a type error. Indeed, if the program
were to be accepted, then the call \lstinline{print(3.0)} would be ambiguous,
since two branches could deal with the floating point number. Note that, since
the cases in a switch cannot overlap, their order is irrelevant to the program's
behavior and its evaluation result.

\paragraph{Union types as an alternative to overloading}
A motivation of such type-directed union elimination in Ceylon is to
model a form of function overloading.
The following example, which is adapted from TypeScript's documentation\footnote{https://www.typescriptlang.org/docs/handbook/unions-and-intersections.html},
demonstrates how to define an ``overloaded'' function \lstinline{padLeft},
which adds some padding to a string. The idea is that there can be two versions
of \lstinline{padLeft}: one where the second argument is a string; and
the other where the second argument is an integer:

\begin{lstlisting}
String space(Integer n){
	if (n==0) {
		return "";
	} else {
		return " "+space(n-1);
	}
}

String padLeft(String v, String|Integer x){
	switch (x)
	case (is String) { return x+v; }
	case (is Integer) { return space(x)+v; }
}

print( padLeft("?", 5) ); // "     ?"
print( padLeft("World", "Hello ") ); // "Hello World"
\end{lstlisting}
%

In the two cases of the switch construct, there are two different implementations
of the \lstinline{padLeft} function: one that appends a string to the left of \lstinline{v},
and the other that calls function \lstinline{space} to generate a string with \lstinline{x} spaces,
and then append that to \lstinline{v}.
Although statically \lstinline{x} has type \lstinline{String|Integer}, as a concrete value
it can only be a string or an integer.
As such, when values with such types are passed to the function,
the corresponding branch is chosen and executed.

\paragraph{Other applications of union types}
Besides being used for overloading, union types can be used for other purposes to.
For instance, we can easily encode the \lstinline{safediv} function in Section~\ref{subsec:elimination}
in Ceylon:
\bruno{Ceylon code for safediv and to string here, and some short piece of text about the
example}
%
\begin{lstlisting}
String | Integer safediv3 (Integer x, Integer y){
	if (y==0) {
		return "Divided by zero";
	} else{
		return (x/y);
	}
}
\end{lstlisting}
%
The return value can be a string or an integer.
No tag is needed for any of them to be a \lstinline{String|Integer},
since union types can be implicitly introduced.
For the declared return type of the function, as long as it is
a supertype of all possible return values, it is valid
in Ceylon.


Furthermore, another interesting application
of union types in Ceylon is to encode nullable types (or optional types)
in a type-safe way.
A similar approach to nullable types has also been recently proposed for Scala~\citep{nieto20nulls}.
The \lstinline{null} value is inhabited by the type \lstinline{Null}.
For type \lstinline{A} to be nullable, one can write \lstinline{A?},
which stands for \lstinline{A|Null}.
If we eliminate a value of type \lstinline{A|Null}, there has to be a branch
to handle the null value.
Ceylon define the top type \lstinline{Anything} as an enumerated class.
\lstinline{Null} is a case of it.
It differs to the bottom type \lstinline{Nothing} in the sense
that it is inhabited.
\lstinline{Anything} has another distinct case called \lstinline{Object}.
It is the root of primitive types, function types, all interfaces and any user-defined class.
\lstinline{Null} is disjoint to \lstinline{Object}, and therefore, to all
these types.
The following diagram presents their subtyping relation briefly.

\tikzset{
  state/.style={
    rectangle,
    rounded corners,
    minimum height=2em,
    inner sep=2pt,
    text centered,
    align=center
  },
}
\begin{center}
\begin{tikzpicture}[->]
  \footnotesize
  \node[state, anchor=center] (a) at (0, 0) {Null};
  \node[state, anchor=center] (b) at (2, 0) {Object};
  \node[state, anchor=center] (c) at (1, 1) {Anything};
  \node[state, anchor=center] (d) at (1, -1) {Char};
  \node[state, anchor=center] (e) at (3, -1) {Integer};
  \node[state, anchor=center] (f) at (2, -2) {Nothing};
  \draw (c) -- (a);
  \draw (c) -- (b);
  \draw (b) -- (d);
  \draw (b) -- (e);
  \draw (e) -- (f);
  \draw (d) -- (f);
  \draw (a) to[out=225,in=180] (f);
\end{tikzpicture}
\end{center}

\bruno{Say more about this, maybe show a (simplified)
  picture of the subtyping lattice?
The point is that Null is not a subtype of Object, right? so, by default object
types have no null value. }
%Similarly, one can use enumerated types to denote various special cases, or easily
%add another type to the argument type of an existing function when considering
%more possible inputs, to improve the program's robustness.
%Next, we will see how Ceylon's static type checking helps programmers.

\begin{comment}
\paragraph{Exhaustiveness}

Ceylon checks the exhaustiveness of a switch by comparing the union of all
cases and the switched term's type.
For the switch to be accepted, the former must be a supertype of the later.
%Recall that adding more components to a union type make it a supertype of the
%initial type, like \lstinline{Integer|Float} to \lstinline{Integer}.
That is to say a switch construct must have enough cases to handle all
possible runtime types of the term.
%
For example, here we define \lstinline{Node} by enumerating its subtypes.
It can be viewed as the union of \lstinline{Leaf} and \lstinline{Branch}.
Then in the function \lstinline{printTree} that takes a \lstinline{Node},
both cases are taken into consideration.
%
\begin{comment}
------------- THE OTHER EXAMPLE -----------------
Adding more subtype in it causes error in exhaustiveness checking in switch.
\begin{lstlisting}
interface Resource of File | Directory | Link { }
interface File satisfies Resource {}
interface Directory satisfies Resource {}
interface Link satisfies Resource {}

void printType(Resource resource){
	switch (resource)
	case (is File) { print("File"); }
	case (is Directory) { print("Directory"); }
	case (is Link) { print("Link"); }
}
\end{lstlisting}

%
\begin{lstlisting}
abstract class Node() of Leaf | Branch {}

class Leaf(shared Object element)
extends Node() {}

class Branch(shared Node left, shared Node right)
extends Node() {}

void printTree(Node node) {
	switch (node)
	case (is Leaf) {
		print("Found a leaf: ``node.element``!");
	}
	case (is Branch) {
		printTree(node.left);
		printTree(node.right);
	}
}

printTree(Branch(Branch(Leaf("aap"), Leaf("noot")), Leaf("mies")));
\end{lstlisting}
% https://ceylon-lang.org/documentation/1.3/tour/types/
%
We can allow the input to be \lstinline{null} by replacing the argument type \lstinline{Node} by \lstinline{Node?}.
Ceylon uses union types to encode nullable types (or the optional types) in a
type-safe way.
The null value is inhabited in \lstinline{Null}, and \lstinline{A?} stands for \lstinline{A|Null}.
If the switched term has a nullable type, the exhaustive checking makes sure there
is a branch to handle it.
%
Similarly, one can use enumerated types to denote various special cases, or easily
add another type to the argument type of an existing function when considering
more possible inputs, to improve the program's robustness.
%
If we can add more subtype in the declaration of \lstinline{Node} without
adapting the function definition, an compiling error will be raised:
case types must cover all cases of the switch type.
Such checking reminds programmers to keep consistent when changing the
related code and avoid potential runtime errors.

\bruno{The following example is better, and I guess useful to illustrate exhaustiveness.}
\begin{lstlisting}
	void printAfterPlusOne(Integer|String x) {
		switch (x)
		case (is Integer|Float) { print(x+1); }
		case (is String) { print("String:"+x); }
	}
\end{lstlisting}
%
Exhaustiveness checking does not prevent the cases in a switch to
accept more than necessary.
For example, in the above function, even though \lstinline{x} cannot be
a \lstinline{Float}, it is not harmful for the first branch to expect
a term of \lstinline{Integer|Float}, since \lstinline{Integer|Float|String}
still covers \lstinline{Integer|String}.


\paragraph{Disjointness}
Ceylon prevents any pair of cases in a switch from overlapping.
And disjointness is introduced to describe such relation between two types.
Being disjoint means the non-existence of a value which can be assigned to
both types.
With the existence of subtyping, a branch in switch/case expression
can take a term of a subtype of its expected type.
Therefore disjointness roughly equals to no common subtype.
However, this does not directly lead to an algorithm. So Ceylon
provides a set of rules: 1) distinct cases in an enumerated type is thought to
be disjoint, and their subtypes are disjoint too; 2) two classes, if they are
not subclass of each other, are disjoint; and etc.
Especially, for a union type $[[A1\/A2]]$, being disjoint with another
type $B$ requires both $[[A1]]$ and $[[A2]]$ to be disjoint with $B$.
Eventually, after decomposing and examining subtyping relation, it is decidable
that two types are disjoint or not.

% https://github.com/ceylon/ceylon-spec/issues/50
% https://github.com/ceylon/ceylon-spec/issues/65

Disjointness is the foundation of
Ceylon's deterministic and order-irrelevant semantics of the switch construct.
Forcing all cases to be disjoint eliminates ambiguity, and avoid subtle bugs that may arize from overlapping cases.
%
People may argue that the programmer should be free to use overlapping cases
and arrange their order intentionally. The following example provides a scenario
where the programmer may not notice the dangerous overlapping.
%
\begin{lstlisting}
	alias Number => Integer | Null;
	String getNumber(Number n) {
		switch (n)
		case (is Integer) { return "``n``"; }
		case (is Null) { return "infinity"; }

	}
	void  printAge(Number|Null x) {
		switch (x)
		case (is Number) { print("Age is ``getNumber(x)``".); }
		case (is Null) { print("Age is not provided."); }
	}
\end{lstlisting}
%
Assume \lstinline{Number} is already defined where \lstinline{null} stands for
infinity. A client uses it to store age information.
Meanwhile, the client use \lstinline{null} to denote missing data.
Without disjointness checking, the above code will be accepted.
But in \lstinline{printAge}, the second branch is always shadowed by the first one.
Any missing data will be interpreted as infinity.
%
Note that swapping the two cases cannot prevent all unexpected behaviors.
In that case the meaning of infinity is hidden.
On contrary, disjoint cases are always distinguishable, and the user, when
using an existing definition, is guaranteed that there is no hidden conflicts.

\paragraph{the bottom type}
To dig deeper for disjointness, it often comes to the the concept
of \emph{bottom type}.
Bottom type is a subtype of all types, in contrast to the top type.
Certainly the bottom type has no value.
In Ceylon, it is called \lstinline{Nothing}, representing the empty set.
%
For two disjoint types $[[A]]$ and $[[B]]$, their intersection $[[A/\B]]$
is naturally a common subtype, and therefore must be equivalent to
the bottom type \lstinline{Nothing}.
\end{comment}

\begin{comment}
\paragraph{Existing problems in Ceylon}
In general, a term of type $A$ is always assignable to any supertype of $A$.
But in Ceylon, the checking of assignability is not complete to
subtyping.
Although the subtyping relation holds between \lstinline{v}'s
(declarative) type and \lstinline{Integer}, \lstinline{v}
is not assignable to \lstinline{Integer}, and the following program
cannot be accepted by Ceylon's compiler.
% https://try.ceylon-lang.org/#

\begin{lstlisting}
	< Character | Integer > & < String | Integer > v = 100;
	switch (v)
	case (is Integer) { print("Integer: ``v``"); }
\end{lstlisting}
\end{comment}

\subsection{Our Methodology}

The switch construct in our calculus \cal is similar to Ceylon's.
Its typing rule guarantees that cases are disjoint and exhaustive.
Reduction preserves types and produces deterministic result in the
runtime, with the help of annotated values.
Here we give an overview of our design and discuss some challenges
we met for the two calculi in the paper.

\paragraph{Disjointness, interacted with intersection types}
After seeing the connection between bottom-like types and disjointness,
it is intuitive to formally define disjointness via bottom-like types.

\begin{definition}\label{def:disjointness}
	A $*_s$ B $\Coloneqq$ $\forall$ C, $[[C <: A]]$ $\wedge$ $[[C <: B]]$ $\rightarrow$ $[[botlike C]]$
	\label{def:union:disj}
\end{definition}

\snow{Here I copy the definition from union.tex. Maybe we can introduce it
earlier.}

Two types are disjoint if and only if all of their common subtypes are bottom-like.
That is to say, there does not exist any term that is assignable to both of
them. Specially, a bottom-like type is disjoint to any types, while the top type
is only disjoint to bottom-like types.
For the detailed discussion and an algorithmic formalization of
disjointness, please refer to Section~\ref{sec:union:disj}.

In literature, there exist a very different definition of disjointness
in calculi with intersection types that also serves for disambiguity
purpose~\cite{oliveira2016disjoint}.
In contrary to subtypes, it restricts the common supertype, or the lowest
upper bound, of two disjoint types to be \emph{top-like}.
Similarly, a top-like type is equivalent to type top, which is the greatest
upper bound of all types.
The difference comes from the subtyping rules for intersections.
Any component of an intersection type is a supertype of it.
If two intersection types share a part, e.g. $[[Int/\Char]]$ and $[[String/\Int]]$
both contains $[[Int]]$, they cannot pass the disjointness checking.
Moreover, assuming $[[Odd]]$ and $[[Even]]$ denote odd numbers and even numbers,
they are both subtype of $[[Int]]$, and therefore are not disjoint.
The application of this definition is dual to our definition:
given a type that is not top-like, consider a scenario where we are looking for
a term that is assignable to the type. If all candidates' types are disjoint,
at most one term can be chosen from them.
\begin{verbatim}
 (\x. x+1 : Int->Int) (1 ,, True) --> (1+1):Int
\end{verbatim}

We alter the definition of disjointness (see Section~\ref{sec:inter:disj})
in our second calculus, which extends \cal by intersection types.
Any two types $[[A]]$ $[[B]]$ have a trivial subtype $[[A/\B]]$.
Generally such an intersection type is not bottom-like.
A new representation for types with no inhabited values is then needed.
At the end, we come up with a concept that is more like the opposite to it:
we use \emph{ordinary} to denote types that have corresponding values.
%
If two types have a common subtype that is ordinary, any inhabited
value of the ordinary type is assignable to them.
Thurs, any ordinary types cannot be a subtype of two disjoint types at
the same time.
%
Literal types are ordinary, like $[[Int]]$.
All function types are viewed as ordinary for simplification.
Compound types like intersection or union types are not included in ordinary
types because they must have subtype that is a literal type if they have
inhabited values.
%
Then we can prove $[[Int/\String\/Bool]]$ is disjoint with $[[Int/\String\/Char]]$.
Although they have subtype $[[Int/\String]]$, it is not ordinary.
And $[[Int\/String]]$ is not disjoint with $[[Bool\/String]]$, because of the
subtype $[[String]]$.
Especially, types corresponding to no values, like $[[Int/\String]]$, are
disjoint with any types.

We cannot enumerate all subtypes of two types to check disjointness.
Instead we have an algorithmic definition of disjointness (Figure~\ref{fig:union:disj-typ}) for \cal.
These rules structurally examine two given types, decomposing unions
and comparing their literal components.
For example, $[[Int\/String]]$ is disjoint to $[[Char]]$ because both types
in it is disjoint to $[[Char]]$.
%
However, this approach cannot be directly employed to the intersection
extension.
Like we said above, some intersection is disjoint to other types
because itself is uninhabited.
So even neither $[[Int]]$ or $[[String]]$ is disjoint with $[[Int\/String]]$,
together $[[Int/\String]]$ is disjoint with it.
Alternatively, for a given type, we use a set of ordinary types to denote
all possible values of it. For instance, \verb"{Int,Char}" represent the
union type $[[Int\/Char]]$.
The set is calculated by function \lstinline{LOS} (\emph{lowest ordinary
subtypes})in Section~\ref{}.
To check whether two types are disjoint, we calculate the intersection
of their sets and see if it is empty.
From another angle, we normalize and simplify types by \lstinline{LOS},
and then are able to directly detect uninhabited ones.

The set representation is justified by the distributivity in subtyping (Figure~\ref{}).
Every type in the set denotes a component in a union type.
That is to say, any inhabited type must have an equivalent union type,
except for literal types.
Only with distributivity, we have $[[(A\/B)/\C]]$ equivalent to
$[[(A/\C)\/(B/\C)]]$.
The soundness and completeness of the algorithmic definition is proved
with respect to the declarative definition.
Same applies to the two definitions for \cal without intersection.


\paragraph{Typing and exhaustiveness}
In \cal, a switch expression has two branches. For multiple cases,
one can write nested switch expression.
We assume the two branches expect $[[A]]$ and $[[B]]$.
To make sure they exhaust all possible types of the switched term $[[e]]$,
there is a premise that $[[e]]$ can be checked by $[[A\/B]]$.
\verb|<=| stands for checking mode in bidirectional typing,
on contrary to inference mode.
In other words, the inferred type of $[[e]]$ should be a subtype of $[[A\/B]]$,
like $[[Int]]$ to $[[Int\/Char]]$.
%
\begin{mathpar}
	\ottdruletypXXswitch{}
\end{mathpar}
%
Another premise requires the two cases to be disjoint.
Besides, the two branches are typed under different assumption of the bound
variable. Although the same type $[[C]]$ is used for both of them in the rule,
it does not prevent them to return different types.
Assuming the inferred type of $[[e1]]$ is $[[C1]]$ and the inferred type of
$[[e2]]$ is $[[C2]]$, we can make $[[C]]$ to be $[[C1\/C2]]$.

\paragraph{Reduction and annotated values}
Our reduction preserves type.
So a term of union type can only evaluates to a value of an union type.
We use type annotations to construct such values.
For example, $[[2:Int\/Bool]]$ is a value, and it is the result of
$[[(switch 1 Int (x p1) Bool (neg y)):Int\/Bool]]$.
This annotation is necessary for precise type preservation under
bidirectional typing. It keeps the inferred type of expressions.

\snow{I think we might be able to relax preservation by allowing subtyping.
For which precise reason we failed in that variant? But I guess it is ok
to omit that unless the reviewers ask}

Lambda values can be unannotated or with two annotations.
The outer one serves the above purpose, while the inner one keeps the original
input type, which is used to decorate the input value in beta reduction.
% The outer one is for the preservation purpose,
% while the inner one records its principal type.
%
\begin{verbatim}
e = switch (x) {Char -> ... , Bool -> ...}

(\x . e : Top -> Int : Int -> Int) (1:Int) --> (\x . e : Int->Int) (1:Int) --> e [x~>1:Int]
\end{verbatim}
The expression was legal when x has type Top but becomes illegal when x has type Int.
It shows that we have to keep the original input type in lambdas.

\snow{Updated}
%\snow{I don't know why we need to keep a lambda's original type. We
%don't need to distinguish two arrow types in switch.}

Deciding to take which branch in the runtime requires knowing the
precise type of the switched term.
Instead of the annotation, we need to look into the wrapped value
(which is called as \emph{pre-value}) for its principal type.
Given $[[2:Int\/Bool]]$, we still know that it is an integer inside.
We then look for a branch that expects a supertype of $[[Int]]$.
As previously discussed, the exhaustiveness checking promises at least
one branch matches.
And the disjointness restriction ensures it can only be one branch.
Therefore the reduction is deterministic.
If the two branch types are both supertype of the value type, they violates
the definition, no matter the Definition~\ref{def:disjointness} or the
improved one.
In the following example, the first branch will be chosen, just like when
we passing $[[1]]$ to it.

\begin{verbatim}
       (switch (2:Int\/Bool) Int (x p1) Bool (neg y))
       --> (2:Int p1) :Int\/Bool
       --> 3 : Int \/ Bool
\end{verbatim}

Wrapping values with annotation also helps us to unify the related reduction rules.
\snow{I guess so?}



\bruno{
1) use of annotatted values for both reduction (the semantics must be type directed); and preservation;
2) Disjointness based on disjoint intersection, and later novel notion of disjointness in the presence of intersections;
3) LOS for algorithmic disjointness;
4) Dealing with exhaustiveness and reduction in the switch construct
}

\begin{comment}
\bruno{For the following text, this is the level of detail that I expect
  for the introduction, but not for the overview! The overview should
  identify key ideas and challenges in more detail. So the following text
  is not what I'm hoping for here. For an example of what I'm hoping for here,
  see for instance Section 2.3 and 2.4 in ``Disjoint Intersection Types'' or
  Section 2.3, 2.4 and 2.5 in "A Type-Directed Operational Semantics for a
  Calculus with a Merge Operator''
}
In this paper, we follow type-based union elimination similar to Ceylon.
We enforce disjointness constraint on the types of branches of a switch
expression. On the contrary to Ceylon, we provide a formal definition
of disjointness. Our disjointness definition is inspired by $\lambda_{i}$
\cite{oliveira2016disjoint}. $\lambda_{i}$ formally defines disjointness
for intersection types and merge operator.
Intersection of two types $[[A]]$ and $[[B]]$ is dijoint in $\lambda_{i}$
if $[[A]]$ and $[[B]]$ do not share a common supertype which is not
\emph{top-like}. So-called \emph{top-like} types are defined in
\cite{oliveira2016disjoint} and are such types which are \emph{supertypes}
of all other types.

Union types are usually considered to be dual of intersection types.
Therefore, we propose a dual definition of \emph{bottom-like} types.
In contrast to \emph{top-like} types, \emph{bottom-like} types are
subtypes of all other types. Further, in contrast to the disjointness
definition in $\lambda_{i}$, two types $[[A]]$ and $[[B]]$ in the simplest
caluclus studied in this paper are disjoint if $[[A]]$ and $[[B]]$
do not share any common \emph{subtype} which is not \emph{bottom-like}.
Note that $\lambda_{i}$ proposed \emph{top-like} with \emph{supertypes}.
We propose \emph{bottom-like} with \emph{subtypes}. This will be discussed
in detail in \Cref{sec:union}.

Adding intersection types in our calculus
together with union types poses non-trivial challenges on disjointness.
Specifically, it makes it impossible to define a complete disjointness
definition. Therefore, the simple disjointness definition which is dual
to $\lambda_{i}$ no longer works. We define a notion of ordinary types
and define disjointness based upon ordinary types to overcome the
challenge in completeness. Informally, updated disjointness definition states
that two types $[[A]]$ and $[[B]]$ are disjoint if they do not share any
common ordinary subtype. This will further be discussed in detail in
\Cref{sec:inter}.

We propose a sound and complete disjointness definition. Caclulus studied
in this paper is type-safe and deterministic.

\bruno{
	Introduce our work, setting the goal to study the construct formally.
	Connect with the work on disjoint intersection types, which also
	employs a notion of disjointness, but for intersection. Explain that
	what is needed is a dual notion of disjointness.\\
	Introduce the first calculus, and explain that it is directly inspired
	by a dual notion of discjointness.\\
	Introduce the second calculus and identify a technical challenge with
	disjointness: the addition of intersection types breaks the previous
	notion of disjointness. Introduce the novel way to find disjoint types.\\
	Summarize/mention key results: type-safety; soundness/completeness of
	disjointness; determinism.\\
	Perhaps here it is also useful to identity, together with Baber, what
	were the most challenging aspects in the formalization, and maybe
	highlight these.
}

\end{comment}
%%% Local Variables:
%%% mode: latex
%%% TeX-master: "../paper"
%%% org-ref-default-bibliography: "../paper.bib"
%%% End:

\section{The Union Calculus (\name)}
\label{sec:union}

\snow{Pls fix ``and without intersection types''. Besides don't use commands for
  text because it leads to incorrect capitalization. You can use ``emph'' instead
  of capitalizing.} \baber{I guess Bruno already fixed that.}
This section introduces the union calculus \name. The distinctive feature
of the \name calculus is a type-based switch expression with disjoint
cases, which can be used to eliminate values with union types.
%Such type-based
%switch expression is inspired by a similar construct in the Ceylon programming
%language.
We adapt the notion of disjointness from previous work on
\emph{disjoint intersection types}~\cite{} to \name, and show that \name is type
sound and deterministic.

%%%%%%%%%%%%%%%%%%%%%
%% Syntax
%%%%%%%%%%%%%%%%%%%%%

\subsection{Syntax}\label{sec:union:syntax}
\snow{The latex definition of typeof in ott needs to be revised. Now ``typeof''
  and ``as'' are in mathmode, I think they look better in text mode.
  Same applies to ``ord''.
  And why there are two latex definition for typeof in ott? The other one
  has variable ``x'' in it. I think we do need such a variable.
  Pls fix the notation.}
\bruno{Lets change ``typeof e as'' to ``switch e'', and Snow is right:
you need the variables, brackets eytc. in the syntax.}
\baber{Done, updated typeof to $[[switch e A e1 B e2]]$.}
\Cref{fig:union:syntax} shows the syntax for \cal. Metavariables
$[[A]]$, $[[B]]$ and $[[C]]$ range over types.  Types include top ($[[Top]]$),
bottom ($[[Bot]]$), function ($[[A -> B]]$) and union ($[[A \/ B]]$)
types. Metavariable $[[e]]$ ranges over program
expressions. Expressions include variables ($[[x]]$), natural numbers
($[[i]]$), type annotations ($[[e:A]]$), lambda abstractions
($[[\x.e]]$), applications ($[[e1 e2]]$) and a novel switch ($[[switch
    e A e1 B e2]]$) expression. \emph{Switch} expression is a case
expression and evaluates a specific branch by matching the
type.
%Details of \typeof expression will further be discussed in typing
%and operational semantics sections.

\paragraph{Values and pre-values.} In \name all the values have
a type annotation. The type annotation represents the dynamic type
that the value has at runtime, and
it is helpful for the type-directed dynamic semantics.
We divide the representation for values into two parts: pre-values and
values.
Metavariable $[[p]]$ ranges over pre-values. Pre-values
consist of natural numbers $[[i]]$
and annotated lambda expressions $[[\x.e : A -> B]]$. Values are
annotated pre-values. Metavariable $[[v]]$ ranges over
values. It is important to note that $[[i]]$ is not a value in this
calculus, instead $[[i:Int]]$ or $[[i:Top]]$ are values.
In addition, lambdas have two annotations. That is $[[\x.e : A -> B : C]]$ is a value.
As already mentioned, $C$ is the dynamic type of the value
at runtime, whereas $[[A -> B]]$ is the original \emph{static type} of the lambda.
For readers familiar with calculi with gradual types~\cite{}, the two annotations
can be also be understood as the \emph{source type} and \emph{target type}
of an upcast: i.e. if the value is well-typed we have that $[[A -> B]] <: [[C]]$.

A context ($[[G]]$) can
either be empty or contains type bindings of variables and associated
types. Finally, a typing mode ($[[dirflag]]$) can either be check ($[[<=]]$)
or inference ($[[=>]]$).

\begin{figure}[t]
  \begin{small}
    \centering
    \begin{tabular}{lrcl} \toprule
      Types & $[[A]], [[B]]$, $[[C]]$ & $\Coloneqq$ & $ [[Top]] \mid [[Bot]] \mid [[Int]] \mid [[A -> B]] \mid [[A \/ B]] $ \\
      Expr & $[[e]]$ & $\Coloneqq$ & $[[x]] \mid [[i]] \mid [[e:A]] \mid [[\x.e]] \mid [[e1 e2]] \mid [[switch e A e1 B e2]]$\\
      PValue & $[[p]]$ & $\Coloneqq$ & $[[i]] \mid [[\x.e : A -> B]] $\\
      Value & $[[v]]$ & $\Coloneqq$ & $[[p:A]]$\\
      Context & $[[G]]$ & $\Coloneqq$ & $ \cdot \mid [[G , x : A]]$ \\
      Mode & $[[dirflag]]$ & $\Coloneqq$ & $[[<=]] \ \mid \ [[=>]]$ \\
      \bottomrule
    \end{tabular}
  \end{small}
  \begin{small}
    \centering
    \drules[s]{$ [[A <: B ]] $}{Algorithmic Subtyping}{top, bot, int, arrow, ora, orb, orc}
  \end{small}
  \caption{Syntax and Algorithmic Subtyping for \cal. \snow{We need to add C as
      a metavariable for types here because it is used later.} \baber{Done.} }
  \label{fig:union:syntax}
\end{figure}
\bruno{Add the syntax for the bi-directional modes in the Figure.} \baber{Done.}

%%%%%%%%%%%%%%%%%%%%%
%% Subtyping
%%%%%%%%%%%%%%%%%%%%%
\subsection{Subtyping}
\label{sec:union:sub}
Algorithmic subtyping rules for \cal are shown in
\Cref{fig:union:syntax}. The subtyping rules are standard for a system
with union types.  \Rref{s-top} states that all types are subtypes of
the $[[Top]]$ type. \Rref{s-bot} states that $[[Bot]]$ type is subtype of
all types. \Rref{s-int, s-arrow} are standard rules for integers and
functions respectively.  Functions are contravariant in input types
and covariant in output types. \Rref{s-ora, s-orb, s-orc} are standard
subtyping rules for union types. The union type of two types $A1$ and $A2$
is a subtype of another type $A$ if both $A1$ and $A2$ are subtypes of
$A$, as stated in \rref{s-ora}. \Rref{s-orb, s-orc} states that if a
type is subtype of one of the components of a union type, then it is subtype of whole
union type.  The subtyping relation for \cal is reflexive and transitive.
\begin{lemma}[Subtyping Reflexivity]
  $[[A <: A]]$.
\label{lemma:union:refl}
\end{lemma}
\begin{comment}
\begin{proof}
  By induction on type A. All cases are trivial to prove.
\end{proof}
\end{comment}
\begin{lemma}[Subtyping Transitivity]
  If \ $[[A <: B]]$ \ and \ $[[B <: C]]$ \ then \ $[[A <: C]]$.
  \label{lemma:union:trans}
\end{lemma}
\begin{comment}
\begin{proof}
  By induction on type B.
  \begin{itemize}
    \item Cases $[[Top]]$, $[[Bot]]$ and $[[Int]]$ are trivial to prove.
    \item Case $[[A -> B]]$ requires double induction on type $[[C]]$
          and $[[A]]$.
    \item Case $[[A \/ B]]$ requires \Cref{lemma:union:sub-or}
  \end{itemize}
\end{proof}\bruno{If space is a concern we can probably drop the lemma statements
for reflexivity and transitivity as these are quite standard.}

\begin{lemma}[Subtyping Union Inversion]
\label{lemma:union:sub-or}
  If \ $[[A \/ B <: C]]$ then:
  \begin{enumerate}
    \item $[[A <: C]]$ and
    \item $[[B <: C]]$
  \end{enumerate}
\end{lemma}
\end{comment}


%%%%%%%%%%%%%%%%%%%%%%%
%% Disjointness
%%%%%%%%%%%%%%%%%%%%%%%
\subsection{Disjointness}
\label{sec:union:disj}
\baber{Mainly focused on technical details in this section. Story and benefits of disjointness may be discussed in another section.}
In this section we discuss in detail the notion of disjointness for
union types and case expression for \cal. In essence disjointness for \cal is
the dual to disjointness for $\lambda_i$~\cite{oliveira2016disjoint}, which is
a calculus with disjoint intersection types. In $\lambda_i$, two
types in are disjoint if they do not share any common
\emph{supertype} which is not \emph{top-like}. In contrast, in
\cal, two types in are disjoint if they do not share any common \emph{subtype} which
is not \emph{bottom-like}.
\bruno{In the overview section we need to talk about more about this and
explain that common subtypes are important for the switch expression.}
We emphasize the significance of
\emph{supertypes} and \emph{subtypes} in $\lambda_i$ and \cal
respectively.

\paragraph{Bottom-Like Types}
\emph{Bottom-like} types are types that are isomorphic (i.e.
both supertypes and subtypes) of the type $\bot$. In \name there
are infinitely many such types, including, for example $\bot \lor \bot$,
$\bot \lor \bot \lor \bot$, as well as $\bot$ itself. Bottom-like types
are important because they allow us to define disjointness.
%are integral part of disjoitness in \cal like
%\emph{top-like} in $\lambda_i$ \cite{oliveira2016disjoint}. Therefore,
%it is important to understand the notion of \emph{bottom-like} types
%before diving into the details of disjointness.
Intuitively, a
\emph{bottom-like} type is a type which behaves like $[[Bot]]$ type.
An inductive definition that captures all the bottom-like types
is shown at the top of \Cref{fig:union:disj-typ}.
Type $[[Bot]]$ is obviously a \emph{bottom-like} type
(\rref{bl-bot}), and a union type of two \emph{bottom-like} types is also
a \emph{bottom-like} type (\rref{bl-or}).  It is trivial to conclude
that a union type is \emph{bottom-like} only if all the primitive
types in union are $[[Bot]]$. The correctness of our definition for
bottom-like types is ensured by the following properties:

\begin{lemma}[Bottom-Like Soundness]
  If \ $[[botlike A]]$ \ then \ $[[A <: B]]$.
\label{lemma:union:bl-soundness}
\end{lemma}

\begin{comment}
\begin{proof}
  By induction on bottom-like relation.
  \begin{itemize}
    \item All cases are trivial to prove.
  \end{itemize}
\end{proof}
\end{comment}

\begin{lemma}[Bottom-Like Completeness]
  If \ $[[A <: B]]$ \ then \ $[[botlike A]]$.
\label{lemma:union:bl-completeness}
\end{lemma}

\begin{comment}
\begin{proof}
  By induction on type $[[A]]$.
  \begin{itemize}
    \item Cases $[[Top]]$, $[[Bot]]$, $[[Int]]$ and $[[A -> B]]$ are trivial to prove.
    \item Case $[[A \/ B]]$ requires \Cref{lemma:union:sub-or}.
  \end{itemize}
\end{proof}
\end{comment}

\paragraph{Declarative Disjointness}
The formal (declarative) definition for disjointness in \name is:
%Recall that two types in \cal are disjoint if they do not share any common subtype which is not
%\emph{bottom-like}. The formal definition of disjoint specifications for this calculus is:

\begin{definition}
  A $*_s$ B $\Coloneqq$ $\forall$ C, $[[C <: A]]$ $\wedge$ $[[C <: B]]$ $\rightarrow$ $[[botlike C]]$
\label{def:union:disj}
\end{definition}

\noindent That is, two types are disjoint if all their common subtypes are bottom-like.
\begin{comment}
With this definition we have that different primitive types are disjoint. For example
$[[Int]] * [[Bool]]$ since the only common subtypes of $[[Int]]$ and $[[Bool]]$
are bottom-like. A more interesting case is the disjointness of two function types.
It turns out that function types are never disjoint, since we can always find
a common subtype for any two function types. For example, if we have $[[Int -> Bool]]$
and $[[String -> Char]]$ then a common subtype that is not bottom-like is
$[[Top -> Bot]]$. Therefore, $[[Int -> Bool]]$ and $[[String -> Char]]$ are not
disjoint.

\noindent Reader may think at this point that $[[Bot]]$ type can simply be used in \Cref{def:union:disj}
instead of $[[botlike C]]$ in the conclusion. Answer to this question is
union type with $[[Bot]]$ as all primitive types is also a least subtype in \cal.
$[[botlike C]]$ also handles this case.
\end{comment}
We illustrate this definition with a few simple examples:

\begin{enumerate}
  \item $\boldsymbol{A = [[Int]], \ B = \ [[Int -> Bool]]:}$ \\
        $[[Int]]$ and $[[Int -> Bool]]$ are disjoint types. All common subtypes of $[[Int]]$ and $[[A -> B]]$ are bottom-like types, 
        including $[[Bot]]$ and union of $[[Bot]]$ types. 
  \item $\boldsymbol{A = [[Int]], \ B = \ [[Bot]]:}$ \\
    $[[Int]]$ and $[[Bot]]$ are disjoint types, since again all common subtypes are bottom-like. In general, the type $[[Bot]]$ (or any other bottom-like type)
    is disjoint to any other type.
  \item $\boldsymbol{A = [[Int]], \ B = \ [[Int]]:}$ \\
        $[[Int]]$ and $[[Int]]$ are not disjoint types because they share a common subtype $[[Int]]$ which
        is not \emph{bottom-like}.
  \item $\boldsymbol{A = [[Int]], \ B = \ [[Top]]:}$ \\
        $[[Int]]$ and $[[Top]]$ are not disjoint types because they share a common
    subtype $[[Int]]$ which is not \emph{bottom-like}. In general no type
    is disjoint to $[[Top]]$. 
  \item $\boldsymbol{A = [[Int -> Bool]], \ B = \ [[String -> Char]]:}$
    The types $[[Int -> Bool]]$ and $[[String -> Char]]$ are not disjoint,
    since they share we can find non-bottom-like types that are subtypes
    of both types. For instance $[[Top -> Bot]]$ is a subtype of both types.
    More generally, any two function types can never be disjoint: it is always
    possible to find a common subtype, which is not bottom-like.
    \\
\end{enumerate}

\begin{comment}
\begin{figure}[t]
  \begin{small}
    \centering
    \drules[ad]{$[[A * B]]$}{Algorithmic Disjointness}{btmr, btml, intl, intr, orl, orr}
  \end{small}
  \caption{Algorithmic Disjointness for \cal.}
  \label{fig:union:ad}
\end{figure}
\end{comment}

\paragraph{Algorithmic Disjointness}
The middle part of \Cref{fig:union:disj-typ} shows an algorithmic
version of disjointness.  \Rref{ad-btmr, ad-btml} state that the $[[Bot]]$
type is disjoint to all types.  \Rref{ad-intl, intr} state that
$[[Int]]$ and $[[A -> B]]$ are disjoint types.  Algorithmic
disjointness can further be scaled to more primitive disjoint types
such as $Bool$ and $String$ by adding more rules similar to
\rref{ad-intl, intr} for additional primitive types.  \Rref{ad-orl,
  ad-orr} are two symmetric rules for union types. Any type $[[C]]$ is
disjoint to an union type $[[A \/ B]]$ if $[[C]]$ is disjoint to both
$[[A]]$ and $[[B]]$.  Algorithmic disjointness is sound and complete
with respect to \Cref{def:union:disj}, and disjointness is a symmetric
relation. The following lemmas summarize key properties of disjointness.

\begin{lemma}[Disjointness Soundness]
  If \ $[[A * B]]$ \ then \ $[[A *s B]]$.
\label{lemma:union:disj-sound}
\end{lemma}

\begin{comment}
\begin{proof}
  By induction on algorithmic disjointness relation.
  \begin{itemize}
    \item Cases \rref{ad-btmr, ad-btml, ad-orl, ad-orr} require induction on hypothesis
          and \Cref{lemma:union:sub-or}.
    \item Cases \rref{ad-intl, ad-intr} require induction on type and \Cref{lemma:union:sub-or}.
  \end{itemize}
\end{proof}
\end{comment}

\begin{lemma}[Disjointness Completeness]
  If \ $[[A *s B]]$ \ then \ $[[A * B]]$.
\label{lemma:union:disj-complete}
\end{lemma}

\begin{comment}
\begin{proof}
  By induction on type A.
  \begin{itemize}
    \item Case $[[Top]]$ requires \Cref{lemma:union:bl-disj}.
    \item Case $[[Bot]]$ is trivial to prove.
    \item Case $[[Int]]$ requires induction on type B and
          \Cref{lemma:union:bl-disj,lemma:union:disj-sym}.
    \item Case $[[A -> B]]$ requires induction on type B and \Cref{lemma:union:disj-sym}.
    \item Case $[[A \/ B]]$ follows directly from inductive hypothesis.
  \end{itemize}
\end{proof}

\begin{lemma}[Bottom-like Types Disjoint]
\label{lemma:union:bl-disj}
  If \ $[[botlike A]]$ \ then: \ $[[A * B]]$.
\end{lemma}

\begin{lemma}[Disjointness Symmetry]
\label{lemma:union:disj-sym}
  If \ $[[A * B]]$ \ then: \ $[[B * A]]$.
\end{lemma}
\end{comment}

\begin{lemma}[Bottom-Like Disjoint]
  If \ $[[botlike A]]$ \ then \ $[[A * B]]$.
\label{lemma:union:bl-disjoint}
\end{lemma}

\begin{lemma}[Disjointness Symmetry]
  If \ $[[A * B]]$ \ then \ $[[B * A]]$.
\label{lemma:union:disj-sym}
\end{lemma}

\begin{figure}[t]
  \begin{small}
    \centering
    \drules[bl]{$[[botlike A]]$}{Bottom-Like Types}{bot, or}
  \end{small}
  \begin{small}
    \centering
    \drules[ad]{$[[A * B]]$}{Algorithmic Disjointness}{btmr, btml, intl, intr, orl, orr}
  \end{small}
  \begin{small}
    \centering
    \drules[typ]{$ [[G |- e dirflag A]] $}{Bidirectional Typing}{int, var, ann, app, sub, abs, switch}
  \end{small}
  \caption{Bottom-Like types, Algorithmic Disjointness and Typing for \cal.}
  \label{fig:union:disj-typ}
\end{figure}
\bruno{The syntax of typeof needs to be fixed, otherwise the variable ``x'' in the
typing rule appears from nowhere!}
\baber{Done, updated typeof to $[[switch e A e1 B e2]]$.}


%%%%%%%%%%%%%%%%%%%%%
%% Typing
%%%%%%%%%%%%%%%%%%%%%
\subsection{Typing}
\label{sec:union:typ}
The typing rules are shown at the bottom of \Cref{fig:union:disj-typ}.
We adopt bidirectional type-checking~\cite{} in our calculus.  There
are two typing modes in bidirectional typing: inference mode
($[[=>]]$) and checking mode ($[[<=]]$). In inference mode, the type of
an expression $[[e]]$ is inferred or calculated based upon certain
information available in the given context $[[G]]$.  While in checking
mode, an expression $[[e]]$ is checked against a given type $[[A]]$.
Typing rules are mostly standard.  An integer
expression $[[i]]$ infers type $[[Int]]$ as stated in \rref{typ-int}.
\Rref{typ-var} states that a variable $[[x]]$ infers type $[[A]]$ if
$[[x]]$ has type $[[A]]$ in the given context. \Rref{typ-ann} states
if an expression $[[e]]$ checks against type $[[A]]$, then the
annotated expression $[[e:A]]$ infers type $[[A]]$.
\Rref{typ-app} type checks a function application and it is the
elimination rule for functions.
\begin{comment}
Expression $[[e1]]$ has to
be a function expression and expression $[[e2]]$ has to check against
input type of $[[e1]]$.  An important point to notice in
\rref{typ-app} is $[[e1]]$ infers type $[[A -> B]]$. This may look
weird at the very first glance because lamda expressions ($[[\x.e]]$)
are not annotated in program expressions $[[e]]$ and it seems not
possible for lambda expression to infer its type.  To answer this
question, we emphasize the use of partial expressions $[[p]]$ and
values $[[v]]$.  Lambda expression is annotated in $[[p]]$ and so in
$[[v]]$ because values are defined as annotated partial expressions.
\end{comment}
\Rref{typ-sub} is the subsumption rule. It states that an expression
$[[e]]$ can be checked against any supertype of its inferred type.
\Rref{typ-abs} is the standard introduction rule for lambda
expressions. To check a lambda expression $[[\x.e]]$ against type $[[A
    -> B]]$, it is sufficient to check lambda body $[[e]]$ against the
output type $[[B]]$ in an extended context with parameter $[[x]]$ of
input type $[[A]]$.

The most interesting and novel typing rule is for
\emph{switch} expressions. Four conditions are necessary for typing
\emph{switch} expressions.
%The remaining conditions are standard for a calculus with
%union types and case expression and have been studied in various
%calculi (\baber{reference to calculi}).
The first condition ($[[G |-
    e <= A \/ B]]$) ensures that case expression $[[e]]$ is well-typed
and checks against type $[[A \/ B]]$.  The next two conditions ensure that
branches of case expression are well-typed and check against some type
$[[C]]$. An important point in these two conditions is that variable
$[[x]]$ is of type $[[A]]$ in first branch and of type $[[B]]$ in
second branch in the extended context.  From the last condition
$[[A *s B]]$, we guarantee that $[[A]]$ and $[[B]]$ are disjoint
types. Overlapping types for the 
branches of case expressions can lead to non-deterministic 
results, and are therefore forbidden.
Since all the branches check against $[[C]]$, the whole
\emph{switch} expression checks against $[[C]]$.

\begin{comment}
\begin{figure}[t]
  \begin{small}
    \centering
    \drules[typ]{$ [[G |- e dirflag A]] $}{Bidirectional Typing}{int, var, ann, app, sub, abs, typeof}
  \end{small}
  \caption{Typing for \cal.}
  \label{fig:union:typ}
\end{figure}
\end{comment}

%%%%%%%%%%%%%%%%%%%%%%%
%% Operations Semantics
%%%%%%%%%%%%%%%%%%%%%%%
\subsection{Operational Semantics}
\label{sec:union:os}
Dynamics of \cal are defined by small-step operational semantics.
\Cref{fig:union:os} shows operational semantics of \cal. Before going
into the details of operational semantics, it is important to recall
partial expressions and values. An integer expression $[[i]]$ or a
lambda expression $[[\x.e]]$ are not values unless annotated. Lambda
expression has to be dually annotated to fulfill the criteria to be a
value. Operational semantics follow call-by-value evaluation strategy.
We empahsize the use of another interesting and novel technique which
is type directed operational semantics \baber{reference here}. Types
are preserved by type annotations during substitution in
beta-reduction and \emph{switch} cases.

\Rref{step-int} annotates integer expressions and makes them values. \Rref{step-appl} reduces left expression
of an application unless it becomes a value. \Rref{step-appr} works only if left expression of an
application is already reduced to value. \Rref{step-appr} then reduces right expression of
application to a value.
\Rref{step-beta} is the beta-reduction. It applies dually annotated lambda expression
$[[\x.e : A1 -> B1 : A2 -> B2]]$ to input value $[[p:C]]$. Substitution replaces free occurances of variable
$[[x]]$ with value $[[p:A1]]$. Note that annotation of partial value $[[p]]$ has changed to $[[A1]]$ from
$[[C]]$ during substitution.
This is to keep the most specific type in annotation. Substituted expression is also
annotated with both of the output types from annotated lambda expression. \baber{More details.}

\Rref{step-absbeta} deals with the simple lambda expressions of the form $[[\x.e]]$ as input to a function.
It follows same as \rref{step-beta} except that both of the input types are kept with $[[\x.e]]$ during
substitution i.e $[[\x.e:A2:A1]]$. \Rref{step-ann} reduces an annotated expression only if it is not a value
and $[[e]]$ reduces to some $[[e']]$. \Rref{step-rmann} drops inner annotation. \Rref{step-lamann} adds one
more type annotation with lambda expressions having single type annotation to make them values.
\Rref{step-switch, step-switchl, step-switchr} deal with the reduction of \emph{switch} expressions.

\Rref{step-switch} reduces the case expression $[[e]]$ unless it becomes a value of the form $[[p:D]]$.
\Rref{step-switchl} evaluates left branch of the \emph{switch} expression if type of $[[e]]$ is subtype
of type of left branch.
\Rref{step-switchr} evaluates right branch of the \emph{switch} expression if type of $[[e]]$ is subtype
of type of right branch.
Subtyping condition in \rref{step-switchl, step-switchr} is interesting and important to consider.
It gives a freedom of various subtypes of $[[A]]$ and $[[B]]$ for corresponding branch instead of only
type $[[A]]$ and type $[[B]]$.
It is important to note that types of left and right branches of \emph{switch} expression cannot overlap
because of disjoitness contraint in typing. Programs with overlapping types in branches of \emph{switch}
expression will not type check in \cal.
A natural property of \cal is if type $[[A]]$ and type $[[B]]$ are two
disjoint types, then subtypes of $[[A]]$ are also disjoint to subtypes of $[[B]]$.

\paragraph{FindType Relation} FindType relation calculates the type of partial expressions
and is shown in lower part of \Cref{fig:union:os}. It takes $[[p]]$ as input and returns its type.
Partial expressions in \cal are $[[i]]$ and $[[\x.e: A -> B]]$.
FindType returns $[[Int]]$ when $[[p]]$ is an integer $[[i]]$ (\rref{findtype-int}). On the other hand,
FindType returns $[[A -> B]]$ against the partial expression $[[\x.e : A->B]]$ (\rref{findtype-arrow}).

\begin{figure}[t]
  \begin{small}
    \centering
    \drules[step]{$[[e --> e']]$}{Operational Semantics}{int, appl, appr, beta, absbeta, ann, rmann, lamann, switch, switchl, switchr}
  \end{small}
  \begin{small}
    \centering
    \drules[findtype]{$[[findtype p A]]$}{FindType}{int, arrow}
  \end{small}
  \caption{Operational semantics for \cal and FindType.}
  \label{fig:union:os}
\end{figure}


%%%%%%%%%%%%%%%%%%%%%%%%%%%%%%
%% Type Safety and Determinism
%%%%%%%%%%%%%%%%%%%%%%%%%%%%%%
\subsection{Type Safety and Determinism}
\label{sec:union:safety}
\cal is type safe and deterministic. In this section we discuss the
proofs of type safety and determinism for \cal. Type safety is usually considered as composition
of type preservation and progress lemma. Type preservation (\Cref{lemma:union:preservation})
states that types are preserved during
reduction. Progress (\Cref{lemma:union:progress}) states well typed programs never get stuck.
A well typed expression $[[e]]$ is
either a value or it can take step to some $[[e']]$. Type of $[[e']]$ is preserved following the
type preservation property. Therefore, preservation and progress together ensure type safety.
We add one more choice in the conslusion of prorgress lemma to handle non-annotated lambda expressions
($[[\x.e]]$).
Determinism (\Cref{lemma:union:determinism}) is also a critical property for a calculus.
Non-deterministic calculi lead to incoherent programs.
Determinism ensures that a program will always produce the same unique result.

\begin{lemma}[Type Preservation]
\label{lemma:union:preservation}
  If \ $[[G |- e dirflag A]]$ and $[[e --> e']]$ then $[[G |- e' dirflag A]]$.
\end{lemma}

\begin{comment}
\begin{proof}
  By induction on typing relation and subsequent inverting reduction relation.
  \begin{itemize}
    \item Cases \rref{typ-int, typ-var, typ-sub, typ-abs} are trivial to prove.
    \item Case \rref{typ-ann} requires helping \cref{lemma:union:check-pexpr-ann}.
    \item Case \rref{typ-app} requires helping \cref{lemma:union:pexpr-check-sub}
          and substitution \cref{lemma:union:substitution} for beta reduction.
    \item Case \rref{typ-typeof} requires substitution \cref{lemma:union:substitution}.
  \end{itemize}
\end{proof}

\baber{ToDo: change name of helping lemmas.}

\begin{lemma}[check-pexpr-ann]
\label{lemma:union:check-pexpr-ann}
  If \ $[[G |- p:C <= A]]$ \ then \ $[[G |- p <= A]]$.
\end{lemma}

\begin{lemma}[pexpr-check-sub]
\label{lemma:union:pexpr-check-sub}
  If \ $[[G |- p <= A]]$ \ and \ $[[A <: B]]$ \ then \ $[[G |- p <= B]]$.
\end{lemma}
\end{comment}

\begin{lemma}[Substitution]
\label{lemma:union:substitution}
  If \ $[[G, x:B , G1 |- e dirflag A]]$ \ and \ $[[G |- e' => B]]$
  then \ $[[G, G1 |- e [ x ~> e' ] dirflag A]]$
\end{lemma}

\begin{lemma}[Progress]
\label{lemma:union:progress}
If \ $[[ [] |- e dirflag A]]$ then
 \begin{enumerate}
  \item either $[[e]]$ is a value.
  \item or $[[e]]$ can take a step to $[[e']]$.
  \item or $[[e]]$ = $[[\x.e]]$ and $[[dirflag]]$ = $[[<=]]$
  \end{enumerate}
\end{lemma}

\begin{comment}
\begin{proof}
By induction on typing relation.
  \begin{itemize}
    \item Cases \rref{typ-int, typ-var, typ-app, typ-sub, typ-abs} are trivial to prove.
    \item Case \rref{typ-anno} requires \cref{lemma:union:value-not-value}.
    \item Case \rref{typ-typeof} requires
    \cref{lemma:union:check-pexpr-ann,lemma:union:check-or-typ,lemma:union:pexpr-inf-typ}.
  \end{itemize}
\end{proof}

\begin{lemma}[Value Decidability]
\label{lemma:union:value-not-value}
$\forall$ $[[e]]$, \ value \ $[[e]]$ \ $\vee$ \ $\neg$ value \ $[[e]]$.
\end{lemma}

\begin{lemma}[check-or-typ]
\label{lemma:union:check-or-typ}
If \ $[[A *s B]]$ \ and \ $[[G |- p <= A \/ B]]$ \ then:
  \begin{enumerate}
    \item either \ $[[G |- p <= A]]$
    \item or \ $[[G |- p <= B]]$
  \end{enumerate}
\end{lemma}

\begin{lemma}[pexpr-inf-typ]
\label{lemma:union:pexpr-inf-typ}
If \ $[[G |- p <= A]]$ \ then:
  \begin{enumerate}
  \item $\exists$ $[[B]]$, \ $[[B <: A]]$
  \item and \ $[[G |- p => B]]$
  \end{enumerate}
\end{lemma}
\end{comment}

\begin{lemma}[Determinism]
\label{lemma:union:determinism}
  If \ $[[G |- e dirflag A]]$
  \begin{enumerate}
  \item and \ $[[e --> e1]]$
  \item and \ $[[e --> e2]]$ \\
  then $[[e1]]$ = $[[e2]]$.
  \end{enumerate}
\end{lemma}

\begin{comment}
\begin{proof}
  By induction on first reduction relation and inverting second reduction relation subsequently.
  All cases are trivial to solve by simple inversions except:
  \begin{itemize}
    \item Case \rref{typ-typeof} requires \cref{lemma:union:check-both-disj-false}.
  \end{itemize}
\end{proof}

\begin{lemma}[check-both-disj-false]
\label{lemma:union:check-both-disj-false}
If \ $[[A *s B]]$ \ and \ $[[G |- p <= A]]$ \ and \ $[[G |- p <= B]]$ \ then \ False.
\end{lemma}
\end{comment}

\baber{Again, should we show the property that a term cannot be checked against two disjoint types?}


%%%%%%%%%%%%%%%%%%%%%%%%%%%%%%
%% Discussion on Disjointness
%%%%%%%%%%%%%%%%%%%%%%%%%%%%%%
\subsection{Discussion on Disjointness}
\label{sec:union:discussion}
Interaction of intersection and union types is a well-known problem in programming languages.
In this section we discuss limited expressive power of our current disjoint specifications
(\cref{def:union:disj}) and propose a robust and scalable definition for disjointness.
\Cref{def:union:disj} works fine in simple calculus as of now. But it adds complexities as soon as we add
intersection types to make the system more expressive.

\paragraph{Updated Disjointness}
An updated and robust declerative disjointness is shown in \Cref{fig:union:ord} as \Cref{def:union:disj1}.
It is important to establish the fact that \Cref{def:union:disj1} works as expected.
We prove equivalence of \Cref{def:union:disj} and \Cref{def:union:disj1} to establish this fact.
Examples illustrated in \Cref{sec:union:disj} also work fine with the updated disjointness definition.
\Cref{def:union:disj1} will be discussed in detail in \Cref{sec:inter} along with corresponding
algorithmic version.

\begin{lemma}[Disjointness Equivalence]
A $*_{s}$ B $\longleftrightarrow$ A $*_{s1}$ B.
\end{lemma}

\begin{figure}
    \centering
    \drules[ord]{$[[ordinary A]]$}{Ordinary Types}{top, int, arrow}
  \medskip
  \begin{definition}
    \centering
    A $*_{s1}$ B $\Coloneqq$ $\forall$ C, $[[ordinary C]]$ $\rightarrow$ $\neg$ ($[[C <: A]]$ $\wedge$ $[[C <: B]]$)
    \label{def:union:disj1}
  \end{definition}
  \caption{Updated disjointness and ordinary types for \cal.
    \snow{I think there is no need to make figure small
    unless we are going to exceed the page limit.} \baber{Sure, I fixed that.}}
  \label{fig:union:ord}
\end{figure}

\section{\cal with Intersection Types: The Challenge of Disjointness}
\label{sec:inter}

In this section we add intersection types to \cal.
%The study of an
%extension of \cal with intersection types is motivated by the fact
%that most languages support both union and intersection types.
%Therefore it is important to understand whether intersection types
%can be easily added or whether there are some challenges.
The addition of intersection types poses a challenge,
since the notion of disjointness inspired from \citet{oliveira2016disjoint}
no longer works. Fortunately, the alternative specification presented in
\Cref{sec:union:discussion} comes to the rescue. Furthermore, we show how to obtain
an algorithmic formulation of disjointness based on a novel notion
called \emph{lowest ordinary subtypes}. All the properties, including
type soundness and determinism are preserved in the extension of \cal
with intersection types.

%%%%%%%%%%%%%%%%%%%%%%%
%% Syntax and Semantics
%%%%%%%%%%%%%%%%%%%%%%%
\subsection{Syntax, Subtyping and Ordinary Types}
\label{sec:inter:system}
The syntax and subtyping for this section mostly follows
from \Cref{sec:union}.  The additional types and subtyping rules are shown in
\Cref{fig:inter:system}.
%This system can trivially be extended with
%more primitive types. We also have $Bool$ and $String$ types in our
%Coq formalization.
The most significant difference and novelty in this section
is the addition of intersection types $[[A/\ B]]$.
Expressions $[[e]]$, pre-values $[[p]]$, annotated values $[[w]]$,
values $[[v]]$, and context $[[G]]$ stay the same as in \Cref{sec:union}.

\paragraph{Subtyping}
The new rules for subtyping are shown in middle part of
\Cref{fig:inter:system}.  \Rref{s-anda,s-andb,s-andc} are for newly
added intersection types. \Rref{s-anda} states that a type $[[A]]$ is
a subtype of intersection of two types $[[B]]$ and $[[C]]$ only if
$[[A]]$ is a subtype of both $[[B]]$ and $[[C]]$. \Rref{s-andb,s-andc} 
state that an intersection type $[[A /\ B]]$ is a subtype
of some type $[[C]]$ if either of its component types ($[[A]]$ or
$[[B]]$) is a subtype of $[[C]]$. The subtyping relation is reflexive
and transitive.

\begin{figure}[t]
    \centering
    \begin{tabular}{lrcl} \toprule
      Types & $[[A]], [[B]]$ & $\Coloneqq$ & $ ... \mid [[A /\ B]] $ \\
      \bottomrule
    \end{tabular}
    \centering
    \drules[s]{$ [[A <: B ]] $}{Additional Subtyping Rules}{anda, andb, andc}
    \centering
    \drules[ord]{$[[ordinary A]]$}{Ordinary Types}{int, arrow}
%    \centering
%    {\renewcommand{\arraystretch}{1.2}
%    \begin{tabular}{|lcl|}
%      \multicolumn{3}{c}{Lowest Ordinary Subtypes (LOS) $[[findsubtypes A]]$} \\
%      \hline
%     $[[findsubtypes Top]]$ & = & \{$ [[Int]], [[Top -> Bot]]$\}  \\
%     $[[findsubtypes Bot]]$ & = & \{\}  \\
%     $[[findsubtypes Int]]$ & = & \{$ [[Int]] $\}  \\
%     $[[findsubtypes A -> B]]$ & = & \{$ [[Top -> Bot]] $\}  \\
%     $[[findsubtypes A \/ B]]$ & = & $ [[findsubtypes A]] \cup [[findsubtypes B]] $\\
%     $[[findsubtypes A /\ B]]$ & = & $ [[findsubtypes A]] \cap [[findsubtypes B]] $\\
%      \hline
%    \end{tabular} }
  \caption{Additional Types, Subtyping and Ordinary Types for \cal with intersection types.}
  \label{fig:inter:system}
\end{figure}

\paragraph{Ordinary Types}
%Declarative disjointness is now defined in terms of ordinary types
%instead of \emph{bottom-like} types.
Lower part of
\Cref{fig:inter:system} shows ordinary types~\cite{davies2000intersection}, 
which is the same as given in \Cref{sec:union:discussion}.
Ordinary types are essentially primitive types. In \cal they
are essentially integers and functions. 
%$[[Top]]$, $[[Int]]$ and $[[A -> B]]$ types are considered as ordinary
%types in \cal with intersection types.

%%%%%%%%%%%%%%%%%%%%%%%
%% Disjointness
%%%%%%%%%%%%%%%%%%%%%%%
\subsection{Disjointness Specification}
\label{sec:inter:disj}
Disjointness is the most interesting aspect of the extension of \cal with
intersection types. Unfortunatelly, \Cref{def:union:disj} does not work with intersection
types. In what follows, we first explain why \Cref{def:union:disj} does not work, and then
we show how to introduce disjointness in the presence of intersection types. 

%We employ updated disjointness for \cal with intersection types as
%discussed in \Cref{sec:union:discussion} and discuss the updated
%disjointness in detail in this section.

\paragraph{Bottom-like Types, Intersection Types and Disjointness}
\noindent %Bottom-like types behave like $[[Bot]]$ type and are discussed in \Cref{sec:union:disj}.
Recall that disjointness in \Cref{sec:union} (\Cref{def:union:disj}) depends
upon bottom-like types, where two types are disjoint only if all their common
subtypes are bottom-like. However, this definition no longer works with the
addition of intersection types. Specifically, according to the subtyping rule
for intersection types, any two types have their intersection as one common subtype.
For non-bottom-like types, their intersection is also not bottom-like. For
example, type $[[Int]]$ and type $[[Bool]]$ now has a subtype $[[Int /\ Bool]]$,
which is not bottom-like. Therefore, the disjointness definition fails, since we
can always find a common non-bottom-like subtype for any two (non-bottom-like)
types.

\begin{comment}
Reader may think at this point to add intersection of non-overlapping types such as $[[Int /\ Bool]]$
in bottom-like types to solve the problem. A trivial and intuitive approach to think of is:

\begin{center}
\drule[]{bl-andsub}
\end{center}

\noindent \Rref{bl-andsub} states that if two types $[[A]]$ and $[[B]]$
are not subtypes of each other (i.e. non-overlapping) then intersection of
such types $[[A /\ B]]$ is bottom-like.
\Rref{bl-andsub} works for simple cases such as $[[Int]]$ and $[[Bool]]$.
But it fails if  $[[A]]$ = $[[Int /\ Bool]]$ and 
$[[B]]$ = $[[Int /\ Bool]]$.
Because $[[A]]$ ($[[Int /\ Bool]]$) and $[[B]]$ ($[[Int /\ Bool]]$) are subtypes
of each other and are not bottom-like as per \rref{bl-andsub}.
So, naive addition of \rref{bl-andsub} skips potential bottem-like types.
Another alternative may be:

\begin{center}
\drule[]{bl-anddisj}
\end{center}

\noindent \Rref{bl-anddisj} states that if two types $[[A]]$ and $[[B]]$ are disjoint,
then intersection of such types $[[A /\ B]]$ is bottom-like.
But \rref{bl-anddisj} imposes additional complexities of mutually
dependent definitions among disjointness and bottom-like.
This makes completeness challenging or even impossible to prove.
\end{comment}
\paragraph{A possible solution: the Ceylon approach}
A possible solution for this issue is to add a subtyping rule which makes intersection of
disjoint types to be subtype of $[[Bot]]$.

\begin{center}
\drule[]{s-disj}
\end{center}
%\bruno{Please use disjointness spec instead of disjoitness algorithm in the rule.}

\noindent This rule is adopted by the Ceylon language~\cite{muehlboeck2018empowering}:
\Rref{s-disj} states that the insersection type $[[A /\ B]]$
of two disjoint types $[[A]]$ and $[[B]]$ is subtype of $[[Bot]]$ type.
In other words, now the type $[[Int /\ Bool]]$ would be a bottom-like type, and the
definition of disjointness used in Section~\ref{sec:union} could still work.
However we do not adopt this solution here for two reasons:

\begin{enumerate}

\item \Rref{s-disj} complicates the system because
  it adds a mutual dependency between subtyping and disjointness:
  disjointness is defined in terms of subtyping, and subtyping now
  uses disjointness as well in \rref{s-disj}. This creates important
  challenges for the metatheory. In particular, the completeness proof
  for disjointness becomes quite challenging.

\item Additionaly, the assumption that intersections of disjoint types
  is equivalent to bottom is too strong for some calculi with intersection
  types. If a merge operator~\cite{reynolds1988preliminary} is allowed in the calculus, 
  intersection types
  with disjoint types can be inhabited with values (for example, in 
  \cite{muehlboeck2018empowering} \baber{Is this right citation?},
  the type $[[Int /\ Bool]]$ is inhabited by $[[1]] ,, \mathsf{True}$), and considering those
  types bottom-like would lead to a problematic definition of disjointness and the loss
  of determinism.

  \ningning{What is a problematic definition of disjointness and loss of
    determinism? What I can think why it's wrong is that the bottom type is
    then made inhabited.}

\end{enumerate}

Instead, we adopt a novel solution in \name.

\paragraph{Disjointness based on Ordinary Types to the rescue}
To solve the problem with the disjointness specification, we resort to
the alternative definition of disjointness presented in \Cref{sec:union:discussion}.

\begin{definition}
\label{def:inter:disj}
  A $*_s$ B $\Coloneqq$ $\forall$ C, \ $[[ordinary C]]$ \ $\rightarrow$ \ $\neg$ \ ($[[C <: A]]$ and $[[C <: B]]$).
\end{definition}

Interestingly, while in Section~\ref{sec:union} such definition was
equivalent to the definition using bottom-like types, this is no
longer the case for \name with intersection types. To see why,
consider again the types $[[Int]]$ and $[[Bool]]$.  $[[Int]]$ and
$[[Bool]]$ do not share any common ordinary subtype. Therefore,
$[[Int]]$ and $[[Bool]]$ are disjoint types according to
\Cref{def:inter:disj}.
\begin{comment}
We extend our previous example of type $[[Int]]$ and type $[[Bool]]$ and show how
disjointness based upon ordinary types categorize them as disjoint types.
An important observation at this point is common subtypes of type $[[Int]]$ and type $[[Bool]]$
cannot include either $[[Int]]$ or $[[Bool]]$. Problematic types are the intersection types
such as $[[Int /\ Bool]]$. We empahsize the point that ordinary types in \cal does not contain
intersection types. Further, all ordinary types are non-overlapping in \cal.
Therefore, now we say that
two types are disjoint if they do not have any common ordinary subtype. $[[Int]]$ and $[[Bool]]$
do not share any common ordinary subtype. Therefore, $[[Int]]$ and $[[Bool]]$ are disjoint types.
\Cref{def:inter:disj} shows the declarative disjointness for \cal with intersection types:


\noindent Two types $[[A]]$ and $[[B]]$ are
disjoint if the two types $[[A /\ B]]$ do
not have any common ordinary subtype. For example, $[[Int]]$ and $[[A -> B]]$
are disjoint types because there is no ordinary type that is a subtype
of both types ($[[Int]]$ and $[[A -> B]]$).
\Cref{def:inter:disj} is the same as 
\Cref{def:union:disj1}. However, while the \Cref{def:union:disj1} in \Cref{sec:union:discussion}
is equivalent the definition of disjointness using bottom-like types (\Cref{def:union:disj}),
in the calculus with intersection types that is no longer the case.
\end{comment}
We further illustrate  
\Cref{def:inter:disj} with a few concrete examples:

\begin{enumerate}
  \item $\boldsymbol{A = [[Int]], \ B = \ [[Int -> Bool]]:}$ \\
        Since there is no ordinary type that is subtype of both $[[Int]]$ and $[[Int -> Bool]]$,
        the two types are disjoint.
  \item $\boldsymbol{A = [[Int \/ Bool]], \ B = \ [[Bot]]:}$ \\
    Since there is no ordinary type that is subtype of both $[[Int \/ Bool]]$ and $[[Bot]]$,
    $[[Int \/ Bool]]$ and $[[Bot]]$ are disjoint types.
    In general, $[[Bot]]$ type is disjoint to all types because $[[Bot]]$ type does not
    have any ordinary subtype.
  \item $\boldsymbol{A = [[Int /\ (Int -> Bool)]], \ B = \ [[Int]]:}$ \\
        There is no ordinary type that is subtype of both $[[Int /\ (Int -> Bool)]]$ and $[[Int]]$.
        Therefore, $[[Int /\ A -> B]]$ and $[[Int]]$ are disjoint types.
        In general, an intersection of two disjoint types ($[[Int /\ A -> B]]$ in this case)
        is always disjoint to all types.
  \item $\boldsymbol{A = [[Int /\ Bool]], \ B = \ [[Int \/ Bool]]:}$ \\
        There is no ordinary type that is subtype of both $[[Int /\ Bool]]$ and $[[Int \/ Bool]]$.
        Therefore, $[[Int /\ Bool]]$ and $[[Int \/ Bool]]$ are disjoint types.
        This follows from the description in example 3 that type $[[Int /\ Bool]]$ will always be
        disjoint to all types.
  \item $\boldsymbol{A = [[Int]], \ B = \ [[Top]]:}$ \\
        In this case, $[[Int]]$ and $[[Top]]$ share a common ordinary subtype which is $[[Int]]$.
        Therefore, $[[Int]]$ and $[[Top]]$ are not disjoint types.
\end{enumerate}

\ningning{like before, add one or two for unions (like one with both unions and intersections).}
\baber{I added example 4 and updated example 2 from $[[Int]]$ to $[[Int \/ Bool]]$.}
%The
%reason we did not write intersection in \Cref{def:union:disj1} is
%because we do not have intersection types in \cal discussed in
%\Cref{sec:union}.
%Updated disjointness specifications are now
%represented as $[[A *s B]]$ and not $[[A]]$ $*_{s1}$ $[[B]]$.

\subsection{Algorithmic Disjointness}

The change in the disjointness specification has a significant impact on its
algorithmic formulation. In particular, it is not obvious at all how to adapt
the algorithmic formulation as in \Cref{fig:union:disj-typ}. To obtain a
sound, complete and decidable formulation of disjointness, we employ the novel
notion of \emph{lowest ordinary subtypes}.

\ningning{separate the definition of LOS into another figure and put the figure here.}
\baber{Done.}

\begin{figure}[t]
    \centering
    {\renewcommand{\arraystretch}{1.2}
    \begin{tabular}{|lcl|}
      \multicolumn{3}{c}{Lowest Ordinary Subtypes (LOS) $[[findsubtypes A]]$} \\
      \hline
     $[[findsubtypes Top]]$ & = & $\{ [[Int]], [[Top -> Bot]]\}$  \\
     $[[findsubtypes Bot]]$ & = & $\{\}$  \\
     $[[findsubtypes Int]]$ & = & $\{ [[Int]] \}$  \\
     $[[findsubtypes A -> B]]$ & = & $\{ [[Top -> Bot]] \}$  \\
     $[[findsubtypes A \/ B]]$ & = & $ [[findsubtypes A]] \cup [[findsubtypes B]] $\\
     $[[findsubtypes A /\ B]]$ & = & $ [[findsubtypes A]] \cap [[findsubtypes B]] $\\
      \hline
    \end{tabular} }
  \caption{Lowest Ordinary Subtypes function for \cal with intersection types.}
  \label{fig:inter:los}
\end{figure}


\paragraph{Lowest Ordinary Subtypes ($[[findsubtypes A]]$)}
\Cref{fig:inter:los} shows the definition of
\emph{lowest ordinary subtypes} (LOS) ($[[findsubtypes A]]$).
LOS are defined as a function which
returns a set of ordinary subtypes of the given input type. 
No ordinary type is subtype of $[[Bot]]$ type. The only ordinary
subtype of $[[Int]]$ is $[[Int]]$ itself. The function case is
interesting. Since no two functions are disjoint in the calculus
proposed in this paper, the case for function types $[[A -> B]]$ returns $[[Top
    -> Bot]]$. This type is the least ordinary function type, which is a subtype
of all function types.
Ordinary
subtypes of $[[Top]]$ are $[[Int]]$ and $[[Top -> Bot]]$.
In the case of union types $[[A \/ B]]$, the
algorithm is collects the LOS of $[[A]]$ and $[[B]]$ and returns the union of the
two sets. For intersection types $[[A
    /\ B]]$ the algorithm collects the LOS of $[[A]]$ and $[[B]]$
and returns the intersection of the two sets.
Note that LOS is defined as a structurally recursive function and therefore
its decidability is immediate.

\paragraph{Algorithmic Disjointness}
With LOS, it is straightforward to give an algorithmic formulation of
disjointness:
%Since we gave a new definition for declarative disjointness.
%Therefore, we define a new algorithm for disjointness as shown in \Cref{def:inter:ad}.

\begin{definition}
\label{def:inter:ad}
%\texttt{`abc \textasciigrave abc}
  A $*_a$ B $\Coloneqq$  $ [[findsubtypes A]] \cap [[findsubtypes B]] $ = $\{\}$.
\end{definition}

\noindent The algorithmic formulation of disjointness in
\Cref{def:inter:ad} states that two
types $[[A]]$ and $[[B]]$ are disjoint
if they do not have any common lowest ordinary subtypes. That is the
set intersection of $[[findsubtypes A]]$ and $[[findsubtypes B]]$ is the empty set.
%In simple words,
%two types $[[A]]$ and $[[B]]$ are disjoint if they do not share any common ordinary subtype because
%$FindSubTypes$ ($[[findsubtypes A]]$) returns a set of ordinary subtypes.
Note that this algorithm is naturally very close to \Cref{def:inter:disj}.
\begin{comment}
We illustrate \Cref{def:inter:ad} with a few examples:

\begin{enumerate}
  \item $\boldsymbol{A = [[Int]], \ B = \ [[A -> B]]:}$ \\
        $[[findsubtypes Int]]$ returns \{$[[Int]]$\} and $[[findsubtypes A -> B]]$ returns
        \{$[[Top -> Bot]]$\}. Set intersection of \{$[[Int]]$\} and \{$[[Top -> Bot]]$\} is
        empty set \{\}. Therefore, $[[Int]]$ and $[[A -> B]]$ are disjoint types.
  \item $\boldsymbol{A = [[Int]], \ B = \ [[Bot]]:}$ \\
        $[[findsubtypes Int]]$ returns \{$[[Int]]$\} and $[[findsubtypes Bot]]$ returns
        \{\}. Set intersection of \{$[[Int]]$\} and \{\} is
        empty set \{\}. Therefore, $[[Int]]$ and $[[Bot]]$ are disjoint types.
        In general, type $[[Bot]]$ is disjoint to all types because $[[findsubtypes Bot]]$
        will always return \{\} and intersection of \{\} with all other sets is \{\}.
  \item $\boldsymbol{A = [[Int /\ A -> B]], \ B = \ [[Int]]:}$ \\
        $[[findsubtypes Int /\ A -> B]]$ returns \{\} and $[[findsubtypes Int]]$ returns
        \{$[[Int]]$\}. Set intersection of \{\} and \{$[[Int]]$\} is
        empty set \{\}. Therefore, $[[Int /\ A -> B]]$ and $[[Int]]$ are disjoint types.
        In general, intersection type of two disjoint types which is $[[Int /\ A -> B]]$ in this case,
        is always disjoint to all types.
  \item $\boldsymbol{A = [[Int]], \ B = \ [[Top]]:}$ \\
        $[[findsubtypes Int]]$ returns \{$[[Int]]$\} and $[[findsubtypes Top]]$ returns
        \{$[[Int]]$, $[[Top -> Bot]]$\}.
        Set intersection of \{$[[Int]]$\} and \{$[[Int]]$, $[[Top -> Bot]]$\} is
        set \{$[[Int]]$\}. Therefore, $[[Int]]$ and $[[Top]]$ are not disjoint types.
\end{enumerate}
\end{comment}

\paragraph{Soundness and Completeness}

We prove that the algorithmic disjointness is sound and complete with respect to
the declarative specification.

\begin{lemma}[Disjointness Soundness]
  If \ $[[A * B]]$ \ then \ $[[A *s B]]$.
\label{lemma:inter:disj-sound}
\end{lemma}

\begin{lemma}[Disjointness Completeness]
  If \ $[[A *s B]]$ \ then \ $[[A * B]]$.
\label{lemma:inter:disj-complete}
\end{lemma}

\baber{I think we may want to show the proofs for soundness and completeness for this section?}
\bruno{Perhaps, if the proofs are simple enough (or we have space) we can discuss}

\subsection{Typing, Semantics and Metatheory}

The typing and operational semantics stay the same as in \Cref{sec:union}. Note
that while those rules stay the same, as types and subtyping rules are extended
with intersection types, the system is naturally more expressive than
in \Cref{sec:union}.

We prove that \cal with intersection types preserves type soundness and determinism.
%\bruno{add determinism}

\begin{lemma}[Type Preservation]
\label{lemma:inter:preservation}
  If \ $[[G |- e dirflag A]]$ and $[[e --> e']]$ then $[[G |- e' dirflag A]]$.
\end{lemma}

\begin{lemma}[Progress]
\label{lemma:inter:progress}
If \ $[[ [] |- e dirflag A]]$ then
 \begin{enumerate}
  \item either $[[e]]$ is a value.
  \item or $[[e]]$ can take a step to $[[e']]$.
  \end{enumerate}
\end{lemma}

\begin{theorem}[Determinism]
\label{lemma:inter:determinism}
  If \ $[[G |- e dirflag A]]$ and \ $[[e --> e1]]$ and \ $[[e --> e2]]$ then $[[e1]]$ = $[[e2]]$.
\end{theorem}

\begin{comment}
\begin{lemma}[Substitution]
\label{lemma:inter:substitution}
  If \ $[[G, x:B , G1 |- e dirflag A]]$ \ and \ $[[G |- e' => B]]$
  then \ $[[G, G1 |- e [ x ~> e' ] dirflag A]]$
\end{lemma}
\end{comment}

%%% Local Variables:
%%% mode: latex
%%% TeX-master: "../paper"
%%% org-ref-default-bibliography: "../paper.bib"
%%% End:
\section{Discussion}
\label{sec:discussion}
In this section we ...

\subsection{Refactoring Subtyping}
\label{sec:inter:refactoring}

\Rref{s-disj} is a significant and novel addition in subtyping of \cal 
with intersection types. It primarily
makes intersection of disjoint types subtype of all types.
Because two types are disjoint only if they do not share any common ordinary subtype.

\Rref{s-disj} is an interesting addition in subtyping of \cal with intersection types.
It says that if $FindSubTypes$ operation returns empty set against some type $[[A]]$, then $[[A]]$
is subtype of all types. In other words, such type behaves like \emph{bottom-like} type.
One particularly interesting case of \rref{s-disj} is when $[[A]]$ is an intersection type of two
disjoint types such as $[[Int /\ A -> B]]$. Since $[[findsubtypes Int /\ A -> B]]$ returns empty set,
therefor \rref{s-disj} makes $[[Int /\ A -> B]]$ subtype of all types.

\paragraph{Dropping Subtyping Rule S-BOT}
Another particular interesting case of \rref{s-disj} is that it makes \rref{s-bot} redundant 
and \rref{s-bot} can safely be dropped without any significant
impact on metatheory and implementation. It is trivial to prove a lemma which says $[[Bot]]$ type is
subtype of all types. We drop \rref{s-bot} from the calculus discussed in this section
and prove \Cref{lemma:inter:bls} to show this property instead. It is because that $[[findsubtypes Bot]]$
returns empty set \{\} and \rref{s-disj} makes all such types with $[[findsubtypes Bot]]$ = \{\} as
subtype of all types.

\begin{lemma}[Bottom Type Least Subtype]
  $[[Bot <: A]]$.
\label{lemma:inter:bls}
\end{lemma}
\section{Related Work}
\label{sec:related}

\bruno{Generally speaking I feel that either here, or in the overview, we
  should actually show concrete examples of other union elimination constructs,
in addition to having some text describing such constructs.}

\begin{itemize}
	\item {Union Types    (discussing work on union types and elimination constructs for union types) <-- Need to coordinate with Ningning/Snow so that there’s not much overlap there. If the Overview discusses some of these in detail, the RW does not need to have much text in it.}
	\item {Overloading (more general work on overloading, including type classes, approaches to overloading based on intersection types, etc…)}
	\item{Disjointness and Disjoint Intersection Types}
\end{itemize}

\paragraph{Union Types}
\begin{comment}
Set-theoretic unions have sound theory and extensively studied in
mathematics. Set-theoretic unions correspond to union types or
disjoint union types in programming languages. Disjoint union types
are also called sum types or variants.  Constructors are explicitly
labeled in disjoint union types and expressions are manipulated using
corresponding labels. Few other interesting calculi (and this paper)
do not use labels and provide type-based union elimination.
%Developing
%a sound and deterministic union elimination construct for type-based
%union elimination have been a challenge in research community.
\end{comment}
Union types were first studied in the setting of programming languages
by \citet{macqueen1984ideal}. Later on, union types have been studied
in the literature in various contexts. \citep{pierce1991programming}
studied union types with intersection types and polymorphism. But he
did not define dynamic semantics for his calculus. Union elimination
is same in \citep{macqueen1984ideal} and \citep{pierce1991programming}
and is not deterministic\bruno{We need to be quite careful when making claims like this.
  For Pierce, since he has not defined the dynamic semantics you cannot say this. For MacQueen
  where can I find his semantics for union elimination?}.  \citet{barbanera1995intersection} proposed
two approaches for subject reduction in a calculus with union
types. However, their union elimination is same as last two and follow
unrestricted union elimination. \bruno{So you are arguying it is non-deterministic?
Again where can I find the semantics? Can you paste this on slack?}

\citet{freeman1991refinement} studied union types in setting of
refinement types for ML\bruno{and ... Do they have an elimination construct for unions?
  If so how is that related to our work?}.  \citet{hosoya2003xduce} studied union types
in XDuce programming language. XDuce offers a novel feature of
so-called regular expression types. Pattern matching can be on
expressions and types in XDuce.  Expressions are considered as special
cases of types. Although XDuce offers type-based switch construct,
it does not ensure disjointness. Thus cases can overlap and
reordering the cases may change the semantics of the program. CDuce
\cite{benzaken2003cduce} is an extension of XDuce. CDuce has improved
pattern matching but it also does not have disjoitness constraint and
follows firt-match policy.  \citet{fallside2001xml} studied union
types in markup language, but with a restriction of disjoint top-level
labels which is different from our work because we propose
disjointness on types.\bruno{In other words, for the last work, the unions
are labelled (i.e. the cases are not based on types)?}

\citet{dunfield2014elaborating} also studied intersection and union
types. Her approach is to elaborate a source language with union types into a
target language with (tagged) sum types. Our
work is different from her work because she used unrestricted union
elimination\bruno{You need to do better than this. I don't think the elimination
construct is unrestricted. You need to explain what that construct allows and disallows.}
in source language and sum types with explicit injectors
in target language.  \citet{castagna2017gradual} studied
set-theoretic union types in gradual typing setting\bruno{Castagna did alot of work
  on union types before this one, so I'm not sure why we are mentioning only this work
  of Castagna, instead of talking about previous work where the ideas were introduced.}. They
give a dynamic type-based cast for union elimination. But they do not
have disjointness constraint on case branches and they check the type
only for first case branch\bruno{Not a very good explanation: you get negated types for the branches}.
Recently, \citet{muehlboeck2018empowering} gave a general framework of a calculi
with intersection and union types. They illustrated the significane of
their framework using Ceylon programming language.  They also studied
disjointness in type-based case expressions.\bruno{Very very weak: this makes it sound
  like that they already did something like our work: they did not. Their work was about
  subtyping relations with unions and intersections and distributivity rules. They briefly mentioned
  the switch construct of Ceylon and disjointness. But they have not studied the formal semantics
of such construct or formally defined disjointness.}

\paragraph{Occurrence Typing}
\bruno{This paragraph is very weak: it gives an incorrect explanation of occurrence typing, and it is not very comprehensive:
  it cover the two works I pointed out (for starters), but nothing else. We may not need to cover much more, but
  we need to make sure that we read some other work on occurrence typing to ensure that we are not missing something that is
closely related.}
Occurence typing or flow typing \citet{tobin2008design} specialize or refine the type of variable in different
case branches. Occurence typing is limmited to have variables in switch expression, whereas \cal
can have any arbitrary expression in switch expression\bruno{Do not confuse occurrence typing with a switch expression.
  Occurrence typing is a technique that can be used when modelling case expressions for union types}.
Also, occurence typing comes with an
unrestricted typing construct for case expression\bruno{Again you're confusing things: occurrence typing != case expression:
you could have our switch expression with occurrence typing,  but we chose not to.}. \citet{castagna2019revisiting} recently studied
occurence typing for set-theoretic types and allowed to have generic expressions in switch expression.

\paragraph{Disjoint Intersection Types}
\citet{pottinger1980type} and \citet{coppo1981functional} initially
studied intersection types in programming languages
literature. Forsythe~\cite{reynolds1988preliminary} is the first 
programming language to have intersection types, but it did not
have union types.  Disjoint intersection types were first
studied by \citet{oliveira2016disjoint} in the $\lambda_{i}$ calculus
to give a coherent calculus for intersection types with a merge
operator. The notion of disjointness used in \cal, discussed in \Cref{sec:union},
is inspired by the notion of disjointness of $\lambda_{i}$. In essence in
disjointness in \cal is the dual notion: while in $\lambda_{i}$ two types
are disjoint if they only have top-like supertypes, in \cal two types
are disjoint if they only have bottom-like supertypes.

None of calculi with disjoint intersection types~\cite{} in the literature
includes union types. One interesting discovery of our work is that the
presence of both intersections and unions in a calculus can affect disjointness.
In particular, as we have seen in Section~\ref{sec:inter}, adding intersection types
required us to change disjointness. The notion of disjointness that was
derived from $\lambda_{i}$ stops working in the presence of intersection types.
Interestingly, a similar issue happens when union types are added to
a calculus with disjoint intersection types. If disjointness of two types $A$
and $B$ is defined to be that such types can only have top-like types,
then adding union types immediately breaks such definition.
For example, the types $[[Int]]$ and $[[Bool]]$ are disjoint but, with union
types, $[[Int \/ Bool]]$ is a common supertype that is not top-like.
We conjecture that, to add union types to disjoint intersection types,
we can use the following definition of disjointness:

\begin{definition}
\label{def:inter:disj}
  A $*_s$ B $\Coloneqq$ $\forall$ C, \ $[[ordinary C]]$ \ $\rightarrow$ \ $\neg$ \ ($[[A <: C]]$ and $[[B <: C]]$).
\end{definition}

\noindent which is, in essence, the dual notion of the definition presented in
Section~\ref{sec:inter}. Under this definition $[[Int]]$ and $[[Bool]]$ would
be disjoint since we cannot find an common ordinary supertype (and $[[Int \/ Bool]]$
is a supertype, but it is not ordinary). Furthermore, there should be a
dual notion to LOS, capturing the greatest ordinary supertypes. Moreover,
if a calculus includes both disjoint switches and a merge operator,
then the two notions of disjointness must coexist in the calculus. 
This will be an interesting path of exploration for future work.

\paragraph{Overloading}
Union types also provide a kind of function overloading or ad-hoc
polymorphism using the switch and type-based case analysis. Programmer
may define the argument type to be union type of certain types. By
type-based case analysis programmer may choose to execute diffent code
for each specific type of input.  Intersection types have also been
studied for function overloading. For example a function with type
$[[Int -> Int /\ Bool -> Bool]]$ can take input values either of type
$[[Int]]$ or $[[Bool]]$.  It returns either $[[Int]]$ or $[[Bool]]$
depending upon the input type.  Function overloading has been studied
in detail in the literature by \cite{castagna1995calculus},
\cite{cardelli1985understanding}, \cite{stuckey2005theory} among
others.  \citet{wadler1989make} studied type classes for the
overloading of arithmetic operators.

\section{Conclusion and Future Work}
\label{sec:conclusion}

We develop a union calculus (\cal) for type-based union elimination. 
We also add intersection types in
\cal. The calculus is proved to be sound, complete, type-safe and deterministic.
Such calculi have a variety of practical applications such as in switch expressions
and function overloading. We plan to extend \cal for practical programming languages.

A major future research direction for \cal is to study it with merge operator.
Intersection types are of significant interest with merge operator. But it is non-trivial
to naivly add the merge operator for a calculus with union and intersection types.
One major problem with naive merge operator is that merge operator induces (in)coherence problem.
Therefore, special care has to be taken to study \cal with merge operator.

\bibliography{paper}


\end{document}
