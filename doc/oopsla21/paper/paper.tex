\documentclass[UKenglish]{acmart}

%\usepackage{geometry}
%\geometry{left=2.5cm,right=2.5cm,top=2.5cm,bottom=2.5cm}

% Basics
%\usepackage{fixltx2e}
%\usepackage{url}
%\usepackage{fancyvrb}
%\usepackage{mdwlist}  % Miscellaneous list-related commands
\usepackage{xspace}   % Smart spacing
%\usepackage{supertabular}


% https://www.nesono.com/?q=book/export/html/347
% Package for inserting TODO statements in nice colorful boxes - so that you
% won’t forget to fix/remove them. To add a todo statement, use something like
% \todo{Find better wording here}.
%\usepackage{todonotes}

% for line numbers
\usepackage{lineno}
\linenumbers

%for backtick `
\usepackage{textcomp}

%% Math
%\usepackage{amsmath}
\usepackage{amsthm}
%\usepackage{amssymb}
%\usepackage{bm}       % Bold symbols in maths mode

%% Theoretical computer science
\usepackage{stmaryrd}
\usepackage{mathtools}  % For "::=" ( \Coloneqq )

%% Code listings
\usepackage{listings}
\usepackage{xcolor}

%for cref -- celever referencing
\usepackage{cleveref}

\usepackage{amsbsy}

\definecolor{codegreen}{rgb}{0,0.6,0}
\definecolor{codegray}{rgb}{0.5,0.5,0.5}
\definecolor{codepurple}{rgb}{0.58,0,0.82}
\definecolor{backcolour}{rgb}{0.95,0.95,0.92}

\lstdefinestyle{mystyle}{
    %backgroundcolor=\color{backcolour},   
    commentstyle=\color{codegreen},
    keywordstyle=\color{magenta},
    numberstyle=\tiny\color{codegray},
    stringstyle=\color{codepurple},
    basicstyle=\ttfamily\footnotesize,
    breakatwhitespace=false,         
    breaklines=true,                 
    captionpos=b,                    
    keepspaces=true,                 
    numbers=left,                    
    numbersep=5pt,                  
    showspaces=false,                
    showstringspaces=false,
    showtabs=false,                  
    tabsize=2
}
\lstset{style=mystyle}

%% Font
\usepackage[euler-digits,euler-hat-accent]{eulervm}


\usepackage{ottalt}

\renewcommand\ottaltinferrule[4]{
  \inferrule*[right={#1}]
    {#3}
    {#4}
}

\newcommand\mynote[3]{\textcolor{#2}{#1: #3}}
\newcommand\bruno[1]{\mynote{Bruno}{red}{#1}}
\newcommand\baber[1]{\mynote{Baber}{blue}{#1}}

\newcommand{\name}{\textsf{Union Types with Disjoint Case Expression}\xspace}
%\newcommand{\dut}{\textsf{Disjoint Union Types}\xspace}
\newcommand{\cal}{$\lambda_{u}$\xspace}
\newcommand{\typeof}{$typeof$\xspace}
\newcommand{\Typeof}{$Typeof$\xspace}

% Ott includes
\inputott{ott-rules}

%% Bibliography style
\bibliographystyle{plainurl}% the mandatory bibstyle

\title{\name}

\author{Neo Morpheus Matrix}

\begin{document}

\begin{abstract}
Type system plays fundamental role to capture nonsensical program expressions at compile time in a
programming language. This often also refers as removing type erros statically at compile time. 
Since the type system defines statics of programs. A more robust type system captures more errors 
at compile time. This gives a guarrantee that certain type errors will not occur at runtime.
With advance types such as intersection types and union types, it has become a challenge to define a 
robust, type-safe, coherent and deterministic type system. One often has to compromise on one 
property to attain another. Union types with pattern matching on types add significant expressive power
in programming language. Function overloading can simply be expressed in a single function with the
help of union types and case analysis on types. Intersection types incorporate many interesting and
advance features that are not easy to implement in classical OOP model.
This study proposes a novel calculus with all aforementioned properties for pattern matching with union types and intersection types. The calculus proposed in this study is named as \cal.
Outline idea in \cal is to allow only non-overlapping or disjoint types in case expressions.
\end{abstract}

\maketitle

%\ottdefnsOrdinary

%\ottdefnsBottomLike

%\ottdefnsDisjointness

%\ottdefnsSubtyping

%\ottdefnsTyping

%\ottdefnsReduction

\section{Introduction}
\label{sec:intro}

\baber{This section will give an introduction.}


%\begin{align*}
%&Isomorphic & A \sim B & ::= [[A <: B]] \wedge [[B <: A]]
%\end{align*}


%\begin{align*}
%&BottomLikeSpec & C & ::= (\forall A ~ B, ~ [[A /\ B]] \sim C \rightarrow \neg ( [[ A <: B ]] ) \wedge \neg ( [[ B <: A ]] )) \vee ([[C <: Bot]])
%\end{align*}

%\begin{align*}
%&DisjSpec & A * B & ::= \forall C, [[C <: A]] \wedge [[C <: B]] \rightarrow  \rfloor [[C]] \lfloor
%\end{align*}

\section{Overview}
\label{sec:overview}

\bruno{I'm sketching a possible structure, but it can be changed ofcourse.}

\subsection{Tagged Union Types}

% Brief intro to union types and one or 2 simple examples that show that we
% can automatically lift values into union type without the need of some
% tag/constructor. Something like:

\ningning{where is syntax highlighting for code? Also probably remove line numbers.}

Union types have been studied intensively in the literature [?]. In particular,
an expression with type $[[A\/B]]$ can be consider to have either type $[[A]]$
\textit{or} type $[[B]]$. Many systems [?] model \textit{tagged union types}
(also called \textit{sum types} or \textit{variants records}),
where explicit \textit{tags} are used to construct terms with union types. For
example, with $\mathsf{inj_1} :: [[A -> A \/ B]]$ and $\mathsf{inj_2} :: [[B ->
A \/ B]]$, we have

\begin{lstlisting}
inj1 "foo": String | Int
inj2 1 : String | Int
\end{lstlisting}

For example, using tagged union types, we can implement safe division as:

\begin{lstlisting}
function safediv (x : Int) (y : Int) : String | Int =
  if (y == 0) then inj1 "Divided by zero"
  else inj2 (x / y)
\end{lstlisting}

Tagged union types are eliminated by \lstinline{case} statements. For example,
we can write the following program, which has different behaviors depending on the
tag of \lstinline{x}, where \lstinline{show} takes an \lstinline{Int} and
returns back its string representation.

\ningning{maybe use \lstinline{switch} for all case analysis?}

\begin{lstlisting}
function tostring (x: String | Int) : String =
  case x of
    inj1 str -> str
    inj2 num -> show num
\end{lstlisting}

Combining union type construction in \lstinline{safediv} and its elimination in
\lstinline{tostring}, we can easily implement an interface which returns the
result of safe division as one \lstinline{String}.

\begin{lstlisting}
> tostring (safediv 42 2)
"21"
> tostring (safe 42 0)
"Divided by zero"
\end{lstlisting}


\subsection{Type-directed Elimination of Union Types}

% Motivate the need for having a construct that can eliminate union types,
% perhaps trying to use an example where the String above would be a kind
% of exception, and the Int would be regular computation. Alternatively
% find an existing example from the literature.

% Discussing existing approaches for eliminating union types;
% point out that elimination constructs based on types (our focus) vs
% elimination constructs based on tags (which are used in algebraic
% datatypes or polymorphic variants (like in OCaml)).
% The focus should be on type-based approaches.

% Try to identify some limitations/problems. For instance how to deal
% with ambiguity (just use order? restrict the construct somehow ...).


While tags are useful to make it explicit which type a value belongs to, they
also add clutter in the programs, and usually require extra space to store the tags. On
the other hand, in systems with subtyping for union types [?], explicit tags are
replaced by implicit coercions represented by the two subtyping rules $[[A <: A \/
B]]$ and $[[B <: A \/ B]]$. We call such types \textit{untagged union types}, or
simply \textit{union types}. In those systems, a term of type $[[A]]$ or $[[B]]$
can be directly used as if it had type $[[A \/ B]]$, and we can model safe
division as

\begin{lstlisting}
function safediv2 (x : Int) (y : Int) : String | Int =
  if (y == 0) then "Divided by zero"
  else (x / y)
\end{lstlisting}

Now elimination of union types cannot reply on tags anymore. Instead, for
union types, elimination is often \textit{type-directed} [?]. For instance,
\lstinline{tostring2} have different behaviors depending on the \textit{type} of
\lstinline{x}.

\begin{lstlisting}
function tostring2 (x: String | Int) : String =
  case x of
    String -> x
    Int -> show x
\end{lstlisting}

\noindent Note here \lstinline{show} is directly applied to \lstinline{x}, as
the type-directed elimination \textit{refines} the type of \lstinline{x} from
\lstinline{String | Int} to \lstinline{Int} in the second branch.

However, compared to tagged union types, extra care must to taken with
elimination of union types. In particular, ambiguity may arise when types in a
union type overlap. For example, consider the type \lstinline{Person | Student},
where we assume \lstinline{Student} is a subtype of \lstinline{Person}. Tagged
union types can easily distinguish the two types by looking at the tag:

\begin{lstlisting}
function isstudent (x: Person | Student) : Bool =
  case x of
    inj1 person -> False
    inj2 student -> True
\end{lstlisting}

But if we apply the straightforward transformation of this function using union
types and type-directed elimination, we will get:

\begin{lstlisting}
function isstudent2 (x: Person | Student) : Bool =
  case x of
    Person  -> False
    Student -> True
\end{lstlisting}

\noindent Now it is unclear what would happen if we apply \lstinline{isstudent2}
to a term of \lstinline{Student}, as the term matches both branches. In some
systems [?], the choice is not decided in the semantics, in the sense either
branch can be chosen, which leads to \textit{incoherent} behaviors. In some
other systems [?], branches are inspected from top to bottom, and the first one
that matches the type gets chosen. However, in those systems, as
\lstinline{Person} is a supertype of \lstinline{Student}, the first branch
subsumes the second one and will always get chosen, and so the second branch
will never get evaluated!


\subsection{Eliminating Union Types in Ceylon}
Ceylon supports type-directed union elimination via the switch/case expression.
It has multiple branches. Each is annotated by a type and contains a function.
Together the overloaded function is applied to a input.
The input has an union type statically.
But its runtime type can only fit in one branch.
That makes it different to the previous discussion. Why?
Because the type annotation of any two branches must be disjoint.

\begin{lstlisting}
  switch (v)
    case (is String) { print("String: ``v``"); }
    case (is Integer) { print("Integer: ``v``"); }
    case (is Float) { print("Float: ``v``"); }
\end{lstlisting}
% https://ceylon-lang.org/documentation/1.3/tour/types/
% case (is Object) { print(v); }
% case (is Nothing) { print("nothing"); }

Being disjoint means it is impossible to find a value that matches with both.
Each of \lstinline{Integer}, \lstinline{Char}, and \lstinline{String} is
disjoint to others.
% Ensured by Ceylon's type system, \lstinline{v} of type \lstinline{Void} can 
% either be an \lstinline{Object} or \lstinline{Nothing}. 
Thurs there is no need to use the order of branches to decide
a program's behavior, and users can totally ignore the order.
Take the previous function \lstinline{isstudent2} as an example.
For its definition to be accepted by Ceylon,
\lstinline{Person} and \lstinline{Student} cannot be subclass of
each other, assuming they are classes.
Since Ceylon does not allow multiple inheritance on class,
the restriction is sufficient to prevent the incoherent scenario.
In the runtime, type of the input must be a subtype of either \lstinline{Person} 
or \lstinline{Student}. According to the subtyping relation,
one branch will be chosen and the input term will be cast
and substituted in.
%
Besides primitive types and classes, 
distinct members in an enumerated type are also viewed as disjoint.
% switch/case also handles other types as long as they are disjoint.
% Some primitive types like \lstinline{Integer}, \lstinline{Char}, and 
% \lstinline{String} are disjoint to others.
%
Forcing all cases in one switch/case expression to be disjoint makes them
interchangeable.
That could reduce confusion of program readers and eliminate a potential
source of bugs.

\begin{lstlisting}
	void <A> printA(A|Null x) {
		switch (x)
		case (is A) { print("an A"); }
		case (is Null) { print("nothing"); }
	}
\end{lstlisting}

For instance, one might expect the second branch to be executed if the above
function takes \lstinline{null} as input.
However, without the disjointness constraint, \lstinline{A} might be nullable itself.
An example is using \lstinline{Integer|Null} to denote a number or infinity
as a division result.
In that case the second branch is always shadowed by the first one.
Note that swapping the two cases cannot prevent all unexpected behaviors.
When the outside \lstinline{Null} has a different meaning other than infinity,
they are not distinguishable, and the branch designed for it will take them all.
% And instead of \lstinline{print}, one might call a function that takes care
% of \lstinline{Nothing} and the other component in \lstinline{A}.
This problem is even more serious when the language implements some syntax
sugar or encode other features via the switch/case construct.
It could be convenient to have a simple construct for unwrapping \lstinline{A|Null} without writing down the full switch/case expression.
But once the details are hidden, users may not be able to image that how does
the order affects the program, as we discussed above.  
On the other hand, with some requirements on the argument and return types, it is 
straightforward to support function overloading by combining them in a switch/case expression.
But it will be hard to find a fair criteria to decide which one
should be prioritized, if two functions can overlap.
% https://github.com/ceylon/ceylon-spec/issues/50
% https://github.com/ceylon/ceylon-spec/issues/65

\begin{lstlisting}
void printAfterPlusOne(Integer|String x) {
	switch (x)
	case (is Integer|Float) { print(x+1); }
	case (is String) { print("String:"+x); }
}
\end{lstlisting}

Although the list of cases must be exhaustive, it does need to
strictly follow the input's type.
For example, in the above function, \lstinline{x} can only be an integer
or a string. But it is not harmful for the first branch to expect
a term of \lstinline{Integer|Float}, since any integer can have such
type.
In general, a term of type $A$ is always assignable to any supertype of $A$.
But in Ceylon, the checking of assignability is not complete to
subtyping.
Although the subtyping relation holds between \lstinline{v}'s
(declarative) type and \lstinline{Integer}, \lstinline{v}
is not assignable to \lstinline{Integer}, and the following program
cannot be accepted by Ceylon's compiler.
% https://try.ceylon-lang.org/#

\begin{lstlisting}
	< Character | Integer > & < String | Integer > v = 100;
	switch (v)
	case (is Integer) { print("Integer: ``v``"); }
\end{lstlisting}

To dig deeper for disjointness, it often comes to the the concept
of \emph{bottom type}.
Bottom type is a subtype of all types, in contrast to the top type.
Certainly the bottom type has no value.
In Ceylon, it is called \lstinline{Nothing}, representing the empty set.
%
With the existence of subtyping, a branch in switch/case expression
can take a term of a subtype of its expected type.
If the term's type is also a subtype of the other branch's expected
type, type information will be insufficient to disambiguate.
To prevent this, disjoint types cannot have any common subtype which has
inhabited values.
For two disjoint types $[[A]]$ and $[[B]]$, their intersection $[[A/\B]]$
is naturally a common subtype, and therefore must be equivalent to
the bottom type \lstinline{Nothing}.

\begin{comment}
Let us call them \emph{bottom-like types}.
% Ceylon treat such types specially.
In a switch construct, every case has a type annotation.
In Ceylon, this type annotation is not allowed to be bottom-like.
The compiler uses this type to narrow the static type of the input term
in switch/case, and rejects the program if the resultant type is equivalent
to \lstinline{Nothing}.

\begin{verbatim}
	Known problem in Ceylon:
	(Int | String) & (Int | Char) & (Char | String) is not treated as a bottom-like type
	
	Potential reason:
	It defines bottom-like as:
	Bottom-like A\& B ::= A * B
	And it doesn't recognize intersection of unions and union of intersection,
	which is the oopsla18 paper claims to have.
	
	Possible consequences:
	- [harmless] we can write a case branch for such type.
	- The definition of disjointness might be not so presice
	
	
	alias IoC => Integer | Character;
	alias IoS => Integer | String;
	alias CoS => Character | String;
	alias Bot => IoC & IoS & CoS;
	// Ceylon does not treat Bot as Bottom-like type
	
	void test(Integer | Bot v){
		switch (v)
		case (is Integer) {}
		case (is Bot) {}
	}
	void test2(Integer | Integer&Character v){
		switch (v)
		case (is Integer) {}
		//    case (is Integer&Character ) {}
	}
	
	
	< Character | Integer > & < String | Integer > v2 = 100;
	switch (v2)
	// case (is Character | String ) { print("String: ``v2``"); }
	case (is Integer) { print("Integer: ``v2``"); }
	
	// <Character|Integer>&<String|Integer> is covered by, but not assignable to, the type Integer (explicitly narrow assigned expression using of Integer)
	
	
	// accepted
	alias IntLike =>  < Character | Integer > & < String | Integer >;
	IntLike v1 = 100;
	switch (v1)
	case (is Integer) { print("Integer: ``v1``"); }
	
	// rejected
	// Specified expression must be assignable to declared type: the assigned type <Character|Integer>&<String|Integer> is covered by, but not assignable to, the type Integer (explicitly narrow assigned expression using of Integer)
	 < Character | Integer > & < String | Integer > v2 = 100;
	switch (v2)
	case (is Integer) { print("Integer: ``v2``"); }
	
	
	// accepted
	alias Bot2 => Nothing;
	Nothing | Integer v3 = 1;
	switch (v3)
	case (is Bot2) { print("test2"); }
	case (is Integer) { print("test2"); }
	
	// rj
	// Narrows to bottom type Nothing: Nothing has empty intersection with Nothing|Integer
	 Nothing | Integer v3 = 1;
	switch (v3)
	case (is Nothing) { print("test2"); }
	case (is Integer) { print("test2"); }
	
	
	alias IoC => Integer | Character;
	alias IoS => Integer | String;
	alias CoS => Character | String;
	alias Bot => IoC & IoS & CoS;
	
	// ac
	<IoC & IoS & CoS> | Integer v3 = 1;
	switch (v3)
	case (is Bot) { print("test"); }

	case (is Bot2) { print("test"); }
	case (is Integer) { print("test"); }
	
	// rj
	// Narrows to bottom type Nothing: IoC&IoS&CoS has empty intersection with IoC&IoS&CoS|Integer
	switch (v3)
	case (is IoC & IoS & CoS) { print("test"); }
	case (is Integer) { print("test"); }
	
\end{verbatim}
\end{comment}

\bruno{
	Introduce the Ceylon approach with examples; introduce the idea of
	disjointness informally; contrast with existing constructs.
	Try to motivate the adapotion of disjointness (this should be discussed
	in the Ceylon documentation to some extent).\\
	Here it is important to give representative examples (look at documentation
	and/or other online resources), including examples
	that demonstrate how Ceylon uses union types to model overloaded functions.
}

\subsection{Our work}

\snow{Some concerns: does the addition of intersection types and distributivity
	really increase the expressiveness of the system?
	Currently the only interesting intersection type has the form of (A->B) \& (C->D),
	like (int->bool)\&(bool->int).
	But there is no way to construct a term of such type.
	The closest thing is $\lambda$ x.x, which has (int->int)\&(bool->bool).}

\bruno{
	Introduce our work, setting the goal to study the construct formally.
	Connect with the work on disjoint intersection types, which also
	employs a notion of disjointness, but for intersection. Explain that
	what is needed is a dual notion of disjointness.\\
	Introduce the first calculus, and explain that it is directly inspired
	by a dual notion of discjointness.\\
	Introduce the second calculus and identify a technical challenge with
	disjointness: the addition of intersection types breaks the previous
	notion of disjointness. Introduce the novel way to find disjoint types.\\
	Summarize/mention key results: type-safety; soundness/completeness of
	disjointness; determinism.\\
	Perhaps here it is also useful to identity, together with Baber, what
	were the most challenging aspects in the formalization, and maybe
	highlight these.
}

%%% Local Variables:
%%% mode: latex
%%% TeX-master: "../paper"
%%% org-ref-default-bibliography: "../paper.bib"
%%% End:
\section{Disjoint Union Types}
\label{sec:union}

\baber{Disjoint Union Types without Intersection types will be discussed in this section.}

\begin{align*}
&Type &A, B&::= [[Top]] ~ | ~[[Bot]]~|~ Int ~| ~ [[A->B]]~ |~ [[A \/ B]]\\
&Expr &e &::= x ~|~ n ~| ~ e:A ~|~[[\x.e]] ~ | ~ e1e2 ~|~ [[typeof e as A e1 B e2]]\\
&PExpr & p &::= n ~|~ [[\x.e : A->B]] \\
&Value &v &::= p : A \\
&Context & [[G]] &::= empty~ |~ [[G , x : A]]
\end{align*}

     {\renewcommand{\arraystretch}{1.5}
     \begin{center}
     \begin{tabular}{|lcll|}
       \hline
      DisjSpec & A $*_s$ B & ::= & ~$\forall C, ~ Ord ~ C ~ \rightarrow \neg [[C <: A /\ B]]$\\
       \hline
      DisjAlgo & A $*_a$ B & ::= & ~ $(FindSubtypes ~ A) ~ `inter` ~ (FindSubtypes ~ B) = []$ \\
       \hline
     \end{tabular}
     \end{center} }
\section{Intersection Types}
\label{sec:intersection}

\baber{Disjoint Union Types with Intersection Types will be discussed in this section.}

\begin{align*}
&Type &A, B&::= [[Top]] ~ | ~[[Bot]]~|~ Int ~| ~ Bool ~ |~ Str ~| ~ [[A->B]]~ |~ [[A \/ B]] ~ |~ [[A /\ B]]\\
&Expr &e &::= x ~|~ n ~| ~ b ~ | ~ s ~| ~ e:A ~|~[[\x.e]] ~ | ~ e1e2 ~|~ [[typeof e as A e1 B e2]]\\
&PExpr & p &::= n ~| ~ b ~ | ~ s ~|~ [[\x.e : A->B]] \\
&Value &v &::= p : A \\
&Context & [[G]] &::= empty~ |~ [[G , x : A]]
\end{align*}

     {\renewcommand{\arraystretch}{1.5}
     \begin{center}
     \begin{tabular}{|lcll|}
       \hline
      DisjSpec & A $*_s$ B & ::= & ~$\forall C, ~ Ord ~ C ~ \rightarrow \neg [[C <: A /\ B]]$\\
       \hline
      DisjAlgo & A $*_a$ B & ::= & ~ $(FindSubtypes ~ A) ~ `inter` ~ (FindSubtypes ~ B) = []$ \\
       \hline
     \end{tabular}
     \end{center} }

\begin{lstlisting}[language=Haskell]
Fixpoint FindSubtypes (A: typ) :=
    match A with
    | t_top         => [t_int; t_bool; t_str; t_arrow t_top t_bot; t_top]
    | t_bot         => []
    | t_int         => [t_int]
    | t_bool        => [t_bool]
    | t_str         => [t_str]
    | t_arrow A1 B1 => [t_arrow t_top t_bot]
    | t_union A1 B1 => (FindSubtypes A1) `union` (FindSubtypes B1)
    | t_and A1 B1   => (FindSubtypes A1) `inter` (FindSubtypes B1)
    end.
\end{lstlisting}
\section{Related Work}
\label{sec:related}

\begin{comment}
\begin{itemize}
	\item {Union Types    (discussing work on union types and elimination constructs for union types) <-- Need to coordinate with Ningning/Snow so that there’s not much overlap there. If the Overview discusses some of these in detail, the RW does not need to have much text in it.}
	\item {Overloading (more general work on overloading, including type classes, approaches to overloading based on intersection types, etc…)}
	\item{Disjointness and Disjoint Intersection Types}
\end{itemize}
\end{comment}

\paragraph{Union types}
\begin{comment}
Set-theoretic unions have sound theory and extensively studied in
mathematics. Set-theoretic unions correspond to union types or
disjoint union types in programming languages. Disjoint union types
are also called sum types or variants.  Constructors are explicitly
labeled in disjoint union types and expressions are manipulated using
corresponding labels. Few other interesting calculi (and this paper)
do not use labels and provide type-based union elimination.
%Developing
%a sound and deterministic union elimination construct for type-based
%union elimination have been a challenge in research community.
\end{comment}
The union type constructor was first introduced by \citet{macqueen1984ideal}.
They proposed a typing rule that eliminates union implicitly.
The rule breaks type preservation under
the conventional reduction strategy of lambda calculus.
\citet{barbanera1995intersection} solved the problem by reducing all
copies of the same redex in parallel.
\citet{dunfield2014elaborating} took another approach to support
mutable references. They restricted the typing rule to only allow single
occurrence of the subterm of union type in the typing expression.
\citet{pierce1991programming} proposed a novel single-branch case construct
for unions. As pointed out by \citet{dunfield2003type}, compared to the single
occurrence approach, the only effect of it is to make the elimination explicit.
\begin{comment}
Moreover, while \citet{dunfield2014elaborating} shows that
subtyping is not necessary for elaboration, it is not obvious how to generalize
elaboration to support subtyping relations such as
\lstinline{Student <: Person} without using the subtyping rule. If the
elaboration were generalized further to support such a subtyping relation, then
a student with type \lstinline{Student | Person} can also be tagged
non-deterministically.
\end{comment}


% \citet{barbanera1995intersection} proposed
% two approaches for type preservation in a calculus with intersection and union
% types.
%However, their union elimination is same as last two and follow
%unrestricted union elimination.

\begin{comment}
\citet{freeman1991refinement} studied union types along with intersection types
in setting of refinement types for ML. Main focus of their work is to infer
more precise types of expressions, which they call
refinement types. Their work is targeted to contribute for the types
in ML and not for the expressions. Therefore, they did not define
expressions and dynamic semantics. On the contrary, our work provides
a complete calculus with type sound dynamic semantics.
\bruno{and ... Do they have an elimination construct for unions?
  If so how is that related to our work?}.
\baber{I added few more lines in Freeman citation.}
\end{comment}
Union types and elimination constructs based on types are
widely used in the context of XML processing languages~\cite{hosoya2003xduce,benzaken2003cduce}.
Generally speaking the elimination constructs in those
languages offer a first-match semantics,
where cases can overlap and reordering the cases may change the semantics of the program.
This is in contrast to our disjointness based approach.
\citet{hosoya2003xduce} studied union types
in the XDuce programming language. XDuce offers a novel feature of
so-called regular expression types. Pattern matching can be on
expressions and types in XDuce.  Expressions are considered as special
cases of types.  CDuce
\cite{benzaken2003cduce} is an extension of XDuce. CDuce has improved
pattern matching but it also
follows first-match semantics. Work on the more foundational aspects
of CDuce, and in particular on \emph{semantic subtyping}~\cite{frisch2002semantic}
and set-theoretic types,
also employs a form of first-match semantics elimination construct, though in a different form.
In particular, work by \citet{castagna2005gentle,castagna2017gradual}
proposes a conditional construct that can test whether a value matches a type.
If it matches then the first branch is executed and the type for the value is refined.
Otherwise, the second branch is executed and the type of the value is refined to be
the negation of the type (expressing that the value does not have such type).
\begin{comment}
\citet{fallside2001xml} studied union
types in markup language, but with a restriction of disjoint top-level
labels which is different from our work on
disjointness on types.\bruno{In other words, for the last work, the unions
are labelled (i.e. the cases are not based on types)?}

\cite{frisch2002semantic} studied union types and intersection types
with semantic subtyping and provide the theoretical basis for CDuce programming language.
However, their case expression (they call it \emph{match}) uses explicit
tags to select a particular branch. Whereas, we propose dynamic type-based
case construct.
\cite{castagna2005gentle} extended this work with type-based case construct
in the context of XDuce programming language. Main motivation of their
work is to elaborate the significance of semantics subtyping.
However, their underlying technique to select the branch in case construct
relies on negation types. We, on the other hand, use disjointness
and static type of the expression. \baber{they also use an extra variable.}
%\cite{frisch2008semantic}.
%\baber{more text.}
\citet{castagna2017gradual} studied
set-theoretic union types, intersection types and negation types
in gradual typing setting. They also propose a dynamic type-based
construct for set-theoretic union elimination which is closely
related to our work.
However, their underlying technique to select branches in case
expression is different from ours.
In particular, negation types play fundamental role in their
typing construct for the case expressions.
Whereas, our work relies on disjointness along with a notion of
static types to to select a particular branch.
They also do not provide dynamic semantics for their source language.
Whereas, we provide a type sound dynamic semantics for \cal.
\baber{no variables in their case construct?}

\baber{below text will be removed for Castagna's work.}
\bruno{Castagna did alot of work
  on union types before this one, so I'm not sure why we are mentioning only this work
  of Castagna, instead of talking about previous work where the ideas were introduced.}
\baber{Castagna's work includes CDuce, which is mentioned. I added semantic subtyping
as well now. Should we add more Castagna's work? I think his above mentioned work
is relevant to our work.}
They give a dynamic type-based cast for union elimination. But they do not
have disjointness constraint on case branches and they check the type
only for first case branch\bruno{Not a very good explanation: you get negated types for the branches}.
\end{comment}

\citet{muehlboeck2018empowering} give a general framework for subtyping
with intersection and union types. They illustrate the significance of
their framework using the Ceylon programming language.
The main objective of their work is to define a generic framework for
deriving subtyping algorithms for
intersection and union types in the presence of various distributive subtyping rules.
For instance, their framework could be useful to derive an algorithmic
formulation for the subtyping relation presented in Figure~\ref{fig:discussion:ds}.
They also briefly cover disjointness in their work. As part of their framework, they
can also check disjointness given some disjointness axioms. For instance,
for \name, such axioms could be similar to \rref{ad-mtmr} or \rref{ad-intl}
in Figure~\ref{fig:union:disj-typ}.
However, they do not have a formal
specification of disjointness. Instead they assume that some sound specification
exists and that the axioms respect such specification.
If some unsound axioms are given to their framework (say $[[Int * Int]]$) this
would lead to a problematic algorithm for checking disjointness.
In our work we provide specifications for disjointness together
with sound and complete algorithmic formulations.
In addition, unlike us,
they do not study the semantics of disjoint switch expressions,
although they mention some Ceylon examples using such expressions.

\paragraph{Occurrence Typing}
Occurence typing or flow typing \cite{tobin2008design} specializes or refines
the type of variable in a type test. For example, with occurrence typing we
could have:

\begin{lstlisting}
Int occurence(Int | String val) {
  if (val is Int) { return val+1; }
  else { return toInt(val)+2; }
}
\end{lstlisting}

\noindent In such code, \lstinline{val} initially has type $[[Int \/ String]]$.
The conditional checks if the \lstinline{val} is of type $[[Int]]$.
If the condition succeeds, it is safe to assume that \lstinline{val} is of type $[[Int]]$,
and the type of \lstinline{val} is refined in the branch to be \lstinline{Int}.
Otherwise, it is safe to assume that \lstinline{val} is of type $[[String]]$, in the
other branch (and the type is refined accordingly).
The motivation to study occurrence typing was to introduce typing in dynamically
typed languages.
Occurrence typing was further studied by \cite{tobin2010logical},
which resulted into the development of Typed Racket.
Variants of occurrence typing are nowadays employed in mainstream languages
such as TypeScript or Flow.
\cite{castagna2019revisiting} proposed a more general formulation of
occurence typing. He extended occurrence typing to refine the type of
generic expressions, not just variables. He also studied the combination
with gradual typing. Occurrence typing provides an alternative
means to eliminate union types using a first-match semantics. That is the
order of the type tests determines the priority that is given to the various
types being tested.

\paragraph{Disjoint Intersection Types}
\citet{pottinger1980type} and \citet{coppo1981functional} initially
studied intersection types in programming languages
literature. Forsythe~\cite{reynolds1988preliminary} is the first
programming language to have intersection types, but it did not
have union types.  Disjoint intersection types were first
studied by \citet{oliveira2016disjoint} in the $\lambda_{i}$ calculus
to give a coherent calculus for intersection types with a merge
operator. The notion of disjointness used in \cal, discussed in \Cref{sec:union},
is inspired by the notion of disjointness of $\lambda_{i}$. In essence,
disjointness in \cal is the dual notion: while in $\lambda_{i}$ two types
are disjoint if they only have top-like supertypes, in \cal two types
are disjoint if they only have bottom-like supertypes.

None of calculi with disjoint intersection types~\cite{oliveira2016disjoint,bi_et_al:LIPIcs:2018:9227,alpuimdisjoint} in the literature
includes union types. One interesting discovery of our work is that the
presence of both intersections and unions in a calculus can affect disjointness.
In particular, as we have seen in Section~\ref{sec:inter}, adding intersection types
required us to change disjointness. The notion of disjointness that was
derived from $\lambda_{i}$ stops working in the presence of intersection types.
Interestingly, a similar issue happens when union types are added to
a calculus with disjoint intersection types. If disjointness of two types \lstinline{A}
and \lstinline{B} is defined to be that such types can only have top-like types,
then adding union types immediately breaks such definition.
For example, the types $[[Int]]$ and $[[Bool]]$ are disjoint but, with union
types, $[[Int \/ Bool]]$ is a common supertype that is not top-like.
We conjecture that, to add union types to disjoint intersection types,
we can use the following definition of disjointness:

\begin{definition}
\label{def:related:disj}
  A $*_s$ B $\Coloneqq$ $\forall$ C, \ $[[ordinary C]]$ \ $\rightarrow$ \ $\neg$ \ ($[[A <: C]]$ and $[[B <: C]]$).
\end{definition}

\noindent which is, in essence, the dual notion of the definition presented in
Section~\ref{sec:inter}. Under this definition $[[Int]]$ and $[[Bool]]$ would
be disjoint since we cannot find an common ordinary supertype (and $[[Int \/ Bool]]$
is a supertype, but it is not ordinary). Furthermore, there should be a
dual notion to LOS, capturing the greatest ordinary supertypes. Moreover,
if a calculus includes both disjoint switches and a merge operator,
then the two notions of disjointness must coexist in the calculus.
This will be an interesting path of exploration for future work.

\paragraph{Overloading}
Union types also provide a form of function overloading or ad-hoc
polymorphism using the switch and type-based case analysis. A programmer
may define the argument type to a union type. By
using type-based case analysis, it is possible to execute different code
for each specific type of input.  Intersection types have also been
studied for function overloading. For example a function with type
$[[Int -> Int /\ Bool -> Bool]]$ can take input values either of type
$[[Int]]$ or $[[Bool]]$.  In such case, we return either $[[Int]]$ or $[[Bool]]$
depending upon the input type.  Function overloading has been studied
in detail in the literature by \cite{castagna1995calculus},
\cite{cardelli1985understanding}, \cite{stuckey2005theory} among
others.  \citet{wadler1989make} studied type classes as an alternative way
to provide overloading based on parametric polymorphism.

%%% Local Variables:
%%% mode: latex
%%% TeX-master: "../paper"
%%% org-ref-default-bibliography: "../paper.bib"
%%% End:

\section{Conclusion and Future Work}
\label{sec:conclusion}

We develop a union calculus (\cal) for type-based union elimination. 
We also add intersection types in
\cal. The calculus is proved to be sound, complete, type-safe and deterministic.
Such calculi have a variety of practical applications such as in switch expressions
and function overloading. We plan to extend \cal for practical programming languages.

A major future research direction for \cal is to study it with merge operator.
Intersection types are of significant interest with merge operator. But it is non-trivial
to naivly add the merge operator for a calculus with union and intersection types.
One major problem with naive merge operator is that merge operator induces (in)coherence problem.
Therefore, special care has to be taken to study \cal with merge operator.

\bibliography{paper}


\end{document}