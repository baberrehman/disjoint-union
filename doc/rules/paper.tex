\documentclass[a4paper,UKenglish,cleveref, autoref]{lipics-v2019}

%% Some recommended packages.
\usepackage{booktabs}   %% For formal tables:
                        %% http://ctan.org/pkg/booktabs
\usepackage{subcaption} %% For complex figures with subfigures/subcaptions
                        %% http://ctan.org/pkg/subcaption
\usepackage{longtable}
\usepackage{xspace}
\usepackage{proof}

%% Bibliography style
\bibliographystyle{plainurl}% the mandatory bibstyle

\title{Disjoint Union Types}

\nolinenumbers

%\authorrunning{B.\, Oliveira and B.\, Rehman}%TODO mandatory. First: Use abbreviated first/middle names. Second (only in severe cases): Use first author plus 'et al.'


\author{Bruno C. d. S. Oliveira}{The University of Hong Kong, Hong Kong, China}
{bruno@cs.hku.hk}{}{}

%\author{Baber Rehman}{The University of Hong Kong, Hong Kong, China}
%       {brehman@cs.hku.hk}{}{}

\begin{comment}
\funding{Funded by Hong
  Kong Research Grant Council projects number 17210617 and 17209519.}

\supplement{https://github.com/baberrehman/coq-duotyping}

\authorrunning{Oliveira et al.}%TODO mandatory. First: Use abbreviated first/middle names. Second (only in severe cases): Use first author plus 'et al.'

\Copyright{Bruno C. d. S. Oliveira}%TODO mandatory, please use full first names. LIPIcs license is "CC-BY";  http://creativecommons.org/licenses/by/3.0/

\ccsdesc[100]{Software and its engineering~Object oriented languages}%TODO mandatory: Please choose ACM 2012 classifications from https://dl.acm.org/ccs/ccs_flat.cfm
\keywords{DuoTyping, OOP, Duality, Subtyping, Supertyping}%TODO mandatory; please add comma-separated list of keywords
\acknowledgements{We thank the anonymous reviewers for their helpful comments.}%optional
\end{comment}

% AMS packages
\usepackage{amsmath}
\usepackage{amssymb}

\usepackage{mathtools}
\usepackage{mdwlist}
\usepackage{pifont}


% Miscellaneous
\usepackage{paralist}
\usepackage{graphicx}
\usepackage{epstopdf}
\usepackage{float}
\usepackage{longtable}
\usepackage{multirow}
\usepackage{lscape}


% Revision tools
\usepackage{xspace}
\usepackage{comment}

%\newcommand{\hl}[2][gray!40]{\colorbox{#1}{#2}}
\newcommand{\hlmath}[2][gray!40]{\colorbox{#1}{$\displaystyle#2$}}
\newcommand{\otthl}[2][gray!40]{ \colorbox{#1}{$\displaystyle#2$}}


% Graphs
\usepackage{tikz}
\usetikzlibrary{matrix}
\usetikzlibrary{arrows,automata}
\usetikzlibrary{positioning}


% Hyper links
\usepackage{url}
\usepackage{
  nameref,%\nameref
  hyperref,%\autoref
}
\usepackage[capitalise]{cleveref}

\usepackage{thmtools, thm-restate}

\usepackage[misc]{ifsym}

\declaretheorem[name=Lemma,
  % numberwithin=section,
  refname={Lemma,Lemmas},
  Refname={Lemma,Lemmas}]{clemma}

\declaretheorem[name=Theorem,
  % numberwithin=section,
  refname={Theorem,Theorems},
  Refname={Theorem,Theorems}]{ctheorem}

\declaretheorem[name=Observation]{observation}

\declaretheorem[name=Proof]{Proof}


\usepackage{rotating}

\usepackage{ottalt}

\renewcommand\ottaltinferrule[4]{
  \inferrule*[right=\scriptsize{#1}]
    {#3}
    {#4}
}

\inputott{ott-rules}
\input{paper_utility.tex}
%\input{sections/formal.tex}

\newcommand\mynote[3]{\textcolor{#2}{#1: #3}}
%\newcommand\mynote[3]{}
\newcommand\bruno[1]{\mynote{Bruno}{red}{#1}}
\newcommand\baber[1]{\mynote{Baber}{blue}{#1}}

\begin{document}

\maketitle


%% -- Starting Point --

\section{Rules}
\label{sec:rules}

%\infer[tred-int]{ n \rightarrow_{Int} n}{ }

%\infer[tred-fun]{ n \rightarrow_{Int} n}{ }

\begin{figure}[t]
  \begin{small}
    \centering
    \begin{tabular}{lrcl} \toprule
      Types & $[[A]], [[B]]$ & $\Coloneqq$ & $ [[Btm]] \mid [[int]] \mid [[A -> B]] \mid [[Or A B]]  $ \\
      Terms & $[[e]]$ & $\Coloneqq$ & $ [[x]]  \mid [[n]]  \mid [[\x . e : A -> B]] \mid [[e1 e2]] \mid [[e:A]] \mid typeof \ e \ as \ \{x:[[A]][[->]][[e1]], x:[[B]][[->]][[e2]]\} $ \\
      Values & v & $\Coloneqq$ & $  [[n]] \mid [[\x : A . e]] $ \\
      Context & $[[GG]]$ & $\Coloneqq$ & $ [[empty]]  \mid [[GG]] , [[x : A]] $ \\
      \bottomrule
    \end{tabular}
  \end{small}\\

  \bigskip

%    \begin{tabular}{|l|l|}
%     \hline
%      $CheckType (v)$  & \\
%      & $CheckType([[n]]) \ = \ [[int]]$ \\
%      & $CheckType([[\x . e : A -> B]]) \ = \ [[A -> B]]$ \\
%       \hline
%    \end{tabular}

  \bigskip

  \begin{small}
    \centering
    \begin{tabular}{lrcl} \toprule
      DisSpec & $[[A]] * [[B]]$ & $\Coloneqq$ & $ {\forall} C, [[C sub A]] \wedge [[C sub B]] \rightarrow  \rfloor [[C]] \lfloor $ \\
      \bottomrule
    \end{tabular}
  \end{small}

  \bigskip

  \begin{small}
    \centering
    \drules[bl]{$ [[botl C ]] $}{Bottom Like}{btm, or}
  \end{small}

  \begin{small}
    \centering
    \drules[ad]{$ [[A *A B ]] $}{Algorithmic Disjointness}{btml, btmr, intl, intr, orl, orr}
  \end{small}

  \begin{small}
    \centering
    \drules[ts]{$ [[A sub B ]] $}{Subtyping}{btm, int, arrow, ora, orb, orc}
  \end{small}

  \begin{small}
    \centering
    \drules[tss]{$ [[A subsub B ]] $}{Subsub}{refl, arrow, ora, orb}
  \end{small}

 \caption{Disjoint Union Type : Syntax}
  \label{fig:disunion}
\end{figure}

\begin{figure}[t] 
   \begin{small}
    \centering
    \drules[t]{$ [[GG |- e => A ]] $}{Bidirectional Typing}{var, int, abs, app, sub, anno, typeof}
   \end{small}

   \begin{small}
    \centering
    \drules[red]{$ [[e --> e' ]] $}{Reduction}{beta, appl, appr, anno, annov, typeof, typeofvl, typeofvr}
  \end{small}

   \begin{small}
    \centering
    \drules[tred]{$ [[v]] [[tred A]] $}{Type Reduction}{int, beta, orl, orr}
  \end{small}

  \caption{Disjoint Union Type : Semantics}
  \label{fig:disunion:semantics}
\end{figure}

%%% Local Variables:
%%% mode: latex
%%% TeX-master: "../paper"
%%% org-ref-default-bibliography: "../paper.bib"
%%% End:


%% Appendix

%\newpage
%\appendix
%\input{sections/appendix.mng}

\end{document}
