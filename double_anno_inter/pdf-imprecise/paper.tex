\documentclass[a4paper]{article}

\usepackage{geometry}
\geometry{left=2.5cm,right=2.5cm,top=2.5cm,bottom=2.5cm}

% Basics
\usepackage{fixltx2e}
\usepackage{url}
\usepackage{fancyvrb}
\usepackage{mdwlist}  % Miscellaneous list-related commands
\usepackage{xspace}   % Smart spacing
\usepackage{supertabular}

% https://www.nesono.com/?q=book/export/html/347
% Package for inserting TODO statements in nice colorful boxes - so that you
% won’t forget to fix/remove them. To add a todo statement, use something like
% \todo{Find better wording here}.
\usepackage{todonotes}

%% Math
\usepackage{amsmath}
\usepackage{amsthm}
\usepackage{amssymb}
\usepackage{bm}       % Bold symbols in maths mode

% http://tex.stackexchange.com/questions/114151/how-do-i-reference-in-appendix-a-theorem-given-in-the-body
\usepackage{thmtools, thm-restate}

%% Theoretical computer science
\usepackage{stmaryrd}
\usepackage{mathtools}  % For "::=" ( \Coloneqq )

%% Code listings
\usepackage{listings}

%% Font
\usepackage[euler-digits,euler-hat-accent]{eulervm}


\usepackage{ottalt}

\renewcommand\ottaltinferrule[4]{
  \inferrule*[right={#1}]
    {#3}
    {#4}
}


% Ott includes
\inputott{ott-rules}


\title{Rules Version 2 (Disjoint Union Type)}
\author{Baber}
\begin{document}

\maketitle

\begin{align*}
&Type &A, B&::= [[Top]] ~ | ~[[Bot]]~|~ Int ~|~ [[A->B]]~ |~ [[A \/ B]] ~ |~ [[A /\ B]]\\
&Expr &e &::= x ~|~ n ~|~ e:A ~|~[[\x.e]] ~ | ~ e1e2 ~|~ [[switch e A e1 B e2]]\\
&PExpr & p &::= n ~|~ [[\x.e : A->B]] \\
%&RExpr & r &::= e1e2 ~|~ [[typeof e as A e1 B e2]] \\
%&UExpr & u &::= r ~|~ p ~ |~ u : A \\
&Value &v &::= p ~|~ [[\x.e]] \\
&Context & [[G]] &::= empty~ |~ [[G , x : A]]
\end{align*}

%\begin{align*}
%&Isomorphic & A \sim B & ::= [[A <: B]] \wedge [[B <: A]]
%\end{align*}


%\begin{align*}
%&BottomLikeSpec & C & ::= (\forall A ~ B, ~ [[A /\ B]] \sim C \rightarrow \neg ( [[ A <: B ]] ) \wedge \neg ( [[ B <: A ]] )) \vee ([[C <: Bot]])
%\end{align*}

%\begin{align*}
%&DisjSpec & A * B & ::= \forall C, [[C <: A]] \wedge [[C <: B]] \rightarrow  \rfloor [[C]] \lfloor
%\end{align*}


%\begin{table}
     {\renewcommand{\arraystretch}{1.5}
     \begin{center}
     \begin{tabular}{|lcl|}
       \hline
  A $*_s$ B & $\Coloneqq$ & $\forall$ C, \ $[[ordinary C]]$ \ $\Longrightarrow$ \ $\neg$ \ ($[[C <: A]]$ and $[[C <: B]]$). \\
       \hline
    A $*_a$ B & $\Coloneqq$ & $ [[findsubtypes A]] \cap [[findsubtypes B]] $ = $\{\}$. \\
       \hline
     \end{tabular}
     \end{center} }
%\end{table}

    \centering
    {\renewcommand{\arraystretch}{1.2}
    \begin{tabular}{|lcl|}
      \multicolumn{3}{c}{Lowest Ordinary Subtypes (LOS) $[[findsubtypes A]]$} \\
      \hline
     $[[findsubtypes Top]]$ & = & $\{ [[Int]], [[Top -> Bot]]\}$  \\
     $[[findsubtypes Bot]]$ & = & $\{\}$  \\
     $[[findsubtypes Int]]$ & = & $\{ [[Int]] \}$  \\
     $[[findsubtypes A -> B]]$ & = & $\{ [[Top -> Bot]] \}$  \\
     $[[findsubtypes A \/ B]]$ & = & $ [[findsubtypes A]] \cup [[findsubtypes B]] $\\
     $[[findsubtypes A /\ B]]$ & = & $ [[findsubtypes A]] \cap [[findsubtypes B]] $\\
      \hline
    \end{tabular} }
    
\bigskip

\ottdefnsOrdinary

\ottdefnsSubtyping

\ottdefnsTyping

\ottdefnsReduction


\end{document}
